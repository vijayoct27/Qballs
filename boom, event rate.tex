%\documentclass[10pt, twocolumn]{article}
%\documentclass[11pt]{article}
%\documentclass[twocolumn,showpacs,preprintnumbers,amsmath,amssymb,prl, superscriptaddress]{revtex4}
%\documentclass[twocolumn, preprintnumbers,amsmath,amssymb,prd, superscriptaddress]{revtex4}
\documentclass[preprintnumbers,amsmath,amssymb,prd,superscriptaddress]{revtex4}
%\documentclass[10pt, preprint,showpacs,preprintnumbers,amsmath,amssymb, superscriptaddress]{revtex4}
%\documentclass[11pt, prd,preprintnumbers,amsmath,amssymb, superscriptaddress]{revtex4}
%\documentclass[11pt, prd,preprintnumbers, amsmath,amssymb, superscriptaddress, nofootinbib, hyperref]{revtex4}

\usepackage{latexsym}
\usepackage{amssymb}
\usepackage{epsfig,amsmath,graphics}
\usepackage{epstopdf}
\usepackage{verbatim}
\usepackage{wasysym}
\usepackage{hyperref}
\usepackage{feynmp-auto} % feynman diagrams
%\usepackage{subfig}
\usepackage[utf8]{inputenc}
\usepackage{xpatch}
\usepackage{xcolor}
\usepackage{mathtools}
\hypersetup{
    colorlinks,
    linkcolor={red!80!black},
    citecolor={green!60!black},
    urlcolor={blue!60!black}
}
\usepackage{appendix}

\newcommand{\Ez}{\mathcal{E}_0}
\newcommand{\Eboom}{\mathcal{E}_\text{boom}}
\newcommand{\OO}{\mathcal{O}}
\newcommand{\LL}{\mathcal{L}}
\newcommand{\HH}{\mathcal{H}}
\newcommand{\TeV}{\text{TeV}}
\newcommand{\GeV}{\text{GeV}}
\newcommand{\MeV}{\text{MeV}}
\newcommand{\keV}{\text{keV}}
\newcommand{\rad}{\text{rad}}
\newcommand{\cm}{\text{cm}}
\newcommand{\angstrom}{\buildrel _{\circ} \over {\mathrm{A}}}
\newcommand{\pslash}{p\hspace{-0.070in}/\,}
\newcommand{\Mpl}{M_{\text{pl}}}
\newcommand{\ket}[1]{\ensuremath{\left|#1\right>}}
\newcommand{\bra}[1]{\ensuremath{\left<#1\right|}}
\newcommand{\braket}[2]{\ensuremath{\left<#1|#2\right>}}
%Large Parentheses
\def\r{\right)}
\def\l{\left(}

\begin{document}

%\preprint{APS/123-QED}

Thus for a heating event characterized by its $L_0$, $\Ez$, and $T_0 \gtrsim T_f$, there is a \emph{boom condition}:
\begin{align}
    \label{eq:energy_boom_condition}
    \Ez \gtrsim
    \Eboom \cdot \text{max}\left\{1, \frac{L_0}{\lambda_T}\right\}^3.
\end{align}

\subsection{Transits}

\paragraph{Boom Condition.}
The energy deposited during a continuous heating event such as a DM transit is best described in terms of a linear energy transfer $(dE/dx)_\text{LET}$, the kinetic energy of SM particles produced per distance traveled by the DM.
If these products have a heating length $L_0$ then the relevant energy deposit must at minimum be taken as the energy transferred over the transit distance $L_0$.
Of course, we can always choose to consider energy deposits over a longer segment of the DM trajectory.
Importantly, as per the general condition \eqref{eq:energy_boom_condition} such a deposition is \emph{less} explosive unless $L_0$ is smaller than the trigger size $\lambda_T$.
Thus, we consider the energy deposited in a transit over the larger of these two length scales.
Assuming the energy of the DM is roughly constant over this heating event, the boom condition for transit heating is:
\begin{align}
\label{eq:transitexplosion}
  \left( \frac{d E}{d x} \right)_\text{LET} \gtrsim
  \frac{\Eboom}{\lambda_T} \cdot \text{Max}
  \left\{\frac{L_0}{\lambda_T}, 1 \right\}^2.
\end{align}

The above argument sums the individual energy deposits along the DM trajectory as though they are all deposited simultaneously.
This is possible if the DM moves sufficiently quickly so that this energy does not diffuse out of the region of interest before the DM has traversed the region.
We therefore require that the diffusion time $\tau_\text{diff} \approx 10^{-12} ~\text{s}$ across a heated region at temperature $T_f$ be larger than the DM crossing-time:
\begin{align}
  \tau_\text{diff} \sim \frac{L^2}{\alpha(T_f)} \gg
  \frac{L}{v_\text{esc}},
\label{eq:SlowDiffusion}
\end{align}
where $\alpha(T)$ is the temperature-dependent diffusivity, and the DM transits at the stellar escape velocity $v_\text{esc} \approx 10^{-2}$.
This condition is more stringent for smaller regions, so we focus on the smallest region of interest, $L = \lambda_T$.
\eqref{eq:SlowDiffusion} is then equivalent to demanding that the escape speed is greater than the conductive speed of the fusion wave front, $v_\text{cond} \sim \alpha(T_f) / \lambda_T$.
Numerical calculations of $v_\text{cond}$ are tabulated in \cite{Woosley}, and indeed condition \eqref{eq:SlowDiffusion} is satisfied for all WD densities.

\paragraph{Event Rate: Wind Scenario.}
The rate of transit events is given by the flux of DM passing through a WD
\begin{align}
  \Gamma_\text{transit} \sim
  \frac{\rho_{\chi}}{m_\chi} R_\text{WD}^2
  \l\frac{v_\text{esc}}{v_\text{halo}}\r^2 v_\text{halo},
\label{eq:TransitFluxCondition}
\end{align}
where $m_\chi$ is the DM mass, $\rho_\chi$ is the local DM density near the WD, and $R_\text{WD} \approx 4000 ~\text{km}$ is the WD radius.
Here $v_\text{halo} \sim 10^{-3}$ is galactic virial velocity, and the transit rate contains an $\OO(100)$ enhancement due to gravitational focusing.

\paragraph{WD Shielding.}
Runaway fusion only occurs in the degenerate WD interior where thermal expansion is suppressed as a cooling mechanism.
The outer layers of the WD, however, are composed of a non-degenerate gas and it is therefore essential that a DM candidate penetrate this layer in order to ignite a SN.
We parameterize this by a DM stopping power $(dE/dx)_\text{SP}$, the kinetic energy lost by the DM per distance traveled in the non-degenerate layer, and demand that
\begin{align}
\label{eq:CrustCondition}
  \left( \frac{d E}{d x} \right)_\text{SP} \ll
  \frac{m_\chi v^2_\text{esc}}{R_\text{envelope}},
\end{align}
where $R_\text{envelope} \approx 50 ~\text{km}$ is the width of a WD envelope \cite{KippenhahnWeigert}.
Note that the DM stopping power in the non-degenerate layer $(dE/dx)_\text{SP}$ and the linear energy transfer in the degenerate interior $(dE/dx)_\text{LET}$ are possibly controlled by different physics and may have very different numerical values.
In addition, a transit heating event satisfying condition \eqref{eq:CrustCondition} will have negligible energy loss over the parametrically smaller trigger size or heating length $L_0$, validating the boom condition \eqref{eq:transitexplosion}.

\subsection{Collisions and Decays}

\paragraph{Boom Condition.}
For a point-like DM-DM collision or DM decay event releasing particles of heating length $L_0$, ignition will occur if the total energy in SM products satisfies condition~\eqref{eq:energy_boom_condition}.
Such an event will likely result in both SM and dark sector products, so we parameterize the resulting energy in SM particles as a fraction $f_\text{SM}$ of the DM mass.
For non-relativistic DM, the DM mass is the dominant source of energy and therefore $f_\text{SM} \lesssim 1$ regardless of the interaction details, although we may well suspect that $f_\text{SM} \ll 1$ for realistic models.
With this parameterization, a single DM-DM collision or DM decay has a boom condition:
\begin{equation}
\label{eq:coldecay}
  m_\chi f_\text{SM}  \gtrsim \Eboom \cdot \text{max} \left \{\frac{L_0}{\lambda_T}, 1 \right \}^3.
\end{equation}
We are thus sensitive to DM masses $m_\chi \gtrsim 10^{16} ~\GeV$.

\paragraph{Event Rate: DM Wind.}
DM with negligible energy loss in the WD medium will traverse the star in $\sim R_\text{WD}/v_\text{esc} \approx 0.1 ~\text{s}$ and have a number density within the WD enhanced relative to the average galactic density by a factor $(v_\text{esc}/v_\text{halo}) \sim 10$.
In the wind scenario, the DM-DM collision rate inside the WD parameterized by a cross-section $\sigma_{\chi \chi}$ is:
\begin{align}
  \Gamma_\text{collision}
  %&\sim n_\chi \sigma_{\chi \chi} v_\text{esc} \cdot n_\chi R_\text{WD}^3 \\
  \sim \l \frac{\rho_\chi}{m_\chi} \r^2 \sigma_{\chi \chi} \l \frac{v_\text{esc}}{v_\text{halo}}\r^3 v_\text{halo} R_\text{WD}^3.
  \label{eq:collisionDM}
\end{align}
Similarly the net DM decay rate inside the WD parameterized by a lifetime $\tau_\chi$ is:
\begin{align}
 \Gamma_\text{decay}
  %&\sim \frac{1}{\tau_\chi} \cdot n_\chi R_\text{WD}^3 \\
  \sim \frac{1}{\tau_\chi} \frac{\rho_{\chi}}{m_\chi} \l \frac{v_\text{esc}}{v_\text{halo}}\r R_\text{WD}^3.
  \label{eq:decayDM}
\end{align}


\paragraph{Event Rate: DM Capture.}
For the DM to be able to stop in a WD, it must have sufficiently strong interactions with stellar constituents:
\begin{equation}
\label{eq:capture}
\left( \frac{d E}{d x} \right)_\text{SP} \gtrsim \frac{m_\chi v_\text{halo}^2}{R_\text{WD}}.
\end{equation}
Note that the DM need only lose velocity of order its galactic virial velocity relative to the WD to get captured. 
If the DM initially has velocity of order $v \ll v_\text{esc}$ relative to the WD, we demand that it lose a fraction $(v/v_\text{esc})^2$ of its energy in the star. 
This can be made more precise for a specified interaction: the simplest case to consider is spin-independent, elastic scatters off ions characterized by a cross section $\sigma_{\chi A}$. 
For $m_\chi \gg m_\text{ion}$, the typical momentum transfer in an elastic scatter is $q \sim \mu_{A} v_\text{esc} \sim 100 ~\MeV$, where $\mu_{A}$ is the reduced mass of the DM-ion system. 
The typical energy transfer in a nuclear elastic scatter is simply $q^2/m_\text{ion} \sim m_\text{ion} v_\text{esc}^2 \approx \MeV$. 
If the DM has an initial relative velocity with the WD $v_\text{halo}$, the DM masses $m_\text{chi} \gtr 10^3 ~\GeV$ will require multiple scatters to lose the necessary fraction of its energy. 

The stopping power of the DM with this interaction is
\begin{equation}
\left( \frac{d E}{d x} \right)_\text{SP} \sim n_\text{ion} \sigma_{\chi A} m_\text{ion} v^2.
\end{equation}
Demanding \eqref{eq:capture} yields a condition on the DM elastic nuclear cross section sufficient for capture:
\begin{equation}
\label{eq:capturecross}
\sigma_{\chi A} \gtrsim  \l \frac{m_\chi}{m_\text{ion}} \r \l \frac{v}{v_\text{esc}} \r^2 \frac{1}{n_\text{ion}} \frac{1}{R_\text{WD}}. 
\end{equation}
Since the momentum transfer $q$ is of order the inverse nuclear size, it is reasonable to expect the DM coherently scatters off all nucleons.
Therefore, the average per-nucleon cross section (spin-independent)
\begin{equation}
\sigma_{\chi n} = A^2 \l \frac{\mu_{A}}{\mu_{n}}\r^2 F^2(q) \sigma_{\chi A} \sim A^4 F^2(q) \sigma_{\chi A},
\end{equation}
where $F^2(q) \approx 0.2$ is the Helm form factor, calculated based on the analytic expression in \cite{LUX thesis}. 
Therefore, the cross section needed for capture \eqref{eq:capturecross} can be compared to limits from direct detection experiments.
Currently, the most stringent bound on DM-nuclear elastic scatters comes from XENON 1T:
\begin{equation}
\label{eq:XENON}
\sigma_{\chi n} < 10^{-45} ~\text{cm}^2 \l \frac{m_\chi}{10^3 ~\GeV} \r.
\end{equation}


Implicitly, we are making the assumption that the scatters responsible for slowing the DM are not sufficient to blow up the WD, i.e. \eqref{eq:transitboom} is never satisfied. 


First we review the evolution of DM within the star once it has been captured. 
The DM eventually thermalizes at a velocity
\begin{equation}
v_\text{th} \sim \sqrt{\frac{T}{m_\chi}} \approx 10^{-12} \l \frac{10^{16} ~\GeV}{m_\chi}\r^{1/2},
\end{equation}
where $T \sim \keV$ is the WD temperature.
At this point the DM will accumulate at the thermal radius where its kinetic energy balances against the gravitational potential energy of the enclosed WD mass:
\begin{align}
  R_\text{th} &\sim \l \frac{T}{G m_\chi \rho_\text{WD}}\r^{1/2} \\
  &\approx 0.1 ~\text{cm} \l \frac{10^{16} ~\GeV}{m_\chi}\r^{1/2}
  \l \frac{10^{31} ~\cm^{-3}}{n_\text{ion}}\r^{1/2}, \nonumber
\end{align}
where we have assumed a constant WD density $\rho_\text{WD} \sim n_\text{ion} m_\text{ion}$ within $R_\text{th}$.
DM will collect at this radius until its total mass exceeds the WD mass within $R_\text{th}$,
\begin{align}
\label{eq:mcore}
    M_\text{crit} &\sim \rho_\text{WD} R^3_\text{th} \\
    &\approx 10^{29}~\GeV ~\l \frac{10^{16} ~\GeV}{m_\chi}\r^{3/2}
  \l \frac{10^{31} ~\cm^{-3}}{n_\text{ion}}\r^{1/2}. \nonumber
\end{align}
The DM cloud will then begin self-gravitational collapse.
The exact nature of this collapse is model-dependent, eventually being arrested by DM-DM interactions or the formation of a black hole.
For composite DM, it is reasonable to suspect that the collapse stabilizes into a core of radius $R_\text{sta}$ larger than the Schwarzschild radius due to some unknown physics. 

There are several timescales relevant to the DM capture scenario. 
Initially there is a timescale for DM to slow down to the thermal velocity $v_\text{th}$ and ``drift" down to the thermal radius.
This is roughly given by
\begin{align}
\label{eq:tdrift}
  t_\text{drift} \sim \frac{R_\text{WD}}{v_\text{th}}
  \approx 50 ~\text{yr} ~ \l \frac{m_\chi}{10^{16} ~\GeV} \r^{1/2}. 
\end{align}
Evidently for $m_\chi \gtrsim 10^{30}~\GeV$, a DM cloud at $R_\text{th}$ does not even form within the lifetime of the star. 
Once formed, the DM cloud will continue to collect DM until it reaches a critical core mass \eqref{eq:mcore} in a time
\begin{align}
\label{eq:tcol}
  t_\text{collect} &\sim \l \frac{M_\text{crit}}{m_\chi}\r \frac{1}{\Gamma_\text{transit}} \\
  & \approx 10 ~\text{s} \l \frac{10^{16} ~\GeV}{m_\chi} \r^{3/2} \l \frac{0.4 ~\GeV/\text{cm}^3}{\rho_\chi} \r,\nonumber
\end{align}
evaluated at $n_\text{ion} \sim 10^{31} ~\cm^{-3}$.
Note that $t_\text{collect}$ has a non-trivial dependence on WD density: this is manifest in the values for $v_\text{esc}$ and $R_\text{WD}$.
Subsequently, the actual timescale for the collapse is independent of DM mass:
\begin{align}
  t_\text{collapse} \sim \frac{R_\text{th}}{v_\text{th}}
  \approx 3 ~\text{s} ~ \l \frac{10^{31} ~\cm^{-3}}{n_\text{ion}}\r^{1/2}.
\end{align}
This expression is valid for the two cases that the DM core either initially remains thermalized while self-gravitating or collapses at free-fall velocity.  

We now turn towards the rate of explosive collisions and decays for captured DM.
Of course, if the slowing of DM itself is sufficient to ignite a SN, then the boom condition is instead given by the transit condition \eqref{eq:transitexplosion}.
Assuming this is not the case, we calculate the rate of DM events during the various stages of the capture described above.  

For decay heating, capture gives an enhancement due to the increased number of DM particles within the WD.
This can be very large if the DM core admits decays, however it is still significantly enhanced over the wind scenario even for inert cores (as in the case that the DM forms a black hole).
We have an enhancement of the net decay rate \eqref{eq:decayDM} by a factor
\begin{align}
\label{eq:enhancedecay}
  \frac{v_\text{esc}}{v_\text{th}}
  \approx 10^{10} \l \frac{m_\chi}{10^{16}~\GeV} \r^{1/2}
\end{align}
due to the increased time spent by the DM in the WD medium before joining the inert core.

In the case of DM-DM collision heating, it is possible that the collapse of the core will induce an ignition event due to the enhancement of DM number density during the collapse.
This would set the lifetime of WDs to $t_\text{collect}$.
During the collapse, the rate of collisions taking place at a radius $r$ within the enclosed volume is given by
\begin{align}
\Gamma_\text{collision}(r) \sim \l \frac{M_\text{core}}{m_\chi} \r^2 \frac{1}{r^3} \sigma_{\chi \chi} v(r),
\end{align}
where $v(r)$ is the velocity of DM - this could be either free-fall velocity or $v_\text{th}$ if the DM remains thermalized.
Integrating to the stable radius $R_\text{sta}$, we find the total number of collisions during the collapse is
\begin{align}
  N_\text{col} \sim \l \frac{M_\text{core}}{m_\chi} \r^2 \frac{\sigma_{\chi \chi}}{R_\text{sta}^2}.
 \end{align}
Assuming the collapse proceeds until the DM core becomes a black hole, the number of collisions is
\begin{align}
  N_\text{col} \sim \frac{\sigma_{\chi \chi}}{G^2 m_\chi^2}.
 \end{align}
If the collapse itself is not explosive, there is still an enhanced collision rate relative to the wind scenario due to DM colliding while in-falling to the core.
Again we look at the conservative situation of an inert core - the rate is obviously much greater if the core is stabilized in a fluid state which admits DM-DM collisions.
The rate of in-falling collisions is enhanced over the wind collision rate \eqref{eq:collisionDM} by a factor
\begin{align}
\label{eq:enhancecollision}
   \frac{R_\text{WD}}{R_\text{sta}} \times \frac{v_\text{esc}}{v_\text{th}},
\end{align}
which again depends on the physics of $R_\text{sta}$.

\paragraph{Event Rate: DM Capture and Multiple Collisions}

Up until now, we only considered the case that a single DM-DM collision releases sufficient energy \eqref{eq:coldecay} in order to trigger runaway fusion.
However, the possibility of a DM core collapse in the star provides up an interesting alternative:

\end{document}