%\documentclass[10pt, twocolumn]{article}
%\documentclass[11pt]{article}
%\documentclass[twocolumn,showpacs,preprintnumbers,amsmath,amssymb,prl, superscriptaddress]{revtex4}
%\documentclass[twocolumn, preprintnumbers,amsmath,amssymb,prd, superscriptaddress]{revtex4}
\documentclass[preprintnumbers,amsmath,amssymb,prd,superscriptaddress]{revtex4}
%\documentclass[10pt, preprint,showpacs,preprintnumbers,amsmath,amssymb, superscriptaddress]{revtex4}
%\documentclass[11pt, prd,preprintnumbers,amsmath,amssymb, superscriptaddress]{revtex4}
%\documentclass[11pt, prd,preprintnumbers, amsmath,amssymb, superscriptaddress, nofootinbib, hyperref]{revtex4}

\usepackage{latexsym}
\usepackage{amssymb}
\usepackage{epsfig,amsmath,graphics}
\usepackage{epstopdf}
\usepackage{verbatim}
\usepackage{wasysym}
\usepackage{hyperref}
\usepackage{feynmp-auto} % feynman diagrams
%\usepackage{subfig}
\usepackage[utf8]{inputenc}
\usepackage{xpatch}
\usepackage{xcolor}
\usepackage{mathtools}
\hypersetup{
    colorlinks,
    linkcolor={red!80!black},
    citecolor={green!60!black},
    urlcolor={blue!60!black}
}
\usepackage{appendix}

\newcommand{\Ez}{\mathcal{E}_0}
\newcommand{\Eboom}{\mathcal{E}_\text{boom}}
\newcommand{\OO}{\mathcal{O}}
\newcommand{\LL}{\mathcal{L}}
\newcommand{\HH}{\mathcal{H}}
\newcommand{\TeV}{\text{TeV}}
\newcommand{\GeV}{\text{GeV}}
\newcommand{\MeV}{\text{MeV}}
\newcommand{\keV}{\text{keV}}
\newcommand{\rad}{\text{rad}}
\newcommand{\cm}{\text{cm}}
\newcommand{\angstrom}{\buildrel _{\circ} \over {\mathrm{A}}}
\newcommand{\pslash}{p\hspace{-0.070in}/\,}
\newcommand{\Mpl}{M_{\text{pl}}}
\newcommand{\ket}[1]{\ensuremath{\left|#1\right>}}
\newcommand{\bra}[1]{\ensuremath{\left<#1\right|}}
\newcommand{\braket}[2]{\ensuremath{\left<#1|#2\right>}}
%Large Parentheses
\def\r{\right)}
\def\l{\left(}

\begin{document}

%\preprint{APS/123-QED}

Thus for a heating event characterized by its $L_0$, $\Ez$, and $T_0 \gtrsim T_f$, there is a \emph{boom condition}:
\begin{align}
    \label{eq:energy_boom_condition}
    \Ez \gtrsim
    \Eboom \cdot \text{max}\left\{1, \frac{L_0}{\lambda_T}\right\}^3.
\end{align}

\subsection{Transits}

\paragraph{Boom Condition.}
The energy deposited during a continuous heating event such as a DM transit is best described in terms of a linear energy transfer $(dE/dx)_\text{LET}$, the kinetic energy of SM particles produced per distance traveled by the DM.
If these products have a heating length $L_0$ then the relevant energy deposit must at minimum be taken as the energy transferred over the transit distance $L_0$.
Of course, we can always choose to consider energy deposits over a longer segment of the DM trajectory.
Importantly, as per the general condition \eqref{eq:energy_boom_condition} such a deposition is \emph{less} explosive unless $L_0$ is smaller than the trigger size $\lambda_T$.
Thus, we consider the energy deposited in a transit over the larger of these two length scales.
Assuming the energy of the DM is roughly constant over this heating event, the boom condition for transit heating is:
\begin{align}
\label{eq:transitexplosion}
  \left( \frac{d E}{d x} \right)_\text{LET} \gtrsim
  \frac{\Eboom}{\lambda_T} \cdot \text{Max}
  \left\{\frac{L_0}{\lambda_T}, 1 \right\}^2.
\end{align}

The above argument sums the individual energy deposits along the DM trajectory as though they are all deposited simultaneously.
This is possible if the DM moves sufficiently quickly so that this energy does not diffuse out of the region of interest before the DM has traversed the region.
We therefore require that the diffusion time $\tau_\text{diff} \approx 10^{-12} ~\text{s}$ across a heated region at temperature $T_f$ be larger than the DM crossing-time:
\begin{align}
  \tau_\text{diff} \sim \frac{L^2}{\alpha(T_f)} \gg
  \frac{L}{v_\text{esc}},
\label{eq:SlowDiffusion}
\end{align}
where $\alpha(T)$ is the temperature-dependent diffusivity, and the DM transits at the stellar escape velocity $v_\text{esc} \approx 10^{-2}$.
This condition is more stringent for smaller regions, so we focus on the smallest region of interest, $L = \lambda_T$.
\eqref{eq:SlowDiffusion} is then equivalent to demanding that the escape speed is greater than the conductive speed of the fusion wave front, $v_\text{cond} \sim \alpha(T_f) / \lambda_T$.
Numerical calculations of $v_\text{cond}$ are tabulated in \cite{Woosley}, and indeed condition \eqref{eq:SlowDiffusion} is satisfied for all WD densities.

\paragraph{Event Rate: Wind Scenario.}
The rate of transit events is given by the flux of DM passing through a WD
\begin{align}
  \Gamma_\text{transit} \sim
  \frac{\rho_{\chi}}{m_\chi} R_\text{WD}^2
  \l\frac{v_\text{esc}}{v_\text{halo}}\r^2 v_\text{halo},
\label{eq:TransitFluxCondition}
\end{align}
where $m_\chi$ is the DM mass, $\rho_\chi$ is the local DM density near the WD, and $R_\text{WD} \approx 4000 ~\text{km}$ is the WD radius.
Here $v_\text{halo} \sim 10^{-3}$ is galactic virial velocity, and the transit rate contains an $\OO(100)$ enhancement due to gravitational focusing.

\paragraph{WD Shielding.}
Runaway fusion only occurs in the degenerate WD interior where thermal expansion is suppressed as a cooling mechanism.
The outer layers of the WD, however, are composed of a non-degenerate gas and it is therefore essential that a DM candidate penetrate this layer in order to ignite a SN.
We parameterize this by a DM stopping power $(dE/dx)_\text{SP}$, the kinetic energy lost by the DM per distance traveled in the non-degenerate layer, and demand that
\begin{align}
\label{eq:CrustCondition}
  \left( \frac{d E}{d x} \right)_\text{SP} \ll
  \frac{m_\chi v^2_\text{esc}}{R_\text{envelope}},
\end{align}
where $R_\text{envelope} \approx 50 ~\text{km}$ is the width of a WD envelope \cite{KippenhahnWeigert}.
Note that the DM stopping power in the non-degenerate layer $(dE/dx)_\text{SP}$ and the linear energy transfer in the degenerate interior $(dE/dx)_\text{LET}$ are possibly controlled by different physics and may have very different numerical values.
In addition, a transit heating event satisfying condition \eqref{eq:CrustCondition} will have negligible energy loss over the parametrically smaller trigger size or heating length $L_0$, validating the boom condition \eqref{eq:transitexplosion}.

\subsection{Collisions and Decays}

\paragraph{Boom Condition.}
For a point-like DM-DM collision or DM decay event releasing particles of heating length $L_0$, ignition will occur if the total energy in SM products satisfies condition~\eqref{eq:energy_boom_condition}.
Such an event will likely result in both SM and dark sector products, so we parameterize the resulting energy in SM particles as a fraction $f_\text{SM}$ of the DM mass.
For non-relativistic DM, the DM mass is the dominant source of energy and therefore $f_\text{SM} \lesssim 1$ regardless of the interaction details, although we may well suspect that $f_\text{SM} \ll 1$ for realistic models.
With this parameterization, a single DM-DM collision or DM decay has a boom condition:
\begin{equation}
\label{eq:coldecay}
  m_\chi f_\text{SM}  \gtrsim \Eboom \cdot \text{max} \left \{\frac{L_0}{\lambda_T}, 1 \right \}^3.
\end{equation}
We are thus sensitive to DM masses $m_\chi \gtrsim 10^{16} ~\GeV$.

However, there is an interesting possibility if DM is captured in the WD that allows collisions of lower mass DM to ignite the star. 
Multiple DM-DM collisions in a sufficiently small region can occur rapidly enough to be counted as a single heating event.
This is similar in nature to a transit heating event, where multiple scatters across a transit length $\lambda_T$ can release an energy $\Eboom$ and satisfy \eqref{eq:transitexplosion} even if an individual scatter is not explosive by itself.   
If a single DM-DM collision is unable to ignite the star, the sum total of the energy released in many collisions can still result in a SN if the following boom condition is satisfied:
\begin{equation}
\label{eq:multcolboom}
 m_\chi f_\text{SM}  \gtrsim \frac{\Eboom}{N_\text{mult}} \cdot \text{max} \left \{\frac{L_0}{\lambda_T}, 1 \right \}^3, ~~~~ N_\text{mult} \gtrsim 1. 
\end{equation}
We define $N_\text{mult}$ as the number of collisions happening within a region of size $\lambda_T^3$ or smaller during a diffusion time $\tau_\text{diff}$.
This necessarily depends on the DM-DM collision cross section, the DM-SM scattering cross section, and the evolution of the captured DM in the star. 
As we will see, the condition for multiple collisions to be explosive \eqref{eq:multcolboom} contains a number of additional assumptions about the DM and its interactions that are not present for a single, explosive collision. 
These are discussed in detail below. 

\paragraph{Event Rate: DM Wind.}
DM with negligible energy loss in the WD medium will traverse the star in $\sim R_\text{WD}/v_\text{esc} \approx 0.1 ~\text{s}$ and have a number density within the WD enhanced relative to the average galactic density by a factor $(v_\text{esc}/v_\text{halo}) \sim 10$.
In the wind scenario, the DM-DM collision rate inside the WD parameterized by a cross-section $\sigma_{\chi \chi}$ is:
\begin{align}
  \Gamma_\text{collision}
  %&\sim n_\chi \sigma_{\chi \chi} v_\text{esc} \cdot n_\chi R_\text{WD}^3 \\
  \sim \l \frac{\rho_\chi}{m_\chi} \r^2 \sigma_{\chi \chi} \l \frac{v_\text{esc}}{v_\text{halo}}\r^3 v_\text{halo} R_\text{WD}^3.
  \label{eq:collisionDM}
\end{align}
Similarly the net DM decay rate inside the WD parameterized by a lifetime $\tau_\chi$ is:
\begin{align}
 \Gamma_\text{decay}
  %&\sim \frac{1}{\tau_\chi} \cdot n_\chi R_\text{WD}^3 \\
  \sim \frac{1}{\tau_\chi} \frac{\rho_{\chi}}{m_\chi} \l \frac{v_\text{esc}}{v_\text{halo}}\r R_\text{WD}^3.
  \label{eq:decayDM}
\end{align}

\paragraph{Event Rate: DM Capture.}
For the DM to be captured in a WD, it must have sufficiently strong interactions with stellar constituents:
\begin{equation}
\label{eq:capture}
\left( \frac{d E}{d x} \right)_\text{SP} \gtrsim \frac{m_\chi v^2}{R_\text{WD}},
\end{equation}
where $v$ is the DM galactic virial velocity relative to the WD.  
For $v \ll v_\text{esc}$, we demand that the DM lose a fraction $(v/v_\text{esc})^2$ of its energy to eventually become stopped in the star. 
Of course, the most probable relative velocity is of order $v_\text{halo}$.
If every DM that passes through the WD is captured, then the capture rate $\Gamma_\text{cap}$ is simply \eqref{eq:TransitFluxCondition}. 
However even if the interaction is unable to capture DM at a velocity $v_\text{halo}$, it may still capture those DM with a lower relative velocity. 
The distribution of DM particles at relative velocity $v$ is given by:

The physics of DM capture can be made more precise for a specific interaction.
The simplest scenario is a spin-independent, elastic scattering off ions characterized by cross section $\sigma_{\chi A}$. 
For $m_\chi \gg m_\text{ion}$, the typical momentum transfer in an elastic scatter is $q \sim \mu_{A} v_\text{esc} \approx 100 ~\MeV$, where $\mu_{A}$ is the reduced mass of the DM-ion system. 
The typical energy transfer in a nuclear elastic scatter is simply $q^2/m_\text{ion} \sim m_\text{ion} v_\text{esc}^2 \approx \MeV$. 
If the DM has an initial relative velocity with the WD of order $v_\text{halo}$, then DM masses $m_\chi \gtrsim 10^3 ~\GeV$ will require multiple elastic scatters to lose the necessary fraction of its energy. 
The stopping power of the DM with this interaction is
\begin{equation}
\left( \frac{d E}{d x} \right)_\text{SP} \sim n_\text{ion} \sigma_{\chi A} m_\text{ion} v^2.
\end{equation}
Demanding \eqref{eq:capture} yields a condition on the DM elastic nuclear cross section sufficient for capture:
\begin{equation}
\label{eq:capturecross}
\sigma_{\chi A} \gtrsim  \l \frac{m_\chi}{m_\text{ion}} \r \l \frac{v}{v_\text{esc}} \r^2 \frac{1}{n_\text{ion}} \frac{1}{R_\text{WD}}. 
\end{equation}
Evidently, the assumption that the scatters responsible for slowing the DM are not sufficient to blow up the WD \eqref{eq:transitexplosion} is a valid one if the cross section satisfies
\begin{equation}
\sigma_{\chi A} < \l \frac{\Eboom}{\lambda_T} \r \l \frac{1}{m_\text{ion} v_\text{esc}^2} \r \l \frac{1}{n_\text{ion}} \r \sim 10^{-8} ~\text{cm}^2,
\end{equation}
evaluated at a WD density $n_\text{ion} \sim 10^{32} \text{cm}^{-3}$. 
Since the momentum transfer $q$ is of order the inverse nuclear size, it is reasonable to expect the DM coherently scatters off all nucleons in the nucleus.
Therefore, the average per-nucleon cross section (spin-independent) is
\begin{equation}
\sigma_{\chi n} = A^2 \l \frac{\mu_{A}}{\mu_{n}}\r^2 F^2(q) \sigma_{\chi A} \sim A^4 F^2(q) \sigma_{\chi A},
\end{equation}
where $F^2(q) \approx 0.2$ is the Helm form factor, calculated based on the analytic expression in \cite{LUX thesis}. 
The cross section needed for capture \eqref{eq:capturecross} can be compared to limits from direct detection experiments.
Currently, the most stringent bound on DM nuclear elastic scatters comes from XENON 1T:
\begin{equation}
\label{eq:xenon}
\sigma_{\chi n} < 10^{-45} ~\text{cm}^2 \l \frac{m_\chi}{10^3 ~\GeV} \r.
\end{equation}

We now review the evolution of DM within the star once it has been captured. 
The DM eventually thermalizes at a velocity
\begin{equation}
v_\text{th} \sim \sqrt{\frac{T}{m_\chi}} \approx 10^{-12} \l \frac{10^{16} ~\GeV}{m_\chi}\r^{1/2},
\end{equation}
where $T \sim \keV$ is the WD temperature.
At this point, the DM accumulates at the radius where its kinetic energy balances against the gravitational potential energy of the enclosed WD mass:
\begin{align}
  R_\text{th} &\sim \l \frac{T}{G m_\chi \rho_\text{WD}}\r^{1/2} \\
  &\approx 0.1 ~\text{cm} \l \frac{10^{16} ~\GeV}{m_\chi}\r^{1/2}
  \l \frac{10^{31} ~\cm^{-3}}{n_\text{ion}}\r^{1/2}, \nonumber
\end{align}
where we have assumed a constant WD density $\rho_\text{WD} \sim n_\text{ion} m_\text{ion}$ within $R_\text{th}$.
Of course, the timescale for captured DM to thermalize and settle at $R_\text{th}$ depends on the nature of the DM-SM interaction.
If the DM is capable of losing energy rapidly in the star, we expect this time to be of order
\begin{align}
\label{eq:tdrift}
  t_\text{settle} \sim \frac{R_\text{WD}}{v_\text{th}}
  \approx 50 ~\text{yr} ~ \l \frac{m_\chi}{10^{16} ~\GeV} \r^{1/2}. 
\end{align}
Note that the settle time can be explicitly calculated in the case that the DM is captured via elastic nuclear scatters. 
For the rest of this section, we demand that the settle time is less than the age of the WD $t_\text{settle} < \tau_\text{WD}$. 
The in-falling DM constitutes a number density of DM throughout the WD volume:
\begin{equation}
n(r) \sim \frac{\Gamma_\text{cap}}{r^2 v(r)}.
\end{equation}
This in-falling DM annihilates at a total rate, integrating over the WD:
\begin{equation}
\label{eq:infallrate}
\Gamma_\text{infall} \sim \int_{R_\text{WD}}^{R_\text{th}} \frac{n^2 \sigma_{\chi \chi} v}{r^3} dr \sim \frac{\Gamma_\text{cap}^2 \sigma_{\chi \chi}}{R_\text{th} v_\text{th}}.
\end{equation}
We can ignore the depletion of in-falling DM as long as \eqref{eq:infallrate} does not exceed the capture rate:
\begin{equation}
\label{eq:steadycollect}
\frac{\Gamma_\text{cap} \sigma_{\chi \chi}}{R_\text{th} v_\text{th}} \lesssim 1. 
\end{equation}
%Interestingly, this condition is independent of $m_\chi$ and translates to an upper bound on the cross section $\sigma_{\chi \chi} \lesssim 10^{-13} ~\text{cm}^2$, at WD densities $n_\text{ion} \sim 10^{31} ~\text{cm}^{-3}$. 

After the initial off-set time $t_\text{settle}$, DM will begin steadily collecting at the thermal radius.
Assuming \eqref{eq:steadycollect} is satisfied, the collection rate is roughly the same as the capture rate $\Gamma_\text{cap}$. 
However, the DM is also annihilating with itself.
Eventually, the collection and annihilation rates become comparable and there is an equilibrium number of DM particles accumulated at the thermal radius:
\begin{equation}
N_\text{eq} \sim \l \frac{\Gamma_\text{cap} R_\text{th}^3}{\sigma_{\chi \chi} v_\text{th}} \r^{1/2}.
\end{equation}
Of course, there is no guarantee that this equilibrium state is achieved within the age of the WD. 
In that case, we can ignore annihilations and the total number of DM particles accumulated is simply
\begin{equation}
N_\text{life} \sim \Gamma_\text{cap} \tau_\text{WD}.
\end{equation}
However, if the total collected mass of DM at some point exceeds the WD mass within $R_\text{th}$, then there is the possibility of self-gravitational collapse.
The critical number of DM particles needed for collapse is given by
\begin{align}
\label{eq:Ncore}
    N_\text{crit} &\sim \frac{\rho_\text{WD} R^3_\text{th} }{m_\chi}. 
\end{align}
This can only be achieved if the time to collect a critical mass of DM is shorter than the time for annihilations to deplete this mass sufficiently \emph{and} shorter than the WD lifetime. 
In other words:
\begin{equation}
\label{eq:collapsecondition}
N_\text{crit} < N_\text{eq}, ~~~~ N_\text{crit} < N_\text{life}.
\end{equation}
Evidently, DM masses below $\sim 10^{6} ~\GeV$ do not have enough time within the age of a WD to collect to the critical number \eqref{eq:Ncore} and trigger a collapse. 
We must also ensure that the collapsing DM is not immediately depleted due to annihilations.
The timescale for self-gravitational collapse is independent of DM mass and is approximately given by:
\begin{align}
  t_\text{ff} \sim \frac{R_\text{th}}{v_\text{th}}
  \approx 1 ~\text{s} ~ \l \frac{10^{32} ~\cm^{-3}}{n_\text{ion}}\r^{1/2}.
\end{align}
Demanding that the free-fall time is less than the time for annihilations to deplete $N_\text{crit}$ by an $\OO(1)$ fraction results in an additional condition:
\begin{align}
\label{eq:xicondition}
\xi \equiv \frac{N_\text{crit} \sigma_{\chi \chi}}{R_\text{th}^2} < 1.
\end{align}
If both \eqref{eq:collapsecondition} and \eqref{eq:xicondition} are satisfied, the collected DM will undergo a gravitational collapse.
The exact nature of this collapse is model-dependent, eventually being arrested by DM-DM interactions or the formation of a black hole.
%For composite DM, it is reasonable to suspect that the collapse stabilizes into a core of radius $R_\text{sta}$ larger than the Schwarzschild radius $R_\text{sch} \sim G m_\chi N_\text{crit}$. 
%However, this depends on unknown physics. 
Of course, even if DM annihilations can be ignored during the initial stages of the collapse, the annihilation rate increases with the number density of DM. 
The number of DM particles in the collapsing core can be written as a function of collapse radius:
\begin{equation}
N(r) \sim \frac{N_\text{crit}}{1 + \xi \l \frac{R_\text{th}}{r}\r^2}. 
\end{equation}
Thus, the effect of annihilations on the collapsing DM mass becomes important when it reaches a size
\begin{equation}
R_{\chi \chi} \sim R_\text{th} \sqrt{\xi},
\end{equation}
below which the mass of collapsing DM depletes as $N \propto r^2$. 

There are two potential evolutions of the captured DM in which to examine in the boom condition \eqref{eq:multcolboom}: either the DM collapses or it does not. 
In the later case, we have 
\begin{equation}
\label{eq:nocollapseann}
\text{min}\{N_\text{eq}, N_\text{life}\} < N_\text{crit}.
\end{equation}
Either the DM has reached its equilibrated number at the thermal radius or is still continuing to accumulate, not yet having the critical mass necessary for collapse within its lifetime. 
For multiple collisions, we must count the number of collisions taking place in a trigger-sized subregion of the DM sphere within the fixed time interval set by $\tau_\text{diff}$.
This is of the form:
\begin{equation}
\label{eq:nocollapse}
N_\text{multi} \sim \l \frac{\text{min}\{N_\text{eq}, N_\text{life}\}}{R_\text{th}^3} \r^2 \sigma_{\chi \chi} v_\text{th} \lambda_T^3 \tau_\text{diff}. 
\end{equation}
Demanding that \eqref{eq:infallrate}, \eqref{eq:nocollapseann}, and \eqref{eq:nocollapse} all be satisfied, we find that
 
We now turn towards multiple DM-DM collisions during a collapse. 
If the DM-SM interaction is sufficiently strong, it is possible for the DM to initially remain thermalized while collapsing.
The gravitational potential energy that is lost during this portion of the collapse goes into heating the surrounding WD medium rather than accelerating the DM. 
This was considered by \cite{Bramante} for the capture and subsequent collapse of asymmetric DM in a WD.  
Rather, we focus on a complementary scenario: the DM collapses at free-fall velocity and annihilates sufficiently rapidly to heat the star. 
Here the number of collisions $N_\text{multi}$ depends on the radius $r$ at which we choose to analyze the collapse:
\begin{equation}
N_\text{multi} \sim \l \frac{N}{r^3}\r^2  \sigma_{\chi \chi} v_\text{ff} ~\text{min}\{\lambda_T, r\}^3 ~\text{min}\{\tau_\text{diff}, r/v_\text{ff}\},
\end{equation}
where $v_\text{ff} \sim \sqrt{\frac{G N m_\chi}{r}}$ is the free-fall velocity of the collapsing DM mass. 
However, there is an optimal radius of the collapse at which $N_\text{multi}$ is maximized.
Of course, $r$ must always be greater than the Schwarzschild radius
\begin{equation}
R_\text{sch} \sim G m_\chi N_\text{crit}.
\end{equation}
If $R_\text{sch} < \lambda_T$, then the DM collapse can possibly 

This is straightforward to see: if the DM is collapsing above the trigger-size $r>\lambda_T$, then $N_\text{multi}$ scales inversely with $r$.
Thus, we should restrict our attention to $r \lesssim \lambda_T$, provided that collapse to such a radius is physically allowed.
$r$ must always be greater than the Schwarzschild radius
\begin{equation}
R_\text{sch} \sim G m_\chi N_\text{crit}.
\end{equation}

 \paragraph{GUTzilla:}



\end{document}