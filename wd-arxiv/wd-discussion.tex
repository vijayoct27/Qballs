The detection of ultra-heavy DM is an open problem which will likely require a confluence of astrophysical probes.
Here we present a guide to constraining these candidates through DM-SM scatters, DM-DM annihilations, and DM decays inside a WD that release sufficient SM energy to trigger runaway fusion.
In particular, we calculate the energy loss of high-energy particles due to SM interactions within the WD medium and determine the conditions for which a general energy deposition will heat a WD and ignite SN.
Ultra-heavy DM that produces greater than $10^{16}~\GeV$ of SM particles in a WD is highly constrained by the existence of heavy WDs and the measured SN rate.
The formalism provided will enable WDs to be applied as detectors for any DM model capable of heating the star through such interactions. 
We have done so for baryonic Q-balls, significantly constraining the allowed parameter space in a complementary way to terrestrial searches. 

We have explored briefly the application of this WD instability to self-gravitational collapse of DM cores, which has very interesting possibilities. 
The decay or annihilation of DM which is captured by a WD and forms a self-gravitating core is highly constrained for DM with mass greater than $10^{16}~\GeV$.
In addition, such collapsing cores can provide enough heating via multiple annihilations to ignite the star for much smaller DM masses than those considered here, e.g.~$10^7~\GeV$, and can induce SN through other means such as the formation and evaporation of mini black holes. 
These will be addressed in future work~\cite{us}.  

Finally, in addition to the constraints mentioned above, the general phenomenology of these DM-induced runaways will be the ignition of sub-Chandrasekhar mass WDs, possibly with no companion star present.
Some of the mechanisms considered above are also likely to initiate fusion far from the center of the star. 
This is in contrast with conventional single-degenerate and double-degenerate mechanisms, which require a companion star and ignite fusion near the center of a super-Chandrasekhar mass WD~\cite{Maoz:2012}.  
This raises the tantalizing possibility that DM encounters with WDs provide an alternative explosion mechanism for type Ia SN or similar transient events, and that these events may be distinguishable from conventional explosions. 
Understanding and searching for possible distinguishing features of DM-induced events is an important follow-up work. 