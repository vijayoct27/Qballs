%\documentclass[10pt, twocolumn]{article}
\documentclass[twocolumn,showpacs,preprintnumbers,amsmath,amssymb,prl]{revtex4}
%\documentclass[11 pt, preprint,preprintnumbers,amsmath,amssymb, prd]{revtex4}

% Preamble adapted from Surjeet Rajendran

\usepackage{latexsym}
\usepackage{amssymb}
\usepackage{epsfig,amsmath,graphics}
\usepackage{epstopdf}
\usepackage{verbatim}
\usepackage{wasysym}
\usepackage{hyperref}
\usepackage{feynmp-auto} % feynman diagrams
%\usepackage{subfig}
\usepackage[utf8]{inputenc}
\usepackage{xpatch}
\usepackage{xcolor}
\hypersetup{
    colorlinks,
    linkcolor={red!80!black},
    citecolor={green!60!black},
    urlcolor={blue!60!black}
}

\newcommand{\OO}{\mathcal{O}}
\newcommand{\LL}{\mathcal{L}}
\newcommand{\HH}{\mathcal{H}}

\newcommand{\GeV}{\text{GeV}}
\newcommand{\rad}{\text{rad}}
\newcommand{\angstrom}{\buildrel _{\circ} \over {\mathrm{A}}}
\newcommand{\pslash}{p\hspace{-0.070in}/\,}
\newcommand{\Mpl}{M_{\text{pl}}}
\newcommand{\ket}[1]{\ensuremath{\left|#1\right>}}
\newcommand{\bra}[1]{\ensuremath{\left<#1\right|}}
\newcommand{\braket}[2]{\ensuremath{\left<#1|#2\right>}}
%Large Parentheses
\def\r{\right)}
\def\l{\left(} 
\begin{document}
\title{White Dwarves as Dark Matter Detectors}
\maketitle
\section{Introduction}

Localized heating of a white dwarf (WD) has the potential to ignite the star \cite{Woosley}. Namely, if a region of size $\lambda_T$ or greater is raised to a critical temperature $T_f$, this would initiate runaway thermonuclear fusion and cause the WD to explode in a supernovae. We consider WD densities in the range $\rho \sim 10^{6} - 10^{9} ~\frac{\text{gm}}{\text{cm}^3}$. For carbon-oxygen WD, this translates to $n \sim 10^{29} - 10^{32} ~\text{cm}^{-3}$ and $n_e \sim 10^{30} - 10^{33} ~\text{cm}^{-3}$ for number densities of nuclei and electrons, respectively. \textcolor{red}{Someone check these density numbers} Over this range of densities, the trigger size approximately varies between $\lambda_T \sim 10^{-5} ~\text{cm} -10^{-2} ~\text{cm}$ \cite{Graham:2015apa}. \textcolor{red}{Get values from SR.}

\section{Overview}

Consider an ultra-heavy dark matter (DM) state transit through the WD. Assume that the DM interacts with WD constituents (ions or electrons) in a general manner as shown in Figure \ref{fig:feynmandiag}, releasing $n_i$ particles of species $i$ each with kinetic energy $\epsilon$. The cross section for this interaction is denoted as $\sigma_{i,\epsilon}$. Of course, this released energy must be transferred to the stellar medium in order for the WD to be heated. For a given particle type, each value of $\epsilon$ is characterized by a distance $R_\epsilon$ from the point of release over which it is deposited. This ``range", and therefore the nominal size of the resulting hot region, is set by the various ways in which the emitted particle interacts with the stellar constituents and is able to dumps its energy. In particular, we define $R_\epsilon$ as the distance over which a particle $i$ and any secondaries transfer $\OO(1)$ of the initial energy $\epsilon$ to electrons or ions in the WD. To demonstrate the significance of this parameter, suppose that the DM simply scattered off nuclei elastically with no particles released. In this case, $R_\epsilon$ effectively vanishes as $\epsilon$ is transferred directly to the kinetic energy of nuclei. However, in the other extreme limit, suppose the interaction released energy into neutrinos. In this case, $R_\epsilon$ is of astronomical length scales and there would be no chance of thermalizing any local region in the star. 

\begin{figure}
\label{fig:feynmandiag}
\includegraphics[scale=.05]{feynmandiag}
\caption{General interaction between ultra-heavy DM and WD constituents, producing $n_i$ additional particles of species $i$.}
\end{figure}

Consider the two possibilities relevant for ignition. If $R_\epsilon> \lambda_T$, the DM must deposit a minimum energy $E_{\text{boom}} \sim R_\epsilon^3 n T_f$ in order to heat up the entire region of size $R_\epsilon$ to the critical temperature $T_f$, where $n$ is the number density of nuclei in the WD. On the other hand, if $R_\epsilon < \lambda_T$ then the minimum energy required is independent of $R$ and given by $E_{\text{boom}} \sim \lambda_T^3 n T_f$. Setting $T_f \sim \text{MeV}$, $\lambda_T \sim 10^{-5} ~\text{cm}$, and $n \sim 10^{32} ~\text{cm}^{-3}$, we find that an energy $E_{\text{boom}} \sim 10^{14} ~\text{GeV}$ transferred to the WD within a localized region smaller than $\lambda_T$ will eventually trigger runaway fusion. This ``explosion" energy must be less than the total energy released during the DM transit over a distance $\text{min}\{\lambda_T, R_\epsilon\}$. Doing so, we find a lower bound on the interaction cross section sufficient to trigger runaway fusion: 
\begin{equation}
\label{eq:explosion}
n_i \sigma_{\epsilon,i} \gtrsim \left\{
        \begin{array}{ll}
            \displaystyle \lambda_T^2 \l \frac{T_f}{\epsilon} \r & \quad R_\epsilon < \lambda_T \\
             \lambda_T^2 \l \frac{R_\epsilon}{\lambda_T}\r^2 \l \frac{T_f}{\epsilon} \r & \quad R_\epsilon > \lambda_T
        \end{array}
    \right..
\end{equation}

Note that in deriving this explosiveness bound, we have assumed the DM transit time is less than the corresponding diffusion time. After a time $\Delta t$ the DM has traversed $v_{\text{esc}} \Delta t$, where the velocity is set by the escape velocity of a WD. Therefore, this amounts to the following condition:
\begin{equation}
\begin{array}{ll}
             \tau_d(\lambda_T, T_f) \gtrsim \frac{\lambda_T}{v_{\text{esc}}}, & \quad \quad R_\epsilon < \lambda_T \\ \\
            \tau_d(R_\epsilon, T_f)  \gtrsim \frac{R_\epsilon}{v_{\text{esc}}},  & \quad \quad R_\epsilon > \lambda_T
        \end{array}
\end{equation}
where $\tau_d$ is the characteristic time for a region of given size and temperature to diffuse $\OO(1)$ of its heat. Within the validity of the heat equation this is simply given by $\tau_d(\lambda_T, T_f) \sim \frac{\lambda_T^2}{\alpha}$, where $\alpha$ is the (temperature-dependent) diffusivity. \textcolor{blue}{Show condition is true for all densities.} We also assume that the time to transfer energy $\epsilon$ out to its characteristic range $R_\epsilon$ is less than the diffusion time scale $\tau_d(\lambda_T, T_f)$. 

\section{Particle Detection in the WD}
In this section, we enumerate the different ways in which standard model particles can lose energy to the WD for the purposes of determining $R_\epsilon$ for each possible channel. The interior of a (carbon-oxygen) WD is a complex environment. Famously, the star is supported against collapse by the degeneracy pressure of an electron as at a characteristic Fermi energy $E_F \sim n_e^{1/3} \sim \OO(\text{MeV})$. In addition, the nuclei are at an ambient temperature $T \sim \text{keV}$ a a strongly-coupled plasma with $\frac{Z e^2}{n^{1/3} T} \gg 1$. In what follows, we calculate the WD medium sensitivity to deposited energy. For the purpose of depositing sufficient energy to trigger supernovae, we focus on high-energy particles $\epsilon \gg \text{MeV}$ which interact via the strong and electromagnetic forces (i.e. electrons, muons, photons, pions, and neutral hadrons). Therefore, the WD may be thought of as a new type of particle detector with electromagnetic and hadronic ``calorimeter" components.  

\subsection{Electromagnetic Interactions}

For charged particles, Coulomb scattering is a useful mechanism for energy transfer. Generically, an incident (spin-0) particle of mass $m_i$, charge $e$, and velocity $\beta$ scattering off a target $M_t$ of charge $Ze$ is described by the ``Rutherford" differential cross section \textcolor{blue}{Originally derived by Bhabha?}
\begin{equation}
\label{eq:rutherford}
\frac{d \sigma (E', \beta)}{dE'} = \frac{2 \pi  \alpha^2 Z^2}{M_t \beta^2} \frac{1}{E'^2} \l1- \frac{\beta^2 E'}{E_{\text{kin}}}\r, 
 \end{equation}
where we have assumed a sufficiently fast incident particle so that interactions are governed by single collisions with energy transfer $E'$ \cite{Agashe:2014kda}. $E_{\text{kin}}$ denotes the maximum energy transfer possible satisfying kinematic constraints \textcolor{blue}{target at rest and zero relative angle between incoming incident and outgoing target momenta}:
\begin{equation}
E_{\text{kin}} = \frac{2 M_t \beta^2 \gamma^2}{1+ 2\gamma(M_t/m_i) +(M_t/m_i)^2}. 
\end{equation}

For sufficiently heavy incident particles, the differential cross section depends only on the velocity of the incident particle. Note that higher-spin particles receive additional corrections to the cross section, but for small energy transfers these corrections are negligible. It is straightforward to understand the parametric dependences of \eqref{eq:rutherford}: there is increased likelihood to scatter for slowly moving incident particles undergoing ``soft-scatters" against lighter targets. Therefore, one would expect that soft scattering dominates the energy loss and that collisions with nuclei of mass $M$ are suppressed by a factor $\OO\l\frac{M}{Z m_e}\r$ as compared to collisions with electrons. This is certainly true for incident charged particles in ordinary matter. However, both of these naive expectations turn out to be false when considering scattering off a degenerate medium. 

To understand the effect of degeneracy, we first consider the energy loss from scattering a high-energy charged particles off non-degenerate targets in the WD. In this case, the stopping power due to collisions with a number density $n$ is given by:
\begin{align}
\label{eq:SP}
\frac{dE}{dx} & = - \int dE' \left(\frac{d \sigma}{dE'}\right) n E' \\
& \sim -\frac{2 \pi n Z^2 \alpha^2}{M_t \beta^2} \log{\l\frac{E_{\text{max}}}{E_{\text{min}}}\r}.
\end{align}
This integration must be performed over all $E'$ within the regime of validity for \eqref{eq:rutherford}, fixing the lower and upper bounds of the ``Coulomb Logarithm". Quantum mechanical uncertainty sets a limit to the accuracy that can be achieved in ``aiming" an incident particle at a target. In terms of impact parameter $b$ for the collision, this translates to a bound $b > \frac{1}{\text{min}\{{m_i, M_t}\} \beta \gamma}$ or, in terms of energy transfer, 
\begin{equation}
E' < E_q = \frac{2 ~\text{min}\{{m_i, M_t}\}^2 Z^2 \alpha^2 \gamma^2}{M_t}
\end{equation}
In addition, the expression for the differential cross section \eqref{eq:rutherford} no longer holds when $E'$ becomes larger than the mass of the target \cite{Rossi}. For our calculations, we take the maximum energy that an incident particle is able to transfer to be $E_{\text{max}} = \text{min}\{E_q, E_{\text{kin}}, M_t\}$. 

On the other hand, the maximum impact parameter $b_{max}$ is determined by charge screening. In a WD, this is simply the screening induced by a degenerate electron gas \textcolor{blue}{similar to a solid} and is given by the Thomas-Fermi length \cite{Teukolsky}
\begin{equation}
l_{\text{sc}} = \l\frac{6 \pi Z e^2 n_e}{E_F}\r^{1/2}\sim \frac{1}{m_e}.
\end{equation}
\textcolor{blue}{This is the equivalent of the Debye length for a degenerate gas at Fermi energy $E_F$}. This corresponds to a lower bound $E_{\text{sc}} = \frac{2 m_e^2 Z^2 \alpha^2}{M_t \beta^2}$. Note that when the incident particle reaches velocity $\beta \gamma \approx \frac{m_e}{\text{min}\{m_i, M_t\}}$, the minimum possible energy transfer due to Thomas-Fermi screening will in fact exceed the maximum. At this point, the form of equation \eqref{eq:SP} becomes modified to avoid pathological, non-negative values of $dE/dx$ and there is negligible stopping power due to collisions. 

However, the lattice structure of ions in the WD introduces further complications \cite{Teukolsky}. The typical lattice binding energy between ions is given by the electric potential 
\begin{equation}
E_B \sim \frac{Z^2 e^2}{n^{-1/3}}.
\end{equation} 
For energy transfers greater than $E_B$, any lattice effects can be safely ignored. On the other hand, momentum transfers from collisions below this threshold may lead to appreciably small energy loss. For simplicity we set the lower bound on nuclei scattering to be $E_{\text{min}} = \text{max} \{E_B,E_{text{sc}}\}$ so that we only consider energy transfers greater than the ionic lattice binding energy. 

When considering collisions with the degenerate electrons, an incident particle transferring energy $E'$ can only scatter those electrons within $E'$ of the Fermi surface. We define a modified density of electrons $n_e(E')$ as:
\begin{equation}
n_e(E') = \left\{
        \begin{array}{ll}
            \displaystyle \int \limits_{E_F -E'}^{E_F}dE ~g(E) & \quad E' \leq E_F \\
            n_e & \quad E_F \leq E'
        \end{array}
    \right.,
\end{equation}
where $g(E)$ is the density of states per unit volume for a three-dimensional free electron gas. This can also be expressed as a suppression of the differential cross section of order $\mathcal{O}(E'/E_F)$ whenever energy less than $E_F$ is transferred. Therefore, unlike in the non-degenerate case, the energy loss due to soft-scatters are in fact subdominant to the contributions from rare, hard-scatters. The stopping power is also highly sensitive to the bounds of integration through a power-law dependence as opposed to the logarithmic sensitivity in the case of a non-degenerate target. We have calculated the stopping power of high-energy particles solely due to Coulomb collisions, differentiating between nuclei and degenerate electron targets. This is shown in Figure \textcolor{blue}{figure} for an electron density $n_e \sim 10^{33} ~\text{cm}^{-3}$. It is found that for heavy incident particles, the stopping power is dominated by collisions with nuclei at low-energies although it is dominated by collisions with degenerate electrons at high-energies. In addition, scattering off degenerate electrons becomes completely screened at a comparatively higher energies. 

\begin{figure}
\includegraphics[scale=.4]{electronenergyloss}
\end{figure}

\begin{figure}
\includegraphics[scale=.4]{muonenergyloss}
\end{figure}

\begin{figure}
\includegraphics[scale=.4]{protonenergyloss}
\end{figure}


\subsection{Hadronic Interactions}

Strong interactions play a key role in the WD detector. For sufficiently energetic particles greater than the nuclear binding energy $E_{nuc} \sim \OO(10 ~\text{MeV})$, nuclear interactions (absorption) result in an $\OO(1)$ number of energetic secondary hadrons (protons, neutrons, pions, etc.) emitted roughly in the direction of the primary particle. These secondary particles approximately split the initial total energy of the primary particle and can collide with other nuclei in the WD. In addition, the nucleus will generally be left in an excited state and relax through the emission of low-energy $\OO(10 ~\text{MeV})$ nucleons (``nuclear evaporation") and photons \cite{Rossi}. The cross section for any such process is approximately set by the nuclear length scale $\sim \text{fm}$. A hadronic shower is a result of all such reactions caused by primary and secondary particles. 

In order to estimate the typical shower length, we assume a primary particle of energy $\epsilon$ interacting with nuclei to produce $N$ secondary particles each of energy $\epsilon/N$. The shower will end once final-state particles reach energies of order $E_{nuc}$. We have ignored the effects of the evaporative stage in each reaction as this will only introduce minor corrections at sufficiently large $\epsilon$. We find a hadronic shower length $L = l_\text{nuc} \l \frac{\log{(\epsilon/E_{nuc})}}{\log{N}}\r$, where $l_\text{nuc}$ is the mean free path for nuclear collisions. For a number density $n \sim 10^{32} ~\text{cm}^{-3}$ and hadronic cross section $\sim 0.1 ~\text{barn}$ \textcolor{blue}{match nuclear data for carbon nonelastic scattering} this results in a shower length $L \approx 10  ~l_\text{nuc} \approx 10^{-6} ~\text{cm}$. Note that this only has a mild, logarithmic sensitivity to the initial energy $\epsilon \gg ~\text{10 ~\text{MeV}}$ and number of secondaries $N \sim \OO(1)$. 

The products of the hadronic shower will be particles of kinetic energy $1-10 ~\text{MeV}$ which are incapable of inducing further nuclear disintegration. Charged hadrons have appreciable electromagnetic interactions with nuclei, while collisions with degenerate electrons are completely screened at these energies. The typical distance for energy transfer is computed by integrating the stopping power \eqref{eq:SP} over the relevant energy range. For instance, in a density of $n \sim 10^{32} ~\text{cm}^{-3}$ protons traverse $\approx 10^{-6} ~\text{cm}$ slowing down from 5 MeV to 1 MeV. \textcolor{blue}{The corresponding distance is significantly larger for charged pions $\sim 10^{-2} ~\text{cm}$, so let us ignore the energy dump from any light final-state hadrons.} On the other hand, electromagnetic couplings are highly suppressed for final-state neutral hadrons. Therefore, the dominant mechanism of energy loss for neutrons will be elastic scattering off nuclei \textcolor{blue}{neutron capture cross section is negligible}. Each scatter transfers a fraction $\l1-\l\frac{m}{m + M}\r^2\r \approx 0.1$ of the neutron energy to nuclei (averaging over scattering angles), where $m$ and $M$ are the masses of the neutron and nuclei, respectively.  At kinetic energies below $\sim \text{MeV}$, the typical energy transfer in an elastic collision is of order the binding energy in the Coulomb lattice $E_B$. Therefore, we do not consider the energy loss of neutrons below $\sim \text{MeV}$. We find that $\OO(10)$ elastic scatters are needed to slow neutrons from 5 MeV to 1 MeV. For a number density $n \sim 10^{32} ~\text{cm}^{-3}$ and elastic cross section $\sim ~\text{barn}$ \textcolor{blue}{set to match nuclear data for carbon nonelastic scattering}, we find that neutrons traverse a total distance (in the form of a random-walk) $\sim 10^{-8} ~\text{cm}$ during this energy deposition. 

\section{Q-balls}
In various supersymmetric extensions of the standard model (SM), non-topological solitons called Q-balls can be produced in the early universe \cite{Coleman:1985ki, Kusenko:1997si}. If these Q-balls were stable, they would comprise a component of the dark matter today. Q-balls can be classified into two groups: supersymmetric electrically charged solitons (SECS) and supersymmetric electrically neutral solitons (SENS). When a neutral baryonic Q-ball interacts with a nucleon, it absorbs its baryonic charge as a minimum-energy configuration and induces the dissociation of the nucleon into free quarks. In this process (known as the ``KKST" process), $\sim \text{GeV}$ of energy is released through the emission of 2-3 pions \cite{Dine:2003ax}. The KKST process provides a useful way to detect such Q-balls. The cross section for interaction is approximately the geometric cross section
\begin{equation}
\sigma_Q \simeq \pi R_Q^2.
\end{equation}
In gauge-mediated models with flat scalar potentials, the Q-ball mass and radius are given by
\begin{equation}
M_Q \sim m_F Q^{3/4}, ~~~ R_Q \sim m_F^{-1} Q^{1/4},
\end{equation}
where $m_F$ is related to the scale of supersymmetry breaking (messenger scale). The condition $M_Q/Q < m_p$ ensures that the Q-ball is stable against decay to nucleons \cite{Dine:2003ax}. 

Note that a sufficiently massive Q-ball will become a black hole if the Q-ball radius is less than the Schwarzschild radius $R_Q \lesssim R_s \sim G M_Q$. In the model described above, this translates into the condition
\begin{equation}
m_F \l\frac{\Mpl}{m_F}\r^3 \lesssim m_Q, ~~~ \l\frac{\Mpl}{m_F}\r^4 \lesssim Q.
\end{equation}
For Q-ball masses of this order, gravitational interactions become relevant while the KKST interaction ceases to exist. 

\subsection{Q-ball Explosiveness}
We assume that for each Q-ball collision, there is equal probability to produce $\pi^0, \pi^+$ and $\pi^-$ under the constraint of charge conservation. Since $\sim 10 ~\text{GeV}$ is released in $\OO(10)$ pions per nuclei dissociation, pions are emitted with velocity $\gamma \approx 5$. The mean distance travelled by a relativistic particle before decaying is $d = \gamma v \tau$. For neutral pions $d_{\pi^0} \sim 10^{-5} ~\text{cm}$ while for charged pions, $d_{\pi^\pm} \sim 10 ~\text{m}$. Note that $d_{\pi^0}$ and $d_{\pi^\pm}$ do not depend on the ambient WD density. 

Numerous experiments have studied the effects of $50 - 500 ~\text{MeV}$ pions incident upon complex nuclei targets such as carbon. It is found that there is roughly equal cross section of order $\sim 0.1 ~\text{barn}$ for a (neutral or charged) pion to either scatter elastically, scatter inelastically, or become absorbed with no final state pion \cite{Pionnuclear}. Of these possibilities, pion absorption is the most relevant for energy loss as this will induce a hadronic shower. During this process, $\sim$ 2-3 hadrons (i.e. protons, neutrons, alpha particles) are emitted. 

\subsection{Q-ball Constraints}

\bibliography{Qballs}
\end{document}

\end{document}


