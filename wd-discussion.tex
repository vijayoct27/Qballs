The detection of ultra-heavy DM is an open problem which will likely require a confluence of astrophysical probes.
Here we present a guide to constraining these candidates through DM-SM scatters, DM-DM annihilations, and DM decays inside a WD that release sufficient SM energy to trigger runaway fusion.
In particular, we calculate the energy loss of high-energy particles due to SM interactions within the WD medium and determine the conditions for which a general energy deposition will heat a WD and ignite SN.
Ultra-heavy DM that produces greater than $10^{16}~\GeV$ of SM particles in a WD is highly constrained by the existence of heavy WDs and the measured SN rate.
The formalism provided will enable WDs to be applied as detectors for any DM model capable of heating the star through such interactions. 
We have done so for baryonic Q-balls, significantly constraining the allowed parameter space in a complementary way to terrestrial searches. 

We have explored very briefly the application of this WD instability to self-gravitational collapse of DM cores, which has very interesting possibilities. 
Such collapsing cores can produce enough heating via multiple annihilations to ignite the star for much smaller DM masses than those considered here, e.g.~$10^7~\GeV$ and can induce SN through other means such as the formation and evaporation of mini black holes. 
These will be addressed in future work~\cite{us}.  

Finally, in addition to the constraints mentioned above, the general phenomenology of these DM-induced runaways will be the ignition of sub-Chandrasekhar mass WDs.
This raises the tantalizing possibility that DM encounters with WDs provide an alternative explosion mechanism for type 1a SN.
For decades it had been widely regarded that type 1a SN are caused by accretion onto carbon-oxygen WDs in binary systems that reach the critical $\approx 1.4 ~M_{\astrosun}$ Chandrasekhar mass limit, although it is now understood that such a process cannot account for all observed SN.
Recent observations~\cite{Scalzo:2014sap, Scalzo:2014wxa} suggest that an $\OO(1)$ fraction of the observed type 1a SN appear to have sub-Chandrasekhar progenitors.
While other mechanisms exist to explain these SN~\cite{Woosley1994,Fink:2007fv,Pakmor:2013wia}, in light of the lack of understanding of DM and its interactions, it is worthwhile to consider whether DM encounters with WDs may also give rise to type 1a SN progenitor.