The detection of ultra-heavy DM is an open problem which will likely require a confluence of astrophysical probes.
Here we present a guide to containing these DM candidates through annihilations, decays, and transits inside a WD that release sufficient SM energy to trigger a type 1a supernova.
In particular, we calculate the energy loss of high-energy particles due to SM interactions within the WD medium and determine the conditions for which a general energy deposition will heat a WD and ignite runaway fusion.
The formalism provided will enable WDs to be applied as detectors for any DM models capable of heating the star through non-gravitational interactions.

In general, the phenomenology of such a DM-induced event will be the ignition of sub-Chandrasekhar mass progenitors.
This raises the tantalizing possibility that DM encounters with a WD can act as an alternative explosion mechanism and progenitor system for type 1a SN.
For decades, it has been widely regarded that type 1a SN are caused by accretion onto carbon-oxygen white dwarfs in binary systems that reach the critical $\approx 1.4 ~M_{\astrosun}$ Chandrasekhar mass limit.
Nevertheless, it is well-known that such a mechanism cannot account for all observed type 1a SN.
Recent observations \cite{Scalzo:2014sap, Scalzo:2014wxa} suggest that an $\OO(1)$ fraction of the observed type 1a SN appear to have sub-Chandrasekhar progenitors.
A leading explanation for this phenomenon is the detonation of a surface layer of helium which drives a shock into the interior of a sub-Chandrasekhar-mass WD \cite{Woosley1994,Fink:2007fv}.
Another possibility is the binary merger of WDs \cite{Pakmor:2013wia}. 
However, in light of the lack of understanding of DM and its interactions, it is worthwhile to consider whether DM encounters with WDs may also give rise to type 1a SN progenitor.