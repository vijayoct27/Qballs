The detection of ultra-heavy DM is an open problem which will likely require a confluence of astrophysical probes.
Here we present a guide to containing these DM candidates through annihilations, decays, and transits inside a WD that release sufficient SM energy to trigger a type 1a supernova.
In particular, we calculate the energy loss of high-energy particles due to SM interactions within the WD medium and determine the conditions for which a general energy deposition will heat a WD and ignite runaway fusion, finding that ultra-heavy DM which is able to produce $\gtrsim 10^{15}~\GeV$ of SM particles in a WD is highly constrained by observation of SN and the lifetimes of WDs. 
The formalism provided will enable WDs to be applied as detectors for any DM models capable of heating the star through non-gravitational interactions.
We have done so for baryonic Q-balls, significantly limiting the allowed parameter space in a complementary way to terrestrial searches. 

We have explored very briefly the application of this WD instability to collapsing cores of DM, which has very interesting possibilities. 
Such cores can produce enough heating via annihilations or decays to ignite the star for much smaller DM masses than those considered here, e.g.~$10^9~\GeV$ and can induce SN though other mechanism such the formation and evaporation of very small black holes. 
The formation and evolution of DM cores at these masses involves some unexplored subtitles, which will address in a future work.  

Finally, in addition to the constraints mentioned above, the general phenomenology of these DM-induced runaways will be the ignition of sub-Chandrasekhar mass WD.
This raises the tantalizing possibility that DM encounters with a WD provide an alternative explosion mechanism for sub-Chandrasekhar WD.
For decades, it has been widely regarded that type 1a SN are caused by accretion onto carbon-oxygen white dwarfs in binary systems that reach the critical $\approx 1.4 ~M_{\astrosun}$ Chandrasekhar mass limit, however, it is well-known that such a mechanism cannot account for all observed type 1a SN.
Recent observations \cite{Scalzo:2014sap, Scalzo:2014wxa} suggest that an $\OO(1)$ fraction of the observed type 1a SN appear to have sub-Chandrasekhar progenitors.
While other mechanisms exit to produce these SN \cite{Woosley1994,Fink:2007fv} \cite{Pakmor:2013wia}, in light of the lack of understanding of DM and its interactions, it is worthwhile to consider whether DM encounters with WDs may also give rise to type 1a SN progenitor.