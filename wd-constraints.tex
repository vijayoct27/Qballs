We now constrain some simplified models of DM which will ignite a WD via one of the processes parameterized in Section \ref{sec:dmignition}. 
These release SM particles that deposit their energy and thermalize ions within a distance described in Section \ref{sec:smheating}. 
First, however, we review how WD observables constrain DM candidates capable of triggering SN.

\subsection{Review of WD Observables}
Following the discussion of \cite{Graham:2015apa}, our constraints come from (1)~the existence of heavy, long-lived white dwarfs, or (2)~the measured type 1a SN rate.
The typical age of a WD is of order the age of the universe $\sim \text{Gyr}$.
RX~J0648.04418 is a nearby star and one of the heavier known WDs, with a mass $\sim 1.25 ~M_{\astrosun}$ \cite{Mereghetti:2013nba} and local dark matter density which we take to be $\rho_\chi \sim 0.4 ~\GeV/\text{cm}^3$.
Of course, this is not the only known heavy WD---the Sloan Digital Sky Survey \cite{SDSS} has found $20+$ others.
 % https://heasarc.gsfc.nasa.gov/db-perl/W3Browse/w3hdprods.pl
The NuStar collaboration has also recently uncovered evidence for the likely existence of heavy WDs near the galactic center \cite{NuStar}, where the DM density is assumed to be much greater $\rho_\chi \gtrsim 10^3 ~\text{GeV}/\text{cm}^3$ \cite{Nesti:2013uwa}.
Such heavy candidates are particularly suited for our constraints as the energy deposit necessary to trigger SN \eqref{eq:energy_boom_condition} is a decreasing function of WD mass.
However, less dense white dwarfs are significantly more abundant in the galaxy.
Thus, even if a sufficiently massive DM is unable to trigger a violent heating event within the lifetime of a WD, it could still ignite enough lighter WDs to affect the measured SN rate of $\sim $ 0.3 per century.
The DM-induced SN rate is estimated using the expected number of white dwarfs per galaxy $\sim 10^{10}$ and their mass distribution \cite{SDSS}.
Simulations indicate that only WD masses heavier than $\sim 0.85 ~M_{\astrosun}$ will result in optically visible SN \cite{Graham:2015apa}.
Therefore, most of the stars exploded in this manner will be in the mass range $\sim 0.85 - 1 ~M_{\astrosun}$, resulting in weaker SN than expected of typical Chandrasekhar mass WDs.

To summarize, a bound on DM parameters can be placed if either a single explosive event occurs during the lifetime of an observed star such as RX~J0648.04418, or the SN rate due to such DM events throughout the galaxy exceeds the measured value.
Note that for low-mass WDs dominated by photon diffusion, $\Eboom$ is a strong function of WD density.
In \cite{Graham:2015apa} the central WD density is used to constrain black hole transits with the justification that the density is nearly constant for much of the star.
The average density for WDs is typically a factor $\sim 10^{-2} - 10^{-1}$ less than the central density, although it is found that the WD density only changes by an $\OO(1)$ fraction from the central value up to a distance $\sim R_\text{WD}/2$ \cite{Chandrasekhar}.
Therefore the central density is a valid approximation as long as we consider heating events within this ``modified" WD volume.
For simplicity, we employ this approach.

\subsection{Inelastic Scattering Constraints}
\label{sec:TransitConstraints}

In order to constrain a DM model with an inelastic scattering interaction, we require that it satisfy the ignition condition \eqref{eq:transitexplosion}.
This is given in terms of an LET, which parameterizes the ability for DM to release sufficient energy to the star in the form of SM particles.
$(dE/dx)_\text{LET}$ for any realistic DM model would necessarily involve a sum over stellar targets along with species that could be produced, as well as an integral over the produced particle spectrum.
However, we will consider a highly simplified interaction in which $\sigma_{Ni\epsilon}$ denotes the cross-section for DM to scatter off a stellar constituent (e.g. ions), producing $N$ particles of SM species $i$ and individual energy $\epsilon$.
If this were the only available channel for the DM to deposit energy, then the LET could be written as
\begin{align}
\label{eq:schematicLET}
  \left( \frac{d E}{d x} \right)_\text{LET} = n_\text{ion} \sigma_{Ni\epsilon} N\epsilon.
\end{align}
Additionally, one may consider the case that the LET $(dE/dx)_\text{LET}$ and DM stopping power $(dE/dx)_\text{SP}$ are equal---that is, the DM loses kinetic energy at the same rate as energy is deposited to the WD.
While such a statement is certainly not true for all DM models (such as the Q-ball, which liberates binding energy rather than transferring kinetic energy), it provides a useful benchmark to express constraints.

With the above schematic for DM-SM scattering, we constrain the inelastic scattering cross section $\sigma_{Ni\epsilon}$ as a function of DM mass $m_\chi$.
This is done in Figure \ref{fig:transitclasses} using the different classes of observation available and for a representative choice of $\epsilon$---as described in Section~\ref{sec:smheating}, there is minimal dependence of the constraint on SM species $i$.
In the case of neutrinos, we may simply demand that $\epsilon$ is sufficiently large that a single neutrino can ignite the star. 

\subsection{Collision and Decay Constraints}
\label{sec:CollisionConstraints}

In order to constrain a DM model through its annihilations or decays within a WD, we require that it satisfy the ignition condition \eqref{eq:coldecay}.
Consider a simplified interaction wherein a single annihilation or decay releases $N$ particles of SM species $i$ and energy $\epsilon$.
Assuming a fractional parameter $f_\text{SM} = 1$, this corresponds to SM products with individual energy $\epsilon \sim m_\chi/N$.
Again, as long as $\epsilon \gg \MeV$ there is minimal dependence of the constraints on number of particles $N$ or species $i$ (with the exception of neutrinos).

In Figures~\ref{fig:collisionclasses} and ~\ref{fig:decayclasses}, we show the constraints on $\sigma_{\chi \chi}$ and $\tau_\chi$ for transiting DM using the different classes of observation available for DM-DM collisions and DM decays, respectively. 
For simplicity, we examine the constraints on captured DM in Figures~\ref{fig:capturecollision} and \ref{fig:capturedecay} only due to the existence of a nearby, heavy WD. 
Here, we also specify a benchmark value for the DM-nuclear cross section $\sigma_{\chi A}$ and, in the case of DM-DM collisions, a stabilizing radius for the DM collapse.

It is important to note that there are additional limits on DM interactions of this kind, complementary to the limits placed from WDs.
For one, demanding that the galactic halo has not substantially depleted during its lifetime yields a cosmological bound on DM self-interactions $\frac{\sigma_{\chi \chi}}{m_\chi} < \frac{\text{b}}{\GeV}$.
This is similar in magnitude to the bounds from colliding galaxy clusters \cite{Randall:2007ph}.
There is also a cosmological bound on DM lifetime $\tau_\chi > 100 ~\text{Gyr}$, independent of the nature of the decay products \cite{Poulin:2016nat}.
In addition, DM annihilations/decays in the galactic halo will necessarily contribute to the cosmic ray (CR) flux seen in terrestrial detectors.
As CRs of energy greater than $10^{12} ~\GeV$ have not yet been observed \cite{ThePierreAuger:2015rha, AbuZayyad:2012ru}, this places a bound on DM interaction parameters $\sigma_{\chi \chi}$ and $\tau_\chi$ which involve the release of such ultra-high energy particles.
The CR constraint on DM can be estimated by requiring that the expected time for an event to strike earth is less than the typical lifetime of a terrestrial detector $\sim 10 ~\text{yr}$.
For a detector of area $\sim (50~\text{km})^2$ \cite{ThePierreAuger:2015rha}, we find that the CR bounds are generally weaker than but within a few orders of magnitude of the WD bounds in the transit scenario. 
This is actually due to a coincidence in the effective ``space-time volumes" of the two systems.
A CR detector sees events within a space-time volume $\sim (R_\text{det}^2 R_\text{halo} \times 10 ~\text{yr})$ which is comparable to the WD space-time volume $\sim (R_\text{WD}^3 \times 10^9 ~\text{yr})$, including the additional gravitational enhancement factors.   
Of course a constraint can also be placed on lower-energy SM products from DM annihilations or decays which would provide an additional source for the measured CR flux, although such an analysis is beyond the scope of this work.
