%\documentclass[10pt, twocolumn]{article}
%\documentclass[twocolumn,showpacs,preprintnumbers,amsmath,amssymb,prl]{revtex4}
\documentclass[11 pt, preprint,preprintnumbers,amsmath,amssymb, prd]{revtex4}

% Preamble adapted from Surjeet Rajendran

\usepackage{latexsym}
\usepackage{amssymb}
\usepackage{epsfig,amsmath,graphics}
\usepackage{epstopdf}
\usepackage{verbatim}
\usepackage{wasysym}
\usepackage{hyperref}
\usepackage{feynmp-auto} % feynman diagrams
%\usepackage{subfig}
\usepackage[utf8]{inputenc}
\usepackage{xpatch}
\usepackage{xcolor}
\hypersetup{
    colorlinks,
    linkcolor={red!80!black},
    citecolor={green!60!black},
    urlcolor={blue!60!black}
}

\newcommand{\OO}{\mathcal{O}}
\newcommand{\LL}{\mathcal{L}}
\newcommand{\HH}{\mathcal{H}}

\newcommand{\GeV}{\text{GeV}}
\newcommand{\rad}{\text{rad}}
\newcommand{\angstrom}{\buildrel _{\circ} \over {\mathrm{A}}}
\newcommand{\pslash}{p\hspace{-0.070in}/\,}
\newcommand{\Mpl}{M_{\text{pl}}}
\newcommand{\ket}[1]{\ensuremath{\left|#1\right>}}
\newcommand{\bra}[1]{\ensuremath{\left<#1\right|}}
\newcommand{\braket}[2]{\ensuremath{\left<#1|#2\right>}}

%Large Parentheses
\def\r{\right)}
\def\l{\left(} 
\begin{document}

\section{Overview}

Localized heating of a white dwarf has the potential to ignite the star. Namely, if a region of size $\lambda_T$ or greater is raised to a critical temperature $T_f$, this would initiate runaway thermonuclear fusion and cause the white dwarf to explode in a supernovae. According to \cite{Woosley}, $\lambda_T \sim 10^{-5} ~\text{cm}$ for carbon-oxygen white dwarfs at the highest densities $\rho \sim 5 \times 10^9 ~\frac{\text{gm}}{\text{cm}^3}$. This is then analytically scaled in \cite{Varela} for varying densities. \textcolor{red}{Get values from Surjeet}

Consider an ultra-heavy dark matter (DM) state transit through the white dwarf. Assume that the DM interacts with white dwarf constituents (ions or electrons) in a general manner as shown in Figure \ref{fig:feynmandiag}, releasing $n_i$ particles of species $i$ each with kinetic energy $\epsilon$. The cross section for this interaction is denoted as $\sigma_{i,\epsilon}$.  

Of course, this released energy must be transferred to the stellar medium in order for the white dwarf to be heated. For a given particle type, each value of $\epsilon$ is characterized by a distance $R_\epsilon$ from the point of release over which it is deposited. This ``range", and therefore the nominal size of the resulting hot region, is set by the various ways in which the emitted particle interacts with the stellar constituents and is able to dumps its energy. In particular, we define $R_\epsilon$ as the distance over which a particle $i$ and any secondaries transfer $\OO(1)$ of the initial energy $\epsilon$ to electrons or ions in the white dwarf. To demonstrate the significance of this parameter, suppose that the DM simply scattered off nuclei elastically with no particles released. In this case, $R_\epsilon$ effectively vanishes as $\epsilon$ is transferred directly to the kinetic energy of nuclei. However, in the other extreme limit, suppose the interaction released energy into neutrinos. In this case, $R_\epsilon$ is of astronomical length scales and there would be no chance of thermalizing any local region in the star. 

Consider the two possibilities relevant for ignition. If $R_\epsilon> \lambda_T$, the DM must deposit a minimum energy $E_{min} \sim R_\epsilon^3 n T_f$ in order to heat up the entire region of size $R_\epsilon$ to the critical temperature $T_f$, where $n$ is the number density of nuclei in the white dwarf. On the other hand, if $R_\epsilon < \lambda_T$ then the minimum energy required is independent of $R$ and given by $E_{min} \sim \lambda_T^3 n T_f$. Setting $T_f \sim \text{MeV}$, $\lambda_T \sim 10^{-5} ~\text{cm}$, and $n \sim 10^{32} ~\text{cm}^{-3}$, we find that an energy $E_{min} \sim 10^{14} ~\text{GeV}$ transferred to the white dwarf within a localized region smaller than $\lambda_T$ will eventually trigger runaway fusion. This ``explosion" energy must be compared to the total energy released during the DM transit over a distance $\text{min}(\lambda_T, R_\epsilon)$. As a result, we find a lower bound on the interaction cross section $\sigma_{i,\epsilon}$ sufficient to trigger runaway fusion: 
\begin{equation}
\label{eq:explosion}
n_i \sigma_{\epsilon,i} \gtrsim \left\{
        \begin{array}{ll}
            \displaystyle \lambda_T^2 \l \frac{T_f}{\epsilon} \r & \quad R_\epsilon < \lambda_T \\
             \lambda_T^2 \l \frac{R_\epsilon}{\lambda_T}\r^2 \l \frac{T_f}{\epsilon} \r & \quad R_\epsilon > \lambda_T
        \end{array}
    \right..
\end{equation}

Note that in deriving this explosiveness bound, we have assumed the DM transit time is less than the corresponding diffusion time. After a time $\Delta t$ the DM has traversed $v_{esc} \Delta t$, where the velocity is set by the escape velocity of a white dwarf. Therefore, this amounts the following condition:
\begin{equation}
\begin{array}{ll}
             \tau_d^{\lambda_T, T_f} \gtrsim \frac{\lambda_T}{v_{esc}} & \quad R_\epsilon < \lambda_T \\
            \tau_d^{R_\epsilon, T_f} \gtrsim \frac{R_\epsilon}{v_{esc}}  & \quad R_\epsilon > \lambda_T
        \end{array},
\end{equation}
where $\tau_d^{\lambda_T, T_f}$ is the time for a region of size $\lambda_T$ and temperature $T_f$ to diffuse $\OO(1)$ of its heat. Within the validity of the heat equation this is simply given by $\tau_d^{\lambda_T, T_f} \sim \frac{\lambda_T^2}{\alpha}$, where $\alpha$ is the (temperature-dependent) diffusivity. We also assume that the time to transfer energy $\epsilon$ out to its characteristic range $R_\epsilon$, denoted as $\tau_\epsilon$, is less than the diffusion time scale $\tau_d^{\lambda_T, T_f}$. 

\section{Determination of $R_\epsilon$}
In this section, we calculate how different types of standard model particles can lose energy to the white dwarf. Note that a given DM transit will be most explosive for higher mass white dwarfs. This can be seen explicitly in the density dependence of $\lambda_T$. We consider densities in the range $\rho \sim 10^{6} - 10^{9} ~\frac{\text{gm}}{\text{cm}^3}$. For carbon-oxygen white dwarfs, this translates to $n \sim 10^{29} - 10^{32} ~\text{cm}^{-3}$ and $n_e \sim 10^{30} - 10^{33} ~\text{cm}^{-3}$ for number densities of nuclei and electrons, respectively. \textcolor{red}{Someone check these density numbers} Over this range of densities, the trigger size approximately varies between $\lambda_T \sim 10^{-5} ~\text{cm} -10^{-2} ~\text{cm}$. 

The interior of a (carbon-oxygen) white dwarf is a complex environment. Famously, the star is supported against collapse by the Fermi pressure of a degenerate electron gas. These electrons are at a characteristic Fermi energy $E_F \sim n_e^{1/3} \sim \OO(\text{MeV})$. In addition, the nuclei are at an ambient temperature $T \sim \text{keV}$ and form a strongly-coupled plasma wherein the plasma parameter $\Gamma = \frac{Z e^2}{n^{1/3} T} \gg 1$ indicates that the Coulomb potential between nuclei is far greater than the thermal energy. Therefore, the screening length is determined by the Thomas-Fermi distance. 

\subsection{Photons}

\subsection{Light Charged Particles}

\subsection{Heavy Charged Particles}

\subsection{Neutral Hadrons}

\section{Q-ball properties}
In various supersymmetric extensions of the standard model (SM), non-topological solitons called Q-balls can be produced in the early universe. If these Q-balls were stable, they would comprise a component of the dark matter today. Q-balls can be classified into two groups: supersymmetric electrically charged solitons (SECS) and supersymmetric electrically neutral solitons (SENS). When a neutral baryonic Q-ball interacts with a nucleon, it absorbs its baryonic charge as a minimum-energy configuration and induces the dissociation of the nucleon into free quarks. In this process (known as the ``KKST" process), $\sim \text{GeV}$ of energy is released through the emission of 2-3 pions. The KKST process provides a useful way to detect such Q-balls. The cross section for interaction is approximately the geometric cross section
\begin{equation}
\sigma_Q \simeq \pi R_Q^2.
\end{equation}
In gauge-mediated models with flat scalar potentials, the Q-ball mass and radius are given by
\begin{equation}
M_Q \sim m_F Q^{3/4}, ~~~ R_Q \sim m_F^{-1} Q^{1/4},
\end{equation}
where $m_F$ is related to the scale of supersymmetry breaking (messenger scale). The condition $M_Q/Q < m_p$ ensures that the Q-ball is stable against decay to nucleons. 

Note that a sufficiently massive Q-ball will become a black hole if the Q-ball radius is less than the Schwarzschild radius $R_Q \lesssim R_s \sim G M_Q$. In the model described above, this translates into the condition
\begin{equation}
m_F \l\frac{\Mpl}{m_F}\r^3 \lesssim m_Q, ~~~ \l\frac{\Mpl}{m_F}\r^4 \lesssim Q.
\end{equation}
For Q-ball masses of this order, gravitational interactions become relevant while the KKST interaction ceases to exist. 

\section{Range of the KKST process}

We assume that for each Q-ball collision, there is equal probability to produce $\pi^0, \pi^+$ and $\pi^-$ under the constraint of charge conservation. Since $\sim 10 ~\text{GeV}$ is released in $\OO(10)$ pions per nuclei dissociation, pions are emitted with velocity $\gamma \approx 5$. The mean distance travelled by a relativistic particle before decaying is $d = \gamma v \tau$. For neutral pions $d_{\pi^0} \sim 10^{-5} ~\text{cm}$ while for charged pions (\textcolor{blue}{which decay via weak interactions and have characteristically longer lifetimes)}, $d_{\pi^\pm} \sim 10 ~\text{m}$. Note that $d_{\pi^0}$ and $d_{\pi^\pm}$ do not depend on the ambient white dwarf density. 

Here we enumerate the different ways in which particles such as pions can interact with the white dwarf constituents, i.e. nuclei and free electrons. The main difference between charged and neutral particles is that the latter do not have appreciable electromagnetic interactions. \textcolor{blue}{These couplings are typically suppressed by higher dimension operators.} For charged particles, Coulomb scattering is a useful mechanism for energy transfer. Generically, an incident (spin-0) particle of mass $m_i$, charge $Ze$, and velocity $\beta$ scattering off a target $m_t$ of charge $Z'e$ is described by the ``Rutherford" differential cross section \textcolor{blue}{Originally derived by Bhabha?}
\begin{equation}
\label{eq:rutherford}
\frac{d \sigma (E', \beta)}{dE'} = \frac{2 \pi  \alpha^2 Z Z'}{m_t \beta^2} \frac{1}{E'^2} \l1- \frac{\beta^2 E'}{E_{max}}\r, 
 \end{equation}
where we have assumed a sufficiently fast incident particle so that interactions are governed by single collisions with energy transfer $E'$ \cite{Rossi}. $E_{max}$ denotes the maximum energy transfer possible satisfying kinematic constraints \textcolor{blue}{target at rest and zero relative angle between incoming incident and outgoing target momenta}:
\begin{equation}
E_{max} = \frac{2 m_t \beta^2 \gamma^2}{1+ 2\gamma(m_t/m_i) +(m_t/m_i)^2}. 
\end{equation}

\textcolor{blue}{Of course, the typical temperature of the white dwarf interior is $\sim 10^{7} ~\text{K} \sim \text{keV}$, so no electron is actually at rest. In fact, the fastest electrons have momentum of order the Fermi momentum $p_F \sim E_F \sim n^{1/3} \approx \text{MeV}$ (the approximation of an extreme relativistic Fermi gas is valid since $m_e \lesssim p_F$). Numerically, we find that $E_{max} \sim 10 ~\text{MeV}$ for a pion of kinetic energy $\sim 500 ~\text{MeV}$.} For sufficiently heavy incident particles, the differential cross section depends only on the velocity of the incident particle. Note that higher-spin particles receive additional corrections to the cross section, but for small energy transfers these corrections are negligible.

It is straightforward to understand the parametric dependences of \eqref{eq:rutherford}: there is increased likelihood to scatter for slowly moving incident particles undergoing ``soft-scatters" against lighter targets. Therefore, one would expect that soft scattering dominates the energy loss and that collisions with nuclei are suppressed by a factor $\OO\l\frac{m_C}{Z m_e}\r$ as compared to collisions with electrons. This is certainly true for incident charged particles in ordinary matter. \textcolor{blue}{For instance, the dominant source of stopping power for $\sim \text{GeV}$ heavy charged particles such as pions in matter is scattering (ionization) off bound electrons. Although there are additional losses due to radiative effects, these only become comparable at significantly higher energies $\OO(1000~\text{GeV})$}. However, both of these naive expectations turn out to be false when considering scattering off a degenerate electron gas. 

To understand the effect of degeneracy, we first consider the energy loss from scattering off non-degenerate free electrons. In general, the stopping power due to collisions with a number density $n$ is given by:
\begin{equation}
\label{eq:SP}
\frac{dE}{dx} = - \int dE' \left(\frac{d \sigma_R}{dE'}\right) n E'.
\end{equation}
\textcolor{blue}{For heavy charged particles the resulting stopping power is of the form $\frac{dE}{dx} \propto \frac{1}{\beta^2} \log{\beta^2}$, setting the upper limit of integration to be equal to $E_{max}$. Actually, this integration must be performed over all $E'$ within the (classical) regime of validity for \eqref{eq:rutherford}. However, it is important to note that the resultant stopping power only depends logarithmically on these bounds.} The range of a particle is defined as the distance travelled along its own trajectory before effectively coming to rest. In terms of the stopping power, this is simply
\begin{equation}
\label{eq:range}
L = \int dE \left(\frac{dE}{dx}\right)^{-1},
\end{equation}
integrated over the incident particle kinetic energy. For a charged pion of initial kinetic energy $\sim 500 ~\text{MeV}$ scattering of an electron density $n \sim 10^{33} ~\text{cm}^{-3}$, we find that $L \sim 10^{-8} ~\text{cm}$. 

Of course, the electrons in a white dwarf are famously degenerate with Fermi energy $E_F \sim \OO(\text{MeV})$. Therefore, for a given energy transfer $E'$ an incident particle can only scatter those electrons within $E'$ of the Fermi surface. We define a modified density of electrons $n(E')$ as:
\begin{equation}
n(E') = \left\{
        \begin{array}{ll}
            \displaystyle \int \limits_{E_F -E'}^{E_F}dE ~g(E) & \quad E' \leq E_F \\
            n & \quad E_F \leq E'
        \end{array}
    \right.,
\end{equation}
where $g(E)$ is the density of states per unit volume for a three-dimensional free electron gas. This can also be expressed as a suppression of the differential cross section of order $\mathcal{O}(E'/E_F)$ whenever energy less than $E_F$ is transferred. Therefore, unlike in the non-degenerate case, the energy loss due to soft-scatters are in fact subdominant to the contributions from rare, hard-scatters. 

\textcolor{blue}{In addition, the resulting stopping power is no longer logarithmically sensitive to the minimum and maximum energy deposits allowed. Therefore, it becomes important to consider values for the energy transfer at which point \eqref{eq:rutherford} breaks down. According to the derivation in \cite{Rossi}, the energy transfer to a target electron is given in terms of an impact parameter $b$ (distance of closest approach): $E' = \frac{2 m_e Z r_e^2}{\beta^2 b^2}$. However, quantum mechanical uncertainty sets a limit to the accuracy that can be achieved in ``aiming" an incident particle at a target electron. This translates to a bound $b > \frac{1}{\beta \gamma m_e}$ or, in terms of energy transfer, $E' < E_q = 2 m_e \gamma^2 \alpha^2$. This quantum correction, which becomes important as $E'$ approaches $E_q$, causes $E'$ to increase with deceasing $b$ less rapidly than would otherwise be expected from $\eqref{eq:rutherford}$. Qualitatively, this has the effect of curtailing the maximum energy at which an incident particle is able to scatter. Of course, this upper bound has virtually no effect in the non-degenerate case for which the energy loss was instead dominated by soft-scattering.}

We find that for degenerate electron scattering, a charged pion of initial kinetic energy $\sim 500 ~\text{MeV}$ at electron density $n \sim 10^{33} ~\text{cm}^{-3}$ has a range of $L \sim 10^{-2} ~\text{cm}$. However, the corresponding range for scattering off nuclei of density $n_C \sim 10^{32} ~\text{cm}^{-3}$ is $L \sim 10^{-4} ~\text{cm}$. Therefore, the leading source of electromagnetic energy loss for $\OO(\GeV)$ heavy charged particles in the white dwarf is Coulomb scattering off nuclei. \textcolor{blue}{Note that the stopping power due to radiative effects i.e. bremsstrahlung only becomes comparable to Coulomb scattering off nuclei to at much higher energies - LPM suppression}. 

Because of the effect of degeneracy on electromagnetic energy loss, nuclear interactions play a key role in determining the range of the KKST process in the white dwarf. The cross section for any nuclear interaction is approximately set by the nuclear length scale $\sim \text{fm}^2$. Numerous experiments have studied the effects of $50 - 500 ~\text{MeV}$ pions incident upon complex nuclei targets such as carbon. It is found that there is approximately equal cross section of order $\OO(100 ~\text{mb})$ for a (neutral or charged) pion to either scatter elastically, scatter inelastically, or become absorbed with no final state pion. \textcolor{blue}{Elastic scattering is not a dominant source of energy loss due to large nuclei masses.} Of these possibilities, pion absorption is the most relevant for energy loss. During this process, an incident pion is absorbed in the nucleus and transfers energy greater than the typical binding energy per nucleon $\sim 10 ~\text{MeV}$. This leads to the emission of $\sim 2-4$ protons, neutrons, or deuterons with an $\OO(1)$ fraction of the total initial energy split among the final states. Furthermore, at high energies these emitted nucleons have considerable $\sim 100 ~\text{mb}$ nonelastic nuclear cross sections which result in multiple final state hadrons including protons, neutrons, pions, etc. The details of these interactions are beyond the scope of this work and typically involve complicated nuclear dynamics. \textcolor{blue}{spallation - nuclear evaporation} Regardless, the process qualitatively resembles a ``hadronic shower" in which an initial high energy nucleon eventually produces many lower energy hadronic final states.

At a nuclear density $n_C \sim 10^{32} ~\text{cm}^{-3}$, the mean free path for nuclear interactions of cross section $\sim 100 ~\text{mb}$ is given by $l_n \sim 10^{-7} ~\text{cm}$. Therefore, the range of KKST process is determined as follows. In terms of $l_n$, the initial pions and resulting high-energy ``shower" traverse a distance $\sim \text{few} \times l_n$ until final states of $\sim \text{MeV}$ energy are produced. At sufficiently low energies the range of charged hadrons due to electromagnetic scattering off nuclei \eqref{eq:range} becomes comparable to $l_n$, and at this point these particles immediately stop. For protons in the specified density, this critical energy occurs at $\OO(10 ~\text{MeV})$. As for long-lived neutral hadrons i.e. neutrons, the nonelastic cross section effectively vanishes $\sim \text{MeV}$. At this point, the dominant stopping mechanism is nuclear elastic scattering. For an $\sim \text{MeV}$ neutron interacting with C and O nuclei, it is found that $\OO(100)$ elastic collisions are needed to sufficiently slow down to thermal velocity. Therefore, neutrons have to traverse an additional distance $\sim 10 \times l_n$ in the form of a random walk before stopping. In summary, an $\OO(1)$ fraction of the energy deposited in the KKST process gets transferred within a range $R \sim 10^{-7} - 10^{-6} ~\text{cm}$.  

\end{document}


