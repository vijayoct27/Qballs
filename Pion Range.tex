%\documentclass[10pt, twocolumn]{article}
%\documentclass[twocolumn,showpacs,preprintnumbers,amsmath,amssymb,prl]{revtex4}
\documentclass[11 pt, preprint,preprintnumbers,amsmath,amssymb, prd]{revtex4}

% Preamble adapted from Surjeet Rajendran

\usepackage{latexsym}
\usepackage{amssymb}
\usepackage{epsfig,amsmath,graphics}
\usepackage{epstopdf}
\usepackage{verbatim}
\usepackage{wasysym}
\usepackage{hyperref}
\usepackage{feynmp-auto} % feynman diagrams
%\usepackage{subfig}
\usepackage[utf8]{inputenc}
\usepackage{xpatch}
\usepackage{xcolor}
\hypersetup{
    colorlinks,
    linkcolor={red!80!black},
    citecolor={green!60!black},
    urlcolor={blue!60!black}
}

\newcommand{\OO}{\mathcal{O}}
\newcommand{\LL}{\mathcal{L}}
\newcommand{\HH}{\mathcal{H}}

\newcommand{\GeV}{\text{GeV}}
\newcommand{\rad}{\text{rad}}
\newcommand{\angstrom}{\buildrel _{\circ} \over {\mathrm{A}}}
\newcommand{\pslash}{p\hspace{-0.070in}/\,}
\newcommand{\Mpl}{M_{\text{pl}}}
\newcommand{\ket}[1]{\ensuremath{\left|#1\right>}}
\newcommand{\bra}[1]{\ensuremath{\left<#1\right|}}
\newcommand{\braket}[2]{\ensuremath{\left<#1|#2\right>}}

%Large Parentheses
\def\r{\right)}
\def\l{\left(} 
\begin{document}

\section{Q-ball properties}
In various supersymmetric extensions of the standard model (SM), non-topological solitons called Q-balls can be produced in the early universe. If these Q-balls were stable, they would comprise a component of the dark matter today. Q-balls can be classified into two groups: supersymmetric electrically charged solitons (SECS) and supersymmetric electrically neutral solitons (SENS). When a neutral baryonic Q-ball interacts with a nucleon, it absorbs its baryonic charge as a minimum-energy configuration and induces the dissociation of the nucleon into free quarks. In this process (known as the ``KKST" process), $\sim \text{GeV}$ of energy is released through the emission of 2-3 pions. The KKST process provides a useful way to detect such Q-balls. The cross section for interaction is approximately the geometric cross section
\begin{equation}
\sigma_Q \simeq \pi R_Q^2.
\end{equation}
In gauge-mediated models with flat scalar potentials, the Q-ball mass and radius are given by
\begin{equation}
M_Q \sim m_F Q^{3/4}, ~~~ R_Q \sim m_F^{-1} Q^{1/4},
\end{equation}
where $m_F$ is related to the scale of supersymmetry breaking (messenger scale) and is at least of $\mathcal{O}(\text{TeV})$. The condition $M_Q/Q < m_p$ ensures that the Q-ball is stable against decay to nucleons. 

Note that a sufficiently massive Q-ball will become a black hole if the Q-ball radius is less than the Schwarzschild radius $R_Q \lesssim R_s \sim G M_Q$. In the model described above, this translates into a condition on the Q-ball interaction cross section 
\begin{equation}
\sigma_Q \lesssim \frac{\Mpl^2}{m_F^4}. 
\end{equation}
For cross sections of this order, gravitational interactions become relevant. In fact, values of $\sigma_Q$ greater than this bound have no meaning since black holes do not interact via the KKST process. 

\section{Q-ball Explosiveness}

Localized heating of a white dwarf has the potential to ignite the star. Namely, if a region of size $\lambda_T$ or greater is raised to a critical temperature $T_f$, this would initiate runaway thermonuclear fusion and cause the white dwarf to explode in a supernovae. According to \cite{Woosley}, $\lambda_T \sim 10^{-5} ~\text{cm}$ for white dwarf densities $\sim 5 \times 10^9 ~\frac{\text{gm}}{\text{cm}^3}$. This is then analytically scaled in \cite{Varela} for varying densities. 

Consider a Q-ball (or similar dark matter candidate) transit through the white dwarf. The energy released per distance travelled is $n_C \sigma_Q \epsilon$, where $\epsilon$ is the typical energy released per nuclei collision and $n_C$ is the number density of nuclei (denoted as $C$ for simplicity). Of course, this released energy must be transferred to the stellar medium in order for the white dwarf to be heated. In particular, any $\epsilon$ is characterized by a length $R$ from the point of release over which it is deposited. This range, and therefore the nominal radius of the resulting hot cylindrical region, is set by the various standard model processes by which the energy is released and subsequently interacts with stellar constituents. To demonstrate the significance of $R$, suppose each Q-ball collision resulted in a simple elastic scattering process. In this case, $R$ effectively vanishes as $\epsilon$ is transferred directly to the kinetic energy of nuclei. However, in the other extreme limit, suppose $\epsilon$ were released purely into neutrinos. In this case, $R$ is of astronomical length scales and the released energy would leave the white dwarf before having a chance to thermalize any region. 

Therefore, it is necessary to determine the relevant $R$ for a Q-ball transit, during which $\epsilon \sim 10 ~\GeV$ of nuclear energy is released per collision in the form of energetic pions. This is done in \textcolor{blue}{section}. For now, consider the two possibilities relevant for ignition: if $R> \lambda_T$, then the Q-ball must deposit a minimum energy $E_{min} \sim R^3 n_C T_f$ in order to heat up the entire region of size $R$ to the critical temperature $T_f$. On the other hand, if $R < \lambda_T$ then the minimum energy required is independent of $R$ and given simply by $E_{min} \sim \lambda_T^3 n_C T_f$. Setting $T_f \sim \text{MeV}$, $\lambda_T \sim 10^{-5} ~\text{cm}$, and $n_C \sim 10^{32} ~\text{cm}^{-3}$, we find that an energy $E_{min} \sim 10^{14} ~\text{GeV}$ transferred to the white dwarf within a localized region smaller than $\lambda_T$ will eventually trigger runaway fusion. 

After a time $\Delta t$ the Q-ball has traversed $v_Q \Delta t$, where the Q-ball velocity is set by the escape velocity of the white dwarf $v_Q \approx 2 \times 10^{-2}$. Since we are interested in ignition, the distance travelled should be set to $\lambda_T$ over a time $\Delta t = \frac{\lambda_T}{v_Q}$. As a result, distinguishing between the possible values for $R$, we find a lower bound on the Q-ball cross section sufficient to trigger runaway fusion: 
\begin{equation}
\label{eq:explosion}
\sigma_Q \gtrsim \left\{
        \begin{array}{ll}
            \displaystyle \lambda_T^2 \l \frac{T_f}{\epsilon} \r & \quad R < \lambda_T \\
             \lambda_T^2 \l \frac{R}{\lambda_T}\r^3 \l \frac{T_f}{\epsilon} \r & \quad R > \lambda_T
        \end{array}
    \right..
\end{equation}

Note that in deriving this explosiveness condition, we have assumed the Q-ball transit time $\Delta t$ is less than the characteristic diffusion time scale. This ensures that the heated region remains in the form of a cylinder and no deposited energy is wasted in diffusion before a sphere of radius $\lambda_T$ is eventually raised to temperature $T_f$ . We also assume that the time to transfer energy $\epsilon$ out to its characteristic range $R$ is less than the diffusion time. We show that these assumptions are indeed satisfied in \textcolor{blue}{section}. Thus, while a given Q-ball transit might still be explosive for shorter diffusion times, there is no need to determine the resulting modifications to \eqref{eq:explosion}. 

\section{Diffusion}
This diffusion time scale for a region of size $\lambda_T$ to lose $\OO(1)$ of its energy is given by $\tau_d \sim \frac{\lambda_T^2}{\alpha}$, where $\alpha$ is the (temperature-dependent) diffusivity. 

\section{Range of KKST process}

In this section we calculate the expected range of the KKST process in the white dwarf. In particular, it will be shown that $R < \lambda_T$ for all white dwarf densities under consideration. This means that the more comprehensive constraint in \eqref{eq:explosion} can be employed. 

Q-balls, or any other dark matter trigger, will be most explosive for higher mass white dwarfs. This can be seen explicitly in the density dependence of $\lambda_T$. We consider densities in the range $\rho \sim 10^{6} - 10^{9} ~\frac{\text{gm}}{\text{cm}^3}$. For carbon-oxygen white dwarfs, this translates to $n_C \sim 10^{29} - 10^{32} ~\text{cm}^{-3}$ and $n \sim 10^{30} - 10^{33} ~\text{cm}^{-3}$ for number densities of nuclei and electrons, respectively. \textcolor{red}{Someone check these density numbers} Over this range of densities, the trigger size approximately varies between $\lambda_T \sim 10^{-5} ~\text{cm} -10^{-2} ~\text{cm}$ \cite{Varela}. 

We assume that for each Q-ball collision, there is equal probability to produce $\pi^0, \pi^+$ and $\pi^-$ under the constraint of charge conservation. Since $\sim 10 ~\text{GeV}$ is released in $\OO(10)$ pions per nuclei dissociation, pions are emitted with velocity $\gamma \approx 5$. The mean distance travelled by a relativistic particle before decaying is $d = \gamma v \tau$. For neutral pions $d_{\pi^0} \sim 10^{-5} ~\text{cm}$ while for charged pions, \textcolor{blue}{which decay via weak interactions and have characteristically longer lifetimes,} $d_{\pi^\pm} \sim 10 ~\text{m}$. Note that $d_{\pi^0}$ and $d_{\pi^\pm}$ do not depend on the ambient white dwarf density. 

Here we enumerate the different ways in which particles such as pions can interact with the white dwarf constituents, i.e. nuclei and free electrons. The main difference between charged and neutral particles is that the latter do not have appreciable electromagnetic interactions. \textcolor{blue}{These couplings are typically suppressed by higher dimension operators.} For charged particles, Coulomb scattering is a useful mechanism for energy transfer. Generically, an incident (spin-0) particle of mass $m_i$, charge $Ze$, and velocity $\beta$ scattering off a target $m_t$ of charge $Z'e$ is described by the ``Rutherford" differential cross section
\begin{equation}
\label{eq:rutherford}
\frac{d \sigma_R (E', \beta)}{dE'} = \frac{2 \pi  \alpha^2 Z Z'}{m_t \beta^2} \frac{1}{E'^2} \l1- \frac{\beta^2 E'}{E_{max}}\r, 
 \end{equation}
where we have assumed a sufficiently fast incident particle so that interactions are governed by single collisions with energy transfer $E'$ \cite{Rossi}. $E_{max}$ denotes the maximum energy transfer possible satisfying kinematic constraints (target at rest and zero relative angle between incoming incident and outgoing target momenta):
\begin{equation}
E_{max} = \frac{2 m_t \beta^2 \gamma^2}{1+ 2\gamma(m_t/m_i) +(m_t/m_i)^2}. 
\end{equation}
\textcolor{blue}{Of course, the typical temperature of the white dwarf interior is $\sim 10^{7} ~\text{K} \sim \text{keV}$, so no electron is actually at rest. In fact, the fastest electrons have momentum of order the Fermi momentum $p_F \sim E_F \sim n^{1/3} \approx \text{MeV}$ (the approximation of an extreme relativistic Fermi gas is valid since $m_e \lesssim p_F$). Numerically, we find that $E_{max} \sim 10 ~\text{MeV}$ when the pion kinetic energy is $\sim 500 ~\text{MeV}$.} For sufficiently heavy incident particles, the differential cross section depends only on the velocity of the incident particle. Note that higher-spin particles receive additional corrections to the cross section, but for small energy transfers these corrections are negligible.

It is straightforward to understand the parametric dependences of \eqref{eq:rutherford}: there is increased likelihood to scatter for slowly moving incident particles undergoing ``soft-scatters" against lighter targets. Therefore, one would expect that soft scattering dominates the energy loss and that collisions with nuclei are suppressed by a factor $\OO\l\frac{m_C}{Z m_e}\r$ as compared to collisions with electrons. This is certainly true for incident charged particles in ordinary matter. \textcolor{blue}{For instance, the dominant source of stopping power for $\sim \text{GeV}$ heavy charged particles such as pions in matter is scattering (ionization) off bound electrons. Although there are additional losses due to radiative effects, these only become comparable at significantly higher energies $\OO(1000~\text{GeV})$}. However, both of these naive expectations turn out to be false when considering scattering off a degenerate electron gas. 

To understand the effect of degeneracy, we first consider the energy loss from scattering off non-degenerate free electrons. In general, the stopping power due to collisions with a number density $n$ is given by:
\begin{equation}
\label{eq:SP}
\frac{dE}{dx} = - \int dE' \left(\frac{d \sigma_R}{dE'}\right) n E'.
\end{equation}
\textcolor{blue}{For heavy charged particles the resulting stopping power is of the form $\frac{dE}{dx} \propto \frac{1}{\beta^2} \log{\beta^2}$, setting the upper limit of integration to be equal to $E_{max}$. Actually, this integration must be performed over all $E'$ within the (classical) regime of validity for \eqref{eq:rutherford}. However, it is important to note that the resultant stopping power only depends logarithmically on these bounds.} The range of a particle is defined as the distance travelled along its own trajectory before effectively coming to rest. In terms of the stopping power, this is simply
\begin{equation}
L = \int dE \left(\frac{dE}{dx}\right)^{-1},
\end{equation}
integrated over the incident particle kinetic energy. For a charged pion of initial kinetic energy $\sim 500 ~\text{MeV}$ scattering of an electron density $n \sim 10^{33} ~\text{cm}^{-3}$, we find that $L \sim 10^{-8} ~\text{cm}$. 

Of course, the electrons in a white dwarf are famously degenerate with Fermi energy $E_F \sim \OO(\text{MeV})$. Therefore, for a given energy transfer $E'$ an incident particle can only scatter those electrons within $E'$ of the Fermi surface. We define a modified density of electrons $n(E')$ as:
\begin{equation}
n(E') = \left\{
        \begin{array}{ll}
            \displaystyle \int \limits_{E_F -E'}^{E_F}dE ~g(E) & \quad E' \leq E_F \\
            n & \quad E_F \leq E'
        \end{array}
    \right.,
\end{equation}
where $g(E)$ is the density of states per unit volume for a three-dimensional free electron gas. This can also be expressed as a suppression of the differential cross section of order $\mathcal{O}(E'/E_F)$ whenever energy less than $E_F$ is transferred. Therefore, unlike in the non-degenerate case, the energy loss due to soft-scatters are in fact subdominant to the contributions from rare, hard-scatters. \textcolor{blue}{In addition, the resulting stopping power is no longer logarithmically sensitive to the minimum and maximum energy deposits allowed. Therefore, it becomes important to consider values for the energy transfer at which point \eqref{eq:rutherford} breaks down. According to the derivation in \cite{Rossi}, the energy transfer to a target electron is given in terms of an impact parameter $b$ (distance of closest approach): $E' = \frac{2 m_e Z r_e^2}{\beta^2 b^2}$. However, quantum mechanical uncertainty sets a limit to the accuracy that can be achieved in ``aiming" an incident particle at a target electron. This translates to a bound $b > \frac{1}{\beta \gamma m_e}$ or, in terms of energy transfer, $E' < E_q = 2 m_e \gamma^2 \alpha^2$. This quantum correction, which becomes important as $E'$ approaches $E_q$, causes $E'$ to increase with deceasing $b$ less rapidly than would otherwise be expected from $\eqref{eq:rutherford}$. Qualitatively, this has the effect of curtailing the maximum energy at which an incident particle is able to scatter. Of course, this upper bound has virtually no effect in the non-degenerate case for which the energy loss was instead dominated by soft-scattering.}

We find that for degenerate electron scattering, a charged pion of initial kinetic energy $\sim 500 ~\text{MeV}$ scattering of electron density $n \sim 10^{33} ~\text{cm}^{-3}$ has a range of $L \sim 10^{-2} ~\text{cm}$. However, the corresponding range for scattering off carbon nuclei of density $n_C \sim 10^{32} ~\text{cm}^{-3}$ is  $L \sim 10^{-4} ~\text{cm}$. Therefore, the dominant source of electromagnetic energy loss for the KKST process in the white dwarf is Coulomb scattering off nuclei. \textcolor{blue}{In addition, heavy charged particles may also lose energy via radiation. However, the stopping power of pions due to bremsstrahlung off nuclei only becomes comparable to that of Coulomb scattering off nuclei at energies $\OO(10 ~\text{GeV})$, ignoring any further suppression of radiative effects that occurs due to high density effects (LPM suppresion).} \textcolor{red}{Is it necessary to show the derivation of this ``critical energy"?}


The cross section for pion-nuclear interactions is approximately set by the nuclear length scale $\sigma_{nuclear} \sim \text{fm}^2$. In fact, numerous experiments have been conduced at meson facilities studying the effects of pions of kinetic energy in the range $50 - 500 ~\text{MeV}$ incident upon complex nuclei such as carbon. 








\end{document}


