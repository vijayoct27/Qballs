Any DM state that produces SM particles in a WD has the potential to ignite the star, provided a sufficient SM energy is produced. 
The distribution in space, momentum, and species of these SM products is dependent on unknown DM physics and is needed to determine the rate of DM-induced SN. 
This can of course be done precisely for a specific DM model, as we do for Q-ball DM in Section~\ref{sec:qballs}.
In this Section, however, we study some general features of DM-WD encounters involving DM that possesses generic interactions with itself and the SM. 
We collect below the basic formulas relating DM model parameters to ignition criteria, SN rate, \etc. 

DM can generically produce SM particles in the WD through three basic processes: DM-DM collisions, DM decays, and DM-SM scattering.
Note that we do not consider elastic scattering as a possible heating mechanism, which is unable to deliver enough energy for ignition.  
For ultra-heavy DM, these processes can be complicated events involving many (possibly dark) final states, analogous to the interactions of heavy nuclei.
We classify DM candidates into three types according to the interaction that provides the dominant source of heating, and refer to these as collision, decay, and scattering candidates.
We additionally make the simplifying assumption that the above interactions are ``point-like", producing SM products in a localized region (smaller than $\Lambda_T$) near the interaction vertex.

The induced-SN rate may be greatly enhanced if DM is captured in the star, and so we will also consider separately "transiting DM" and "captured DM". 
In general there is some loss of DM kinetic energy due to DM-SM scatters (which may be negligible for ignition heating). 
In the transit case, this energy loss is negligible and the DM simply passes through the star.
In the capture scenario, the energy loss is not directly capable of ignition but is sufficient to stop the DM and cause it to accumulate inside the star.
Of course, due to the velocity spread of DM in the rest frame of a WD, there may be both transiting and captured DM in the star. 
We will simply focus on whichever population provides the dominant ignition probability for a given DM model.
Energy loss may be due to a variety of processes, but for simplicity below we focus on elastic scattering of DM with carbon ions. 

\subsection{DM-SM Inelastic Scattering}

\paragraph{Ignition Condition.}
Runaway fusion only occurs in the degenerate WD interior where thermal expansion is suppressed as a cooling mechanism.
The outer layers of the WD, however, are composed of a non-degenerate gas and it is therefore essential that a DM candidate penetrate this layer in order to ignite a SN.
We parameterize this by a DM stopping power $(dE/dx)_\text{SP}$, the kinetic energy lost by the DM per distance traveled in the non-degenerate layer, and demand that
\begin{align}
\label{eq:CrustCondition}
  \left( \frac{d E}{d x} \right)_\text{SP} \ll
  \frac{m_\chi v^2_\text{esc}}{R_\text{env}},
\end{align}
where $R_\text{env} \sim 50 ~\text{km}$ is the width of a WD envelope \cite{KippenhahnWeigert}.

The energy deposited during a continuous heating event is best described in terms of a linear energy transfer $(dE/dx)_\text{LET}$, the kinetic energy of SM particles produced per distance traveled by the DM.
If these products have a heating length $L_0$ then the relevant energy deposit must at minimum be taken as the energy transferred over the transit distance $L_0$.
Of course, we can always choose to consider energy deposits over a longer segment of the DM trajectory.
Importantly, as per the general condition \eqref{eq:energy_boom_condition} such a deposition is \emph{less} explosive unless $L_0$ is smaller than the trigger size $\lambda_T$.
Thus, we consider the energy deposited in a transit over the larger of these two length scales.
Assuming the energy of the DM is roughly constant over this heating event, the ignition condition for transit heating is:
\begin{align}
\label{eq:transitexplosion}
  \left( \frac{d E}{d x} \right)_\text{LET} \gtrsim
  \frac{\Eboom}{\lambda_T} \cdot \text{max}
  \left\{\frac{L_0}{\lambda_T}, 1 \right\}^2.
\end{align}
Note that the DM stopping power in the non-degenerate layer $(dE/dx)_\text{SP}$ and the linear energy transfer in the degenerate interior $(dE/dx)_\text{LET}$ are possibly controlled by different physics and may have very different numerical values.
In addition, a transit heating event satisfying condition \eqref{eq:CrustCondition} will have negligible energy loss over the parametrically smaller trigger size or heating length $L_0$, validating \eqref{eq:transitexplosion}.

The above argument sums the individual energy deposits along the DM trajectory as though they are all deposited simultaneously.
This is valid if the DM moves sufficiently quickly so that this energy does not diffuse out of the region of interest before the DM has traversed the region.
We therefore require that the diffusion time $\tau_\text{diff}$ across a heated region of size $L$ at temperature $T_f$ be larger than the DM crossing-time:
\begin{align}
  \tau_\text{diff} \sim \frac{L^2}{\alpha(T_f)} \gg
  \frac{L}{v_\text{esc}},
\label{eq:SlowDiffusion}
\end{align}
where $\alpha(T)$ is the temperature-dependent diffusivity, and the DM transits at the stellar escape velocity $v_\text{esc} \sim 10^{-2}$.
This condition is more stringent for smaller regions, so we focus on the smallest region of interest, $L = \lambda_T$.
\eqref{eq:SlowDiffusion} is then equivalent to demanding that the escape speed is greater than the conductive speed of the fusion wave front, $v_\text{cond} \sim \alpha(T_f) / \lambda_T$.
Numerical calculations of $v_\text{cond}$ are tabulated in \cite{Woosley}, and indeed condition \eqref{eq:SlowDiffusion} is satisfied for all WD densities.

\paragraph{Event Rate: Transit Scenario.}
The rate of transit events is directly given by the flux of DM passing through a WD
\begin{align}
  \Gamma_\text{trans} \sim
  \frac{\rho_{\chi}}{m_\chi} R_\text{WD}^2
  \l\frac{v_\text{esc}}{v_\text{halo}}\r^2 v_\text{halo},
\label{eq:TransitFluxCondition}
\end{align}
where $\rho_\chi$ is the DM density in the region of the WD, and $R_\text{WD}$ is the WD radius.
Here $v_\text{halo} \sim 10^{-3}$ is the virial velocity of our galactic halo, and the transit rate contains an $(v_\text{esc}/v_\text{halo})^2 \sim 100$ enhancement due to gravitational focusing.

We will not consider here captured DM heating via scattering events, as such heating will typically cause ignition before capture occurs.  
However, it is possible to cause ignition after capture if the capture and thermalization of DM leads to an enhanced scattering process. 

\subsection{Collisions and Decays}

\paragraph{Ignition Condition.}
For a point-like DM-DM collision or DM decay event releasing particles of heating length $L_0$, ignition will occur if the total energy in SM products satisfies condition~\eqref{eq:energy_boom_condition}.
Such an event will likely result in both SM and dark sector products, so we parameterize the resulting energy in SM particles as a fraction $f_\text{SM}$ of the DM mass.
For non-relativistic DM, the DM mass is the dominant source of energy and therefore $f_\text{SM} \lesssim 1$ regardless of the interaction details.
With this parameterization, a single DM-DM collision or DM decay has an ignition condition:
\begin{equation}
\label{eq:coldecay}
  m_\chi f_\text{SM}  \gtrsim \Eboom \cdot \text{max} \left \{\frac{L_0}{\lambda_T}, 1 \right \}^3.
\end{equation}
We are thus sensitive to DM masses $m_\chi > 10^{15} ~\GeV$.

\paragraph{Event Rate: DM Wind.}
DM that is not captured traverses the WD in a time $R_\text{WD}/v_\text{esc}$, and the rate of DM-DM collisions within the WD parameterized by cross-section $\sigma_{\chi \chi}$ is:
\begin{align}
  \Gamma_\text{ann}
  \sim \l \frac{\rho_\chi}{m_\chi} \r^2 \sigma_{\chi \chi} \l \frac{v_\text{esc}}{v_\text{halo}}\r^3 v_\text{halo} R_\text{WD}^3. 
  \label{eq:collisionDM}
\end{align}
Similarly the net DM decay rate inside the WD parameterized by a lifetime $\tau_\chi$ is:
\begin{align}
 \Gamma_\text{decay}
   \sim \frac{1}{\tau_\chi} \frac{\rho_{\chi}}{m_\chi} \l \frac{v_\text{esc}}{v_\text{halo}}\r R_\text{WD}^3.
  \label{eq:decayDM}
\end{align}

\paragraph{Event Rate: DM Capture.}
The physics of DM capture can be made more precise for a specific DM-SM interaction.
Consider a spin-independent, elastic scattering off ions with cross section $\sigma_{\chi A}$. 
The rate of DM capture in gravitating bodies via elastic scatters is of course very well-studied \cite{Press:1985ug, Gould:1987ir}. 
However, this rate must be modified when the DM requires multiple scatters to lose the necessary energy for capture. 
See Appendix \ref{sec:capture} for details. 
Ultimately in the scenario of interest, the capture rate is parametrically of the form
\begin{equation}
\Gamma_\text{cap} \sim \Gamma_\text{trans} \cdot \text{min}\left \{1, \bar{N}_\text{scat} \frac{m_\text{ion} v_\text{esc}^2}{m_\chi v_\text{halo}^2}  \right \}, 
\end{equation}
where $\bar{N}_\text{scat}$ is the average number of DM scatters during a transit of the WD:
\begin{equation}
\bar{N}_\text{scat} \sim n_\text{ion} \sigma_{\chi A} R_\text{WD}.
\end{equation}
Note that when the DM is at escape velocity the typical momentum transfer is $q \sim \mu_{\chi A} v_\text{esc} \sim 200 ~\MeV$, which corresponds to an energy transfer $q^2/m_\text{ion} \sim 10 ~\MeV$. 
For momentum transfers less than or of order the inverse nuclear size, DM scattering off nuclei will be coherently enhanced. 
The average per-nucleon cross section (spin-independent) is given by
\begin{equation}
\sigma_{\chi A} = A^2 \l \frac{\mu_{\chi A}}{\mu_{\chi n}}\r^2 F^2(q) \sigma_{\chi n} \sim A^4 F^2(q) \sigma_{\chi n},
\end{equation}
where $F^2(q)$ is the Helm form factor \cite{Helm:1956zz}.
We find $F^2(q) \approx 0.1$ when the momentum transfer is of order $q \sim 200 ~\MeV$.  
Currently, the most stringent bound on DM spin-independent nuclear cross sections is from XENON 1T \cite{Aprile:2017iyp}:
\begin{equation}
\label{eq:xenon}
\sigma_{\chi n} < 5 \times 10^{-46} ~\text{cm}^2 \l \frac{m_\chi}{10^3 ~\GeV} \r.
\end{equation}
Thus, any DM candidate whose scattering cross section satisfies the direct detection constraint \eqref{eq:xenon} is necessarily capturing less than $1 \%$ of those DM which transit the star.
Therefore, the capture rate in this regime scales as $\Gamma_\text{cap} \propto \frac{\sigma_{\chi A}}{m_\chi^2}$. 

First we review the evolution of captured DM in the WD ignoring annihilations or decays. 
For the remainder of this section all numerical quantities are evaluated at a WD density $n_\text{ion} \sim 10^{31} ~\cm^{-3}$, for which the relevant WD parameters are \cite{cococubed}: 
\begin{equation}
M_\text{WD} \sim 1.25 ~M_{\astrosun}, ~~~~ R_\text{WD} \sim 4000 ~\text{km}, ~~~~ v_\text{esc} \sim 2 \times 10^{-2}. 
\end{equation}
We also assume a typical WD temperature $T \sim \text{keV}$.
Once DM is captured, it eventually thermalizes to an average velocity
\begin{equation}
v_\text{th} \sim \sqrt{\frac{T}{m_\chi}} \sim 10^{-12} \l \frac{m_\chi}{10^{16} ~\GeV}\r^{-1/2}
\end{equation}
and settles at the thermal radius
\begin{align}
R_\text{th} \sim \l \frac{T}{G m_\chi \rho_\text{WD}}\r^{1/2} \sim 0.1 ~\cm \l \frac{m_\chi}{10^{16} ~\GeV}\r^{-1/2},
\end{align}
where its kinetic energy balances against the gravitational potential energy of the (enclosed) WD mass. 
This thermalization time can be explicitly calculated if the DM predominantly loses energy via elastic nuclear scatters, as in \cite{Kouvaris:2010jy}. 
First, the DM passes through the WD many times until the size of its orbit becomes fully contained within the star.
This occurs after a time
\begin{equation}
t_1 \sim \l \frac{m_\chi}{m_\text{ion}} \r^{3/2} \frac{R_\text{WD}}{v_\text{esc}} \frac{1}{\bar{N}_\text{scat}} \frac{1}{\text{max}\{\bar{N}_\text{scat}, 1\}^{1/2}} \sim 2 \times 10^{15}~\text{s} \l \frac{m_\chi}{10^{16} ~\GeV} \r^{3/2} \l \frac{\sigma_{\chi A}}{10^{-35} ~\cm^2} \r^{-3/2}. 
\end{equation}
This stage is relevant only if the energy loss after a single transit does not exceed $\sim m_\chi v_\text{esc}^2$
\begin{equation}
\l \frac{m_\text{ion}}{m_\chi} \r \text{max}\{\bar{N}_\text{scat},1\} < 1,
\end{equation}
which is the case for any cross sections which satisfy \eqref{eq:xenon}. 
Subsequently, the DM completes many orbits within the star until dissipation reduces the orbital size to the thermal radius.
This occurs after a time
\begin{equation}
t_2 \sim \l \frac{m_\chi}{m_\text{ion}} \r \frac{1}{n_\text{ion} \sigma_{\chi A}} \frac{1}{v_\text{ion}} \left \{ 1+\log \l \frac{m_\chi}{m_\text{ion}}\r \right \} \sim 3 \times 10^{13}~\text{s}\l \frac{m_\chi}{10^{16} ~\GeV} \r \l \frac{\sigma_{\chi A}}{10^{-35} ~\cm^2} \r^{-1}. 
\end{equation}
where $v_\text{ion} \sim \sqrt{\frac{T}{m_\text{ion}}}$.
Note that the time to complete a single orbit is simply the gravitational free-fall timescale:
\begin{equation}
\label{eq:freefalltime}
t_\text{ff} \sim \sqrt{\frac{1}{G \rho_\text{WD}}} \sim 0.1 ~\text{s}.
\end{equation}
The DM will begin steadily accumulating at $R_\text{th}$ after a time $t_1 + t_2$.
This thermalization time is dominated by $t_1$ at low scattering cross-sections and is less than the age of the WD for:
\begin{equation}
\label{eq:settle}
\sigma_{\chi A} \gtrsim 10^{-36} ~\cm^2 \l \frac{m_\chi}{10^{16} ~\GeV} \r.
\end{equation}
However, if the collected mass of DM at the thermal radius ever exceeds the WD mass within this volume, then there is the possibility of self-gravitational collapse of the DM.
The time to collect a critical number of DM particles needed for self-gravitation is given by
\begin{align}
\label{eq:Ncore}
    t_\text{sg} \sim \frac{\rho_\text{WD} R^3_\text{th}}{m_\chi \Gamma_\text{cap}} \sim 10^{9} ~\text{s} \l \frac{m_\chi}{10^{16} ~\GeV} \r^{-1/2} \l \frac{\sigma_{\chi A}}{10^{35} ~\cm^2} \r^{-1}. 
\end{align}
For $m_\chi > 10^{15} ~\GeV$ and any cross-sections satisfying \eqref{eq:xenon} and \eqref{eq:settle}, $t_\text{sg}$ is by far the shortest timescale, i.e. self-gravitation happens instantly relative to thermalization.   

We now turn towards the rate of DM-DM collisions at various stages of the above evolution. 
Of course, the settling DM constitutes a number density throughout the WD volume.
Assuming \eqref{eq:settle} holds true, the total rate of annihilations for the in-falling DM is dominated near the thermal radius:
\begin{equation}
\label{eq:infall}
\Gamma_\text{ann} \sim \frac{\Gamma_\text{cap}^2 m_\chi^2 G^2}{\sigma_{\chi A}^2 v_\text{ion}^2 R_\text{th}^2} t_\text{ff}^3. 
\end{equation}
A constraint can also be derived from DM-DM collisions at the thermal radius. 
As this population of DM only lasts for a time $t_\text{sg}$, the condition that a single collision does not occur before self-gravitation is:
\begin{equation}
\frac{(\Gamma_\text{cap} t_\text{sg})^2}{R_\text{th}^3} \sigma_{\chi \chi} v_\text{th} t_\text{sg} < 1.
\end{equation}
This is ultimately a weaker constraint than that derived from considering the annihilations of in-falling DM \eqref{eq:infall}.
However, by far the strongest constraint comes in considering the collapse of the self-gravitating DM sphere. 
The ability for collapsing DM to ignite a WD via annihilations into SM particles (or formation of a black hole) is especially interesting. 
Such a process can also release sufficient energy to trigger SN at lower DM masses, and is the focus of future work \cite{us}.
For our purposes, it is practically impossible for a collapse to proceed without a single collision occurring.
During the collapse, the rate of collisions depends on the radius of the sphere and the typical time spent at that radius.
The latter is determined by the time it takes DM to cool and shed gravitational potential energy.
For elastic nuclear scatters, this is given by:
\begin{equation}
t_\text{cool} \sim \frac{m_\chi}{\rho_\text{WD} \sigma_{\chi A} \text{max}\{v_\chi, v_\text{ion}\}}, ~~~~ v_\chi \sim \sqrt{\frac{G N_\text{sg} m_\chi}{r}}. 
\end{equation}
Of course, there may be some stabilizing pressure which prevents the DM from ever collapsing or annihilating below a certain radius.
This is to be expected for heavy, composite DM, although such a stable radius would depend on unknown physics. 
For simplicity, we will assume the DM does stabilize at such a radius $R_\text{sta}$.
Demanding that a single DM-DM collision does not occur while the DM core is of order $R_\text{sta}$ gives a condition:
\begin{equation}
\frac{N_\text{sg}^2}{R_\text{sta}^3} \sigma_{\chi \chi} v_\chi t_\text{cool} < 1, 
\end{equation}
taking the number of collapsing DM particles to be fixed at $N_\text{sg} \sim \Gamma_\text{cap} t_\text{sg}$.
Note that for such heavy DM, the initial evolution of the collapse is actually dominated by further collection at the thermal radius rather than collapse (by cooling).
Depending on the radius, this may reduce the relevant timescale for collapse. 
However, we have also ignored the parametric increase in the number of collapsing DM particles due to this over-collection. 
As such, we examine in detail the bounds on DM ignition of a WD from gravitational collapse in forthcoming work \cite{us}.

Lastly, we compute the rate of decays for captured DM.  
Of course, this rate is proportional to the number of DM particles in the WD available for decay at any given instance.  
In the wind scenario \eqref{eq:decayDM}, this number is of order $\sim (\Gamma_\text{trans} \times t_\text{ff})$.  
In the capture scenario, this number is instead determined by the thermalization time inside the WD:
\begin{equation}
\Gamma_\text{decay} \sim  \frac{1}{\tau_\chi} \Gamma_\text{cap} t_2.
\label{eq:decayDMcap}
\end{equation}
Here we have conservatively assumed that after a thermalization time, the DM quickly collapses and stabilizes to a state incapable of further decay.
If not, the maximum possible decay rate is given by replacing $t_2 \to \tau_\text{WD}$ in the captured DM decay rate \eqref{eq:decayDMcap}.