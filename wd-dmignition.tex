The unknown physics of DM sets the rate of SM particle production within the star as well as the initial distribution in space, momentum, and species of the products.
This information is needed to determine if a given DM encounter with a WD results in runaway fusion and with what frequency.
Of course, this can be done precisely for a specific DM model.
In this Section, we describe several general, illustrative classes of DM-WD encounters which demonstrate the explosiveness of ultra-heavy DM interactions.
We also calculate the typical rates at which these events take place in a WD.

\subsection{Classifying DM-WD Encounters}

DM can generically heat the WD medium through the three schematic interactions depicted in Figure \ref{fig:feynman}: DM-DM collisions, DM decays, and DM-SM scattering.
Note that for ultra-heavy DM these can be complicated events involving many (possibly dark) final states, analogous to the interactions of heavy nuclei.
We classify DM candidates into three types according to the interaction that provides the dominant source of heating, and refer to these as collision, decay, and transit candidates.
We additionally make simplifying assumptions about the spatial extent of these interactions.
For collisions and decays, we consider only ``point-like" heating events with all SM products produced in a localized region (smaller than the trigger size).
For transits, we consider only DM-SM scatters that result in a continuous release of particles along the DM trajectory.

We can also classify candidates according to the evolution of the DM itself inside the star.
Generally there will be some loss of DM kinetic energy due to DM-SM scatters---this is either incidental to the eventual heating of the star or represents the dominant heating mechanism.
We consider two simple, limiting cases depending on the magnitude of this energy loss relative to the DM kinetic energy: ``DM wind" and ``DM capture".
In the DM wind scenario, there is negligible energy loss and the DM simply passes through the star.
In the DM capture scenario, the energy loss due to DM-SM scatters is not capable of igniting runaway but is sufficient to stop the DM and cause it to accumulate inside the star.
We consider both scenarios for collision and decay candidates, for which the capture results in a significantly enhanced rate of events.
For simplicity we consider only the wind scenario for transit candidates, though an enhanced explosiveness from slowed DM continuously scattering off stellar constituents is certainly possible in some models.

\subsection{Transits}

\paragraph{Ignition Condition.}
Runaway fusion only occurs in the degenerate WD interior where thermal expansion is suppressed as a cooling mechanism.
The outer layers of the WD, however, are composed of a non-degenerate gas and it is therefore essential that a DM candidate penetrate this layer in order to ignite a SN.
We parameterize this by a DM stopping power $(dE/dx)_\text{SP}$, the kinetic energy lost by the DM per distance traveled in the non-degenerate layer, and demand that
\begin{align}
\label{eq:CrustCondition}
  \left( \frac{d E}{d x} \right)_\text{SP} \ll
  \frac{m_\chi v^2_\text{esc}}{R_\text{env}},
\end{align}
where $R_\text{env} \sim 50 ~\text{km}$ is the width of a WD envelope \cite{KippenhahnWeigert}.

The energy deposited during a continuous heating event such as a DM transit is best described in terms of a linear energy transfer $(dE/dx)_\text{LET}$, the kinetic energy of SM particles produced per distance traveled by the DM.
If these products have a heating length $L_0$ then the relevant energy deposit must at minimum be taken as the energy transferred over the transit distance $L_0$.
Of course, we can always choose to consider energy deposits over a longer segment of the DM trajectory.
Importantly, as per the general condition \eqref{eq:energy_boom_condition} such a deposition is \emph{less} explosive unless $L_0$ is smaller than the trigger size $\lambda_T$.
Thus, we consider the energy deposited in a transit over the larger of these two length scales.
Assuming the energy of the DM is roughly constant over this heating event, the ignition condition for transit heating is:
\begin{align}
\label{eq:transitexplosion}
  \left( \frac{d E}{d x} \right)_\text{LET} \gtrsim
  \frac{\Eboom}{\lambda_T} \cdot \text{max}
  \left\{\frac{L_0}{\lambda_T}, 1 \right\}^2.
\end{align}
Note that the DM stopping power in the non-degenerate layer $(dE/dx)_\text{SP}$ and the linear energy transfer in the degenerate interior $(dE/dx)_\text{LET}$ are possibly controlled by different physics and may have very different numerical values.
In addition, a transit heating event satisfying condition \eqref{eq:CrustCondition} will have negligible energy loss over the parametrically smaller trigger size or heating length $L_0$, validating \eqref{eq:transitexplosion}.

The above argument sums the individual energy deposits along the DM trajectory as though they are all deposited simultaneously.
This is possible if the DM moves sufficiently quickly so that this energy does not diffuse out of the region of interest before the DM has traversed the region.
We therefore require that the diffusion time $\tau_\text{diff}$ across a heated region of size $L$ at temperature $T_f$ be larger than the DM crossing-time:
\begin{align}
  \tau_\text{diff} \sim \frac{L^2}{\alpha(T_f)} \gg
  \frac{L}{v_\text{esc}},
\label{eq:SlowDiffusion}
\end{align}
where $\alpha(T)$ is the temperature-dependent diffusivity, and the DM transits at the stellar escape velocity $v_\text{esc} \sim 10^{-2}$.
This condition is more stringent for smaller regions, so we focus on the smallest region of interest, $L = \lambda_T$.
\eqref{eq:SlowDiffusion} is then equivalent to demanding that the escape speed is greater than the conductive speed of the fusion wave front, $v_\text{cond} \sim \alpha(T_f) / \lambda_T$.
Numerical calculations of $v_\text{cond}$ are tabulated in \cite{Woosley}, and indeed condition \eqref{eq:SlowDiffusion} is satisfied for all WD densities.

\paragraph{Event Rate: Wind Scenario.}
The rate of transit events is given by the flux of DM passing through a WD
\begin{align}
  \Gamma_\text{trans} \sim
  \frac{\rho_{\chi}}{m_\chi} R_\text{WD}^2
  \l\frac{v_\text{esc}}{v_\text{halo}}\r^2 v_\text{halo},
\label{eq:TransitFluxCondition}
\end{align}
where $\rho_\chi$ is the DM density in the region of the WD, and $R_\text{WD}$ is the WD radius.
Here $v_\text{halo} \sim 10^{-3}$ is the virial velocity of our galactic halo, and the transit rate contains an $(v_\text{esc}/v_\text{halo})^2 \sim 100$ enhancement due to gravitational focusing.

\subsection{Collisions and Decays}

\paragraph{Ignition Condition.}
For a point-like DM-DM collision or DM decay event releasing particles of heating length $L_0$, ignition will occur if the total energy in SM products satisfies condition~\eqref{eq:energy_boom_condition}.
Such an event will likely result in both SM and dark sector products, so we parameterize the resulting energy in SM particles as a fraction $f_\text{SM}$ of the DM mass.
For non-relativistic DM, the DM mass is the dominant source of energy and therefore $f_\text{SM} \lesssim 1$ regardless of the interaction details.
With this parameterization, a single DM-DM collision or DM decay has an ignition condition:
\begin{equation}
\label{eq:coldecay}
  m_\chi f_\text{SM}  \gtrsim \Eboom \cdot \text{max} \left \{\frac{L_0}{\lambda_T}, 1 \right \}^3.
\end{equation}
We are thus sensitive to DM masses $m_\chi \gtrsim 10^{16} ~\GeV$.

\paragraph{Event Rate: DM Wind.}
DM that is not captured traverses the WD in a time $R_\text{WD}/v_\text{esc}$, and the rate of DM-DM collisions within the WD parameterized by cross-section $\sigma_{\chi \chi}$ is:
\begin{align}
  \Gamma_\text{ann}
  \sim \l \frac{\rho_\chi}{m_\chi} \r^2 \sigma_{\chi \chi} \l \frac{v_\text{esc}}{v_\text{halo}}\r^3 v_\text{halo} R_\text{WD}^3. 
  \label{eq:collisionDM}
\end{align}
Similarly the net DM decay rate inside the WD parameterized by a lifetime $\tau_\chi$ is:
\begin{align}
 \Gamma_\text{decay}
   \sim \frac{1}{\tau_\chi} \frac{\rho_{\chi}}{m_\chi} \l \frac{v_\text{esc}}{v_\text{halo}}\r R_\text{WD}^3.
  \label{eq:decayDM}
\end{align}

\paragraph{Event Rate: DM Capture.}
For the remainder of this section all numerical quantities are evaluated at a WD density $n_\text{ion} \sim 10^{31} ~\cm^{-3}$, for which the relevant WD parameters are \cite{cococubed}: 
\begin{equation}
M_\text{WD} \sim 1.25 ~M_{\astrosun}, ~~~~ R_\text{WD} \sim 4000 ~\text{km}, ~~~~ v_\text{esc} \sim 2 \times 10^{-2}. 
\end{equation}
We also assume a typical WD temperature $T \sim \text{keV}$.

For the DM to ultimately be captured in a WD it must lose energy $\sim m_\chi v^2$, where $v$ is the DM velocity (in the rest frame of the WD) asymptotically far away.
Properly, this DM velocity is described by a boosted Maxwell distribution peaked at $v_\text{halo}$. 
The physics of DM capture can be made more precise for a specific interaction.
Consider a spin-independent, elastic scattering off ions with cross section $\sigma_{\chi A}$. 
The average number of DM scatters during a transit of the WD is approximately:
\begin{equation}
\bar{N}_\text{scat} \sim n_\text{ion} \sigma_{\chi A} R_\text{WD}.
\end{equation}
If $\bar{N}_\text{scat} < 1$, then $\bar{N}_\text{scat}$ represents the probability for a \emph{single} scatter to occur during the transit. 
When the DM is at escape velocity the typical momentum transfer is $q \sim \mu_{\chi A} v_\text{esc} \sim 200 ~\MeV$, which corresponds to an energy transfer $q^2/m_\text{ion} \sim 10 ~\MeV$. 
For momentum transfers less than or of order the inverse nuclear size, DM scattering off nuclei will be coherently enhanced. 
The average per-nucleon cross section (spin-independent) is given by
\begin{equation}
\sigma_{\chi A} = A^2 \l \frac{\mu_{\chi A}}{\mu_{\chi n}}\r^2 F^2(q) \sigma_{\chi n} \sim A^4 F^2(q) \sigma_{\chi n},
\end{equation}
where $F^2(q)$ is the Helm form factor \cite{Helm:1956zz}.
We find $F^2(q) \approx 0.1$ when the momentum transfer is of order $q \sim 200 ~\MeV$.  
Currently, the most stringent bound on DM spin-independent nuclear cross sections is from XENON 1T \cite{Aprile:2017iyp}:
\begin{equation}
\label{eq:xenon}
\sigma_{\chi n} < 5 \times 10^{-46} ~\text{cm}^2 \l \frac{m_\chi}{10^3 ~\GeV} \r.
\end{equation}

We now compute the capture rate of DM in a WD through elastic scatters. 
The capture of DM in gravitating bodies is of course very well-studied \cite{Press:1985ug, Gould:1987ir}. 
However, the rate must be modified when the DM requires multiple scatters to lose the necessary energy for capture.
In the scenario of interest, the capture rate is parametrically of the form:
\begin{equation}
\Gamma_\text{cap} \sim \Gamma_\text{trans} \cdot \text{min}\left \{1, \bar{N}_\text{scat} \frac{m_\text{ion} v_\text{esc}^2}{m_\chi v_\text{halo}^2}  \right \}. 
\end{equation}
See \cite{Bramante:2017xlb} for a more detailed numerical calculation of this rate.  
Note that any DM candidate whose scattering cross section satisfies the direct detection constraint \eqref{eq:xenon} is necessarily capturing less than $1 \%$ of those DM which transit the star.
Therefore, the capture rate in this regime scales as $\Gamma_\text{cap} \propto \frac{\sigma_{\chi A}}{m_\chi^2}$. 

First we review the evolution of captured DM in the WD ignoring annihilations or decays. 
Once DM is captured, it eventually thermalizes to an average velocity
\begin{equation}
v_\text{th} \sim \sqrt{\frac{T}{m_\chi}} \sim 10^{-12} \l \frac{m_\chi}{10^{16} ~\GeV}\r^{-1/2}
\end{equation}
and settles at the thermal radius
\begin{align}
R_\text{th} \sim \l \frac{T}{G m_\chi \rho_\text{WD}}\r^{1/2} \sim 0.1 ~\cm \l \frac{m_\chi}{10^{16} ~\GeV}\r^{-1/2},
\end{align}
where its kinetic energy balances against the gravitational potential energy of the (enclosed) WD mass. 
This thermalization time can be explicitly calculated if the DM predominantly loses energy via elastic nuclear scatters, as in \cite{Kouvaris:2010jy}. 
First, the DM passes through the WD many times until the size of its orbit becomes fully contained within the star.
This occurs after a time
\begin{equation}
t_1 \sim \l \frac{m_\chi}{m_\text{ion}} \r^{3/2} \frac{R_\text{WD}}{v_\text{esc}} \frac{1}{\bar{N}_\text{scat}} \frac{1}{\text{max}\{\bar{N}_\text{scat}, 1\}^{1/2}} \sim 7 \times 10^7~\text{s} \l \frac{m_\chi}{10^{16} ~\GeV} \r^{3/2} \l \frac{\sigma_{\chi A}}{10^{-30} ~\cm^2} \r^{-3/2}. 
\end{equation}
This stage is relevant only if the energy loss after a single transit does not exceed $\sim m_\chi v_\text{esc}^2$
\begin{equation}
\l \frac{m_\text{ion}}{m_\chi} \r \text{max}\{\bar{N}_\text{scat},1\} < 1,
\end{equation}
which is the case for any cross sections which satisfy \eqref{eq:xenon}. 
Subsequently, the DM completes many orbits within the star until dissipation reduces the orbital size to the thermal radius.
This occurs after a time
\begin{equation}
t_2 \sim \l \frac{m_\chi}{m_\text{ion}} \r \frac{1}{n_\text{ion} \sigma_{\chi A}} \frac{1}{v_\text{ion}} \left \{ 1+\log \l \frac{m_\chi}{m_\text{ion}}\r \right \} \sim 3 \times 10^8~\text{s}\l \frac{m_\chi}{10^{10} ~\GeV} \r \l \frac{\sigma_{\chi A}}{10^{-38} ~\cm^2} \r^{-1}. 
\end{equation}
where $v_\text{ion} \sim \sqrt{\frac{T}{m_\text{ion}}}$.
Note that the time to complete a single orbit is simply the gravitational free-fall timescale:
\begin{equation}
\label{eq:freefalltime}
t_\text{ff} \sim \sqrt{\frac{1}{G \rho_\text{WD}}} \sim 0.1 ~\text{s}.
\end{equation}
After a time $t_1 + t_2$, the DM will begin steadily accumulating at $R_\text{th}$ (assuming thermalization occurs within the age of the WD). 
However, if the collected mass of DM at the thermal radius ever exceeds the WD mass within this volume, then there is the possibility of self-gravitational collapse of the DM.
The time to collect a critical number of DM particles needed for self-gravitation is given by
\begin{align}
\label{eq:Ncore}
    t_\text{sg} \sim \frac{\rho_\text{WD} R^3_\text{th}}{m_\chi \Gamma_\text{cap}} \sim 10^{4} ~\text{s} \l \frac{m_\chi}{10^{16} ~\GeV} \r^{-1/2} \l \frac{\sigma_{\chi A}}{10^{-30} ~\cm^2} \r^{-1}. 
\end{align}

We now turn towards the rate of DM-DM collisions at various stages of the above evolution. 
Of course, the settling DM constitutes a number density throughout the WD volume.
The total rate of annihilations for the infalling DM is given by:
A  constraint can also be derived from DM-DM collisions at the thermal radius. 
As this population of DM only lasts for a time $t_\text{sg}$, the condition that a single collision does not occur before self-gravitation is:
\begin{equation}
\frac{(\Gamma_\text{cap} t_\text{sg})^2}{R_\text{th}^3} \sigma_{\chi \chi} v_\text{th} t_\text{sg} < 1. 
\end{equation}
Ultimately, the strongest constraint comes in considering the collapse of the self-gravitating DM sphere. 
The ability for collapsing DM to ignite a WD via annihilations into SM particles (or formation of a black hole) is especially interesting. 
Such a process may also release sufficient energy to trigger SN at lower DM masses, and is the focus of future work \cite{us}.
During the collapse, the rate of collisions depends on the radius of the sphere and the typical time spent at that radius.
The latter is determined by the time it takes DM to cool and shed gravitational potential energy.
For elastic nuclear scatters, this is given by:
\begin{equation}
t_\text{cool} \sim \frac{m_\chi}{\rho_\text{WD} \sigma_{\chi A} \text{max}\{v_\chi, v_\text{ion}\}}, ~~~~ v_\chi \sim \sqrt{\frac{G N_\text{sg} m_\chi}{r}}. 
\end{equation}
Of course, there may be some stabilizing pressure which prevents the DM from ever collapsing or annihilating below a certain radius.
This is to be expected for heavy, composite DM, although such a stable radius would depend on unknown physics. 
For simplicity, we will assume the DM does stabilize at such a radius $R_\text{sta}$.
Demanding that a single DM-DM collision does not occur when the DM core is of order $R_\text{sta}$ gives a condition:
\begin{equation}
\frac{N_\text{sg}^2}{R_\text{sta}^3}\sigma_{\chi \chi} v_\chi t_\text{cool} < 1, 
\end{equation}
where we have assumed the number of collapsing DM particles is fixed at $N_\text{sg} \sim \Gamma_\text{cap} t_\text{sg}$.
This is in fact a conservative estimate since for such heavy DM, the initial evolution of the collapse is dominated by further collection at the thermal radius rather than collapse (by cooling).
Again, we examine in detail the bounds on DM ignition of a WD from gravitational collapse in forthcoming work \cite{us}.  

Lastly, we compute the rate of decays for captured DM.  
Of course, this rate is proportional to the number of DM particles in the WD available for decay at any given instance.  
In the wind scenario \eqref{eq:decayDM}, this number is simply $\sim (\Gamma_\text{trans} \times t_\text{ff})$.  
In the capture scenario, this number is instead determined by the settling time inside the WD:
\begin{equation}
\Gamma_\text{decay} \sim  \frac{1}{\tau_\chi} \Gamma_\text{cap} t_2
\label{eq:decayDMcap}
\end{equation}
Comparing the two, we find the rate of captured DM decays in the WD is simply enhanced by a numerical factor $\sim \l\frac{v_\text{esc}^3}{v_\text{halo}^2 v_\text{ion}} \r \approx 10^5$.   