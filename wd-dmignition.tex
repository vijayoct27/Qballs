Any DM interaction that produces SM particles in a WD has the potential to ignite the star, provided that sufficient SM energy is produced.
The distribution in space, momentum, and species of these SM products is dependent on unknown DM physics and is needed to determine the rate of DM-induced ignition.
This can be done precisely for a specific DM model, as we do for Q-balls in Section~\ref{sec:qballs}.
In this Section, however, we study some general features of DM-WD encounters involving DM that possesses interactions with itself and the SM.
We collect below the basic formulas relating DM model parameters to ignition criteria, SN rate, etc.

DM can generically heat a WD through three basic processes: DM-SM scattering, DM-DM collisions, and DM decays.
For ultra-heavy DM, these processes can be complicated events involving many (possibly dark) final states, analogous to the interactions of heavy nuclei.
In the case of DM-SM scattering, we consider both elastic and inelastic DM scatters off WD constituents, e.g.~carbon ions.
We classify DM candidates into three types according to the interaction that provides the dominant source of heating, and refer to these as scattering, collision, and decay candidates.
We also make the simplifying assumption that the above events are ``point-like", producing SM products in a localized region (smaller than the heating length) near the interaction vertex.
Where this is not the case (as in our elastic scattering and Q-ball constraints, see Sections~\ref{sec:TransitConstraints} and~\ref{sec:qballs}), then the same formalism applies but with the event size added to the stopping length.

The SN rate may be greatly enhanced if DM is captured in the star, so we also consider separately ``transiting DM" and ``captured DM".
In general, there is some loss of DM kinetic energy in the WD.
In the transit scenario, this energy loss is negligible and the DM simply passes through the star.
In the capture scenario, the energy loss is not directly capable of ignition but is sufficient to stop the DM and cause it to accumulate in the star.
Energy loss may be due to a variety of processes, but for simplicity we will focus on an DM-nuclei elastic scattering.
Of course, due to the velocity spread of DM in the rest frame of a WD, there will necessarily be both transiting and captured DM populations in the star.

\subsection{DM Transit}

\paragraph{DM-SM Scattering.}
Runaway fusion only occurs in the degenerate WD interior where thermal expansion is suppressed as a cooling mechanism.
The outer layers of the WD, however, are composed of a non-degenerate gas and it is therefore essential that a DM candidate penetrate this layer in order to ignite a SN.
We parameterize this by a DM stopping power $(dE/dx)_\text{SP}$, the kinetic energy lost by the DM per distance traveled in the non-degenerate layer, and demand that
\begin{align}
\label{eq:CrustCondition}
  \left( \frac{d E}{d x} \right)_\text{SP} \ll
  \frac{m_\chi v^2_\text{esc}}{R_\text{env}}.
\end{align}

DM-SM scattering will result in a continuous energy deposit along the DM trajectory (if the interaction is rare enough for this not to be true, then the encounter is analogous to the case of DM decay).
This is best described by a linear energy transfer $(dE/dx)_\text{LET}$, the kinetic energy of SM particles produced per distance traveled by the DM.
If these products have a heating length $L_0$ then the energy deposit must at minimum be taken as the energy transferred along a distance $L_0$ of the DM trajectory.
Importantly, as per the ignition condition~\eqref{eq:energy_boom_condition}, such a deposition is \emph{less} explosive unless $L_0$ is smaller than the trigger size $\lambda_T$.
We thus consider the energy deposited over the larger of these two length scales.
Assuming the energy of the DM is roughly constant during this heating event, the ignition condition is:
\begin{align}
\label{eq:transitexplosion}
  \left( \frac{d E}{d x} \right)_\text{LET} \gtrsim
  \frac{\Eboom}{\lambda_T} \cdot \text{max}
  \left\{\frac{L_0}{\lambda_T}, 1 \right\}^2.
\end{align}
Note that the DM stopping power $(dE/dx)_\text{SP}$ and the linear energy transfer $(dE/dx)_\text{LET}$ are related in the case of elastic scatters, but in general the two quantities may be controlled by different physics.
In addition, a transit event satisfying condition~\eqref{eq:CrustCondition} will have negligible energy loss over the parametrically smaller distances $\lambda_T$ or $L_0$, validating~\eqref{eq:transitexplosion}.

The above condition sums the individual energy deposits along the DM trajectory as though they are all deposited simultaneously.
This is valid if the DM moves sufficiently quickly so that this energy does not diffuse out of the region of interest before the DM has traversed the region.
We therefore require that the diffusion time $\tau_\text{diff} \sim 10^{-12}~\text{s}$ across a heated region of size $L$ at temperature $T_f$ be larger than the DM crossing-time:
\begin{align}
  \tau_\text{diff} \sim \frac{L^2}{\alpha(T_f)} \gg
  \frac{L}{v_\text{esc}},
\label{eq:SlowDiffusion}
\end{align}
where $\alpha(T)$ is the temperature-dependent diffusivity, and the DM transits at the stellar escape velocity $v_\text{esc} \sim 10^{-2}$.
This condition is more stringent for smaller regions, so we focus on the smallest region of interest, $L = \lambda_T$.
Then~\eqref{eq:SlowDiffusion} is equivalent to demanding that the escape speed is greater than the conductive speed of the fusion wave front, $v_\text{cond} \sim \alpha(T_f) / \lambda_T$.
Numerical calculations of $v_\text{cond}$ are tabulated in~\cite{Woosley}, and indeed condition~\eqref{eq:SlowDiffusion} is satisfied for all WD densities.

The rate of transit events is directly given by the flux of DM through a WD
\begin{align}
  \Gamma_\text{trans} \sim
  \frac{\rho_{\chi}}{m_\chi} R_\text{WD}^2
  \l\frac{v_\text{esc}}{v_\text{halo}}\r^2 v_\text{halo},
\label{eq:TransitRate}
\end{align}
where $\rho_\chi$ is the DM density in the region of the WD, and $R_\text{WD}$ is the WD radius.
Here $v_\text{halo} \sim 10^{-3}$ is the virial velocity of our galactic halo.
Note the $(v_\text{esc}/v_\text{halo})^2 \sim 100$ enhancement due to gravitational focusing.

We will not consider here captured DM that heats the star via scattering events, as such heating will typically cause ignition before capture occurs.
However, it is possible to cause ignition after capture if the collection of DM leads to an enhanced scattering process.

\paragraph{DM-DM Collisions and DM Decays.}

For a point-like DM-DM collision or DM decay event releasing particles of heating length $L_0$, ignition will occur if the total energy in SM products satisfies condition~\eqref{eq:energy_boom_condition}.
Such an event will likely result in both SM and dark sector products, so we parameterize the resulting energy in SM particles as a fraction $f_\text{SM}$ of the DM mass.
For non-relativistic DM, the DM mass is the dominant source of energy and therefore $f_\text{SM} \lesssim 1$ regardless of the interaction details.
A single DM-DM collision or DM decay has an ignition condition:
\begin{equation}
\label{eq:coldecay}
  m_\chi f_\text{SM} \gtrsim \Eboom \cdot \text{max} \left \{\frac{L_0}{\lambda_T}, 1 \right \}^3.
\end{equation}
Thus the WD is sensitive to annihilations/decays of DM masses $m_\chi > 10^{16} ~\GeV$.

DM that is not captured traverses the WD in a time $t_\text{ff} \sim R_\text{WD}/v_\text{esc}$, and the rate of DM-DM collisions within the WD parameterized by cross-section $\sigma_{\chi \chi}$ is:
\begin{align}
  \Gamma^\text{ann}_\text{SN}
  \sim \l \frac{\rho_\chi}{m_\chi} \r^2 \sigma_{\chi \chi} \l \frac{v_\text{esc}}{v_\text{halo}}\r^3 v_\text{halo} R_\text{WD}^3.
  \label{eq:collisionDM}
\end{align}
Similarly the net DM decay rate inside the WD parameterized by a lifetime $\tau_\chi$ is:
\begin{align}
 \Gamma^\text{decay}_\text{SN}
   \sim \frac{1}{\tau_\chi} \frac{\rho_{\chi}}{m_\chi} \l \frac{v_\text{esc}}{v_\text{halo}}\r R_\text{WD}^3.
  \label{eq:decayDM}
\end{align}

\subsection{DM Capture}

\paragraph{Review of DM Capture.}

We first summarize the capture and subsequent evolution of DM in the WD, ignoring annihilations or decays---see Appendix~\ref{sec:capture} for details.
Consider a spin-independent, elastic scattering off carbon ions with cross section $\sigma_{\chi A}$.
The rate of DM capture in gravitating bodies is of course very well-studied~\cite{Press:1985ug, Gould:1987ir}.
However, this rate must be modified when the DM requires multiple scatters to lose the necessary energy for capture.
Ultimately, for ultra-heavy DM the capture rate is of the form
\begin{align}
  \Gamma_\text{cap} &\sim \Gamma_\text{trans} \cdot
  \text{min}\left\{1, \overbar{N}_\text{scat} \frac{m_\text{ion} v_\text{esc}^2}{m_\chi v_\text{halo}^2} \right\},
\end{align}
where $\overbar{N}_\text{scat} \sim n_\x{ion} \sigma_{\chi A} R_\x{wd}$ is the average number of DM-carbon scatters during one DM transit. 
Once DM is captured, it eventually thermalizes with the stellar medium at velocity $v_\text{th} \sim \l T_\x{WD}/m_\chi \r^{1/2}$. 
The dynamics of this processes depends on the strength of the DM-carbon interaction, namely on whether energy loss to carbon ions provides a small perturbation to the DM's gravitational orbit within the star or whether DM primarily undergoes Brownian motion in the star due to collisions with carbon. 
For simplicity, we will focus here only on the former case, corresponding roughly to interactions
\begin{align}
\label{eq:slowcapture}
    \sigma_{\chi A} \lesssim \frac{m_\chi}{\rho_\x{WD} R_\x{WD}}
    \sim 10^{-26}~\x{cm}^2 \; \l \frac{10^{16} \GeV}{m_\chi} \r.
\end{align}
Note that the opposite regime indeed also provides constraints on captured DM and is unconstrained by other observations, see Figure~\ref{fig:elastic-capture}, however the resulting limits are similar to those presented here.
Note that all numerical quantities in this section are evaluated at a WD central density $n_\text{ion} \sim 10^{31}~\cm^{-3}$.

In the limit~\eqref{eq:slowcapture}, captured DM will thermalize by settling to a radius $R_\x{th}$ given by the balance of gravity and the thermal energy $T_\x{WD}$, 
\begin{align}
  R_\text{th} \approx 0.1 ~\cm \l \frac{m_\chi}{10^{16} ~\GeV}\r^{-1/2}.
\end{align}
This settling proceeds in two stages.
Captured DM will initially be found on a large, bound orbit that exceeds the size of the WD, decaying after many transits of the star until the orbital size is fully contained within the WD.
This occurs after a time
\begin{equation}
\label{eq:thermalization1}
t_1 \approx 7\times 10^{16}~\text{s}
  \l \frac{m_\chi}{10^{16} ~\GeV} \r^{3/2}
  \l \frac{\sigma_{\chi A}}{10^{-35} ~\cm^2} \r^{-3/2}.
\end{equation}
The DM then completes many orbits within the star until its orbital size decays to the thermal radius, occurring after a further time 
\begin{equation}
\label{eq:thermalization2}
t_2  \approx 10^{14}~\text{s}\l \frac{m_\chi}{10^{16} ~\GeV} \r
  \l \frac{\sigma_{\chi A}}{10^{-35} ~\cm^2} \r^{-1}.
\end{equation}
Note that the difference in scalings between $t_1$ and $t_2$ is due to the fact that, while the two times are ultimately determined by scattering in the star, the dynamics of the settling DM are quite distinct in each case.
$t_1$ is dominated by the time spent on the largest orbit outside the WD (which additionally depends on $\sigma_{\chi A}$) while $t_2$ is dominated by the time spent near the thermal radius.
Subsequently the DM will begin steadily accumulating at $R_\text{th}$, with the possibility of self-gravitational collapse if the collected mass of DM exceeds the WD mass within this volume.
This occurs after a time
\begin{align}
\label{eq:tsg}
t_\text{sg} &\approx
  10^{9} ~\text{s} \l \frac{m_\chi}{10^{16} ~\GeV} \r^{-1/2}
  \l \frac{\sigma_{\chi A}}{10^{-35} ~\cm^2} \r^{-1}.
\end{align}
Of course, not all of these stages may be reached within the age of the WD $\tau_\text{WD} \sim 5 ~\text{Gyr}$.
The full time to collect and begin self-gravitating is $t_1 + t_2 + t_\x{sg}$.

At any point during the above evolution, captured DM has the potential to trigger a SN. 
We will consider ignition via either the decay or annihilation of captured DM.  
Of particular interest are events occurring within a collapsing DM core, as such cores have the additional ability to ignite a WD for DM masses less than $\Eboom \sim 10^{16} ~\GeV$, either via multiple DM annihilations or by the formation of a black hole.
This is the focus of forthcoming work~\cite{us}.
In the following, we restrict attention to the limit~\eqref{eq:slowcapture} and require DM masses sufficiently large so that a single collision or decay is will ignite the star, and give only a quick assessment of DM core collapse.

\paragraph{Captured DM-DM Collisions.}
We now turn to the rate of DM-DM collisions for captured DM.
Of course, the thermalizing DM constitutes a number density of DM throughout the WD volume.
Assuming that $t_1 + t_2 < \tau_\text{WD}$, the total rate of annihilations for this ``in-falling" DM is peaked near the thermal radius and is of order:
\begin{equation}
\label{eq:infall}
\Gamma_\text{infall} \sim \frac{(\Gamma_\text{cap} t_2)^2}{R_\text{th}^3} \sigma_{\chi \chi} v_\text{th}.
\end{equation}
If $\Gamma_\text{infall} t_2 > 1$, then a SN will be triggered by the in-falling DM population.
Otherwise if $\Gamma_\text{infall} t_2 < 1$, the DM will start accumulating at the thermal radius.
If $t_\text{sg} \ll t_2$ (as expected for such heavy DM masses) there will be no collisions during this time and thus a collapse will proceed.
For a DM sphere consisting of $N$ particles at a radius $r$, the rate of annihilations is
\begin{align}
\label{eq:collapse}
\Gamma_\text{collapse} &\sim \frac{N^2}{r^3} \sigma_{\chi \chi} v_\chi, \\
 v_\chi &\sim \sqrt{\frac{G N m_\chi}{r}}.
\end{align}
In the regime of interest, it will generally be the case that the number of collapsing particles is greater than the self-gravitational threshold (this is discussed in Appendix \ref{sec:capture}). 
Of course, there may be some stabilizing physics which prevents the DM from collapsing and annihilating below a certain radius, such as formation of a black hole or bound states.
To illustrate the stringent nature of the collapse constraint we will simply assume some benchmark stable radius, as in Figure~\ref{fig:capture-collision}.
We assume that the timescale for collapse at this radius is set by DM cooling $t_\x{cool}$, which is related to $t_2$. 
Note that if a single collision has not occurred during collapse, one may additionally examine annihilations of the subsequent in-falling DM down to the stable radius---for simplicity, we do not consider this scenario.

\paragraph{Captured DM Decays.}
Lastly, we compute the rate of decays for captured DM, which is simply proportional to the number of DM particles in the WD available for decay at any given instance.
In the transit scenario~\eqref{eq:decayDM}, this rate is $\Gamma \sim \tau_\chi^{-1} \Gamma_\text{trans} t_\text{ff}$.
In the capture scenario, this number is instead determined by the thermalization time within the WD $\Gamma \sim \tau_\chi^{-1} \Gamma_\text{cap} t_2$. 
conservatively assuming that after a thermalization time, the DM quickly collapses and stabilizes to an ``inert" core incapable of further decay.
If this is not the case, then the captured DM decay rate is given by replacing $\Gamma \sim \tau_\chi^{-1} \Gamma_\text{cap} t_\text{WD}$.
