Having derived constraints on generic models of ultra-heavy DM, we turn towards a concrete example.
In various supersymmetric extensions of the SM, non-topological solitons called Q-balls can be produced in the early universe \cite{Coleman:1985ki, Kusenko:1997si}.
If these Q-balls were stable, they would comprise a component of the DM today.
For gauge-mediated models with flat scalar potentials, the Q-ball mass and radius are given by
\begin{equation}
\label{eq:Qballprop}
M_Q \sim m_S Q^{3/4}, ~~~ R_Q \sim m_S^{-1} Q^{1/4},
\end{equation}
where $m_S$ is related to the scale of supersymmetry breaking, and $Q$ is the global charge of the Q-ball---in our case, baryon number.
The condition $M_Q/Q < m_p$ ensures that the Q-ball is stable against decay to nucleons.
When an electrically neutral baryonic Q-ball interacts with a nucleon, it absorbs its baryonic charge and induces the dissociation of the nucleon into free quarks.
During this proton decay-like process, $\sim \text{GeV}$ of energy is released through the emission of 2--3 pions~\cite{Kusenko:1998}.
We assume that for each Q-ball collision, there is equal probability to produce $\pi^0$ and $\pi^\pm$ under the constraint of charge conservation.
Note that a sufficiently massive Q-ball will become a black hole if $R_Q \lesssim G M_Q$.
In the model described above, this translates into a condition $(M_\text{pl}/m_S)^4 \lesssim Q$.

We now determine the explosiveness of a Q-ball transit.
As in Section \ref{sec:Constraints}, this process is described by the parameter
\begin{equation}
\label{eq:QballLET}
\l\frac{dE}{dx}\r_\text{LET} \sim n_\text{ion} \sigma_Q N \epsilon,
\end{equation}
where the nuclear collision results in $N \sim 30$ pions released, each with kinetic energy $\epsilon \sim 500 ~\text{MeV}$.
These pions induce hadronic showers which terminate in low-energy hadrons that rapidly transfer their energy to ions via elastic scatters, as discussed in Section~\ref{sec:smheating}.
Thus the Q-ball transit has a heating length within the trigger size, and the Q-ball cross-section necessary to trigger runaway fusion is given by equations~\eqref{eq:transitexplosion} and~\eqref{eq:QballLET}:
\begin{equation}
 \sigma_Q \gtrsim \frac{1}{n_\text{ion}} \frac{\Eboom}{\lambda_T}
 \l \frac{1}{10~\GeV} \r.
\end{equation}
We see $\sigma_Q \approx 10^{-12} ~\text{cm}^2$ is sufficient to blow up a $\sim 1.25 ~M_{\astrosun}$ WD.
The cross-section for this interaction is approximately geometric
\begin{align}
\sigma_Q \sim \pi R_Q^2,
\end{align}
and so $Q \gtrsim 10^{42} ~(m_S/\text{TeV})^4$ can be adequately constrained from the observation of a single, heavy WD.
Note that the Q-ball interaction described above results in minimal slowing or transfer of kinetic energy for Q-balls this massive, so transits will easily penetrate the non-degenerate WD layer \eqref{eq:CrustCondition}.

The strongest previous constraints on Q-balls come from Super-Kamiokande as well as air fluorescence detectors of cosmic rays \cite{Dine:2003ax}.
However, the constraints due to white dwarfs are in a fundamentally complementary region of parameter space.
These are plotted in Figure~\ref{fig:Qballconstraint}.
As a comparison, the combined limits from Super-K and the OA, TA cosmic ray detectors are shown in red.
We have also included the constraints from a WD which result from considering gravitational heating during a Q-ball transit, as in \cite{Graham:2015apa}.