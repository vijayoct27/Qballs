\documentclass{article}

\usepackage[colorlinks=true]{hyperref}
\usepackage{amsmath}
\usepackage{amssymb}
\usepackage{accents}
\usepackage{parskip}
\usepackage{bm}
\usepackage{xcolor}
\usepackage{mathtools}
\usepackage{wasysym}

%%%%%%
\usepackage[framemethod=TikZ]{mdframed}
\mdfdefinestyle{MyFrame}{%
    linecolor=black,
    outerlinewidth=1pt,
    roundcorner=10pt,
    innertopmargin=\baselineskip,
    innerbottommargin=\baselineskip,
    skipabove=\baselineskip,
    skipbelow=\baselineskip,
    innerrightmargin=20pt,
    innerleftmargin=20pt,
    backgroundcolor=gray!20!white}
\usepackage{xpatch}
\makeatletter
\xpatchcmd{\endmdframed}
  {\aftergroup\endmdf@trivlist\color@endgroup}
  {\endmdf@trivlist\color@endgroup\@doendpe}
  {}{}
\makeatother
%%%%%%

\newcommand{\OO}{\mathcal{O}}
\newcommand{\LL}{\mathcal{L}}
\newcommand{\HH}{\mathcal{H}}
\newcommand{\NN}{\mathcal{N}}
\newcommand{\DD}{\mathcal{D}}
\newcommand{\hc}{\text{ h.c.}}

\newcommand{\angstrom}{\buildrel _{\circ} \over {\mathrm{A}}}
\newcommand{\GeV}{\text{GeV}}
\newcommand{\Hz}{\text{Hz}}
\newcommand{\eV}{\text{eV}}

\newcommand{\ket}[1]{\ensuremath{\left|#1\right>}}
\newcommand{\bra}[1]{\ensuremath{\left<#1\right|}}
\newcommand{\braket}[2]{\ensuremath{\left<#1|#2\right>}}


\title{Computing the Heating Length}
\author{Paul Riggins}
\begin{document}
\maketitle{}

Suppose a particle of species $i$ is flying through the white dwarf with initial energy $\epsilon_{i0}$ and velocity $v_{i0}$ and depositing energy in particles of species $j$ at a rate $dE_{ij}/dx$ (a function of $\epsilon_{i0}$) where $x$ is the distance along particle $i$'s trajectory, assuming its motion is approximately linear and does not include hard scatters that change its direction of flight significantly. Suppose this interaction has cross-section $\sigma_{ij}(\epsilon_i(x))$, where $\epsilon_i(x)$ is the energy of the $i$ particle, which will vary with $x$ as it loses energy. Suppose the $j$ particles have number density $n_j$ and the mean free path between $i-j$ scatters is given by $l_{ij}(x) = 1/n_j \sigma_{ij}$. Each $j$ particle that is struck will receive energy $\epsilon'$ and proceed on a flight of its own, depositing energy at a rate $dE_{jk}/dx$ into particles of species $k$. In this way energy will be transferred $i\to j\to k\to\cdots\to \text{C}$, where $\text{C}$ are the carbon atoms. We are interested in the distance $L_i(\epsilon)$ over which $\OO(1)$ of the initial energy $\epsilon_{i0}$ is spread amongst these carbon atoms, since ultimately it is these ions that must reach the critical temperature $\sim \text{MeV}$ in order to begin fusion.

For simplicity, will assume that every part of this cascade is collinear because of the high energies involved. This is also a conservative approach to the problem, since any hard scatters that would reverse the direction of motion would only serve to localize the heating more. We will also be assuming that each particle of the process ``survives,'' that is, it is not absorbed nor does it decay as a goes about depositing its energy. If it would be absorbed or decay, this could presumably be captured by an appropriate form of $dE/dx$.

\subsection{$i\to\text{C}$}

Consider first the simplest process were the initial particle deposits its energy immediately into carbon ions. This will give us a ``first-order'' heating length $L_i^{(1)}$, determined only by a single step. After a distance $x$, the $i$ particle has lost energy to ions
\begin{align}
\Delta \epsilon_{i\text{C}}^{(1)}(x) = \int_0^x dx'\, \frac{dE_{i\text{C}}}{dx'}(\epsilon_{i0})
\end{align}
where we note that the rate of energy loss is a function of the initial energy. This energy loss is some subset of the total energy lost in that distance: $\Delta \epsilon_{i\text{C}}^{(1)}(x) \leq \epsilon_{i}(x) - \epsilon_{i0}$. Letter us suppose that once the energy is in the ions it will thermalize efficiently among them. Then we can find a first-order heating length $L_i^{(1)}(\epsilon_{i0})$ by requiring that $x = L_i^{(1)}(\epsilon_{i0})$ sees
\begin{align}
\Delta \epsilon_{i\text{C}}^{(1)}(L_i^{(1)}) = \int_0^{L_i^{(1)}} dx'\, \frac{dE_{i\text{C}}}{dx'}(\epsilon_{i0}) \sim \epsilon_{i0}
\end{align}

\subsection{$i\to j\to\text{C}$}

Suppose now that the $i$ particle first strikes a $j$ particle which then subsequently strikes a $\text{C}$ ion. There may also be a contribution from direct $i \to\text{C}$ scattering, which we will simply take from the previous section.

The new relevant mean free paths are $l_{ij}(x) = 1/n_j\sigma_{ij}$ and $l_{j\text{C}}(x) = 1/n_\text{C}\sigma_{j\text{C}}$.  This energy is distributed such that a particle at $x$ receives energy
\begin{align}
\epsilon_{j0}(x) &= l_{ij}(x)\frac{dE_{ij}}{dx}(\epsilon_{i0}) \label{eqn:jInitialEnergy}\\
\epsilon_{\text{C}0}(y) &= l_{j\text{C}}(y)\frac{dE_{j\text{C}}}{dy}(\epsilon_{j0}(x))
\end{align}
where $y$ is the distance from the creation of the appropriate $j$ particle. The total number of $j$ of particles that have received energy after a distance $x$ is
\begin{align}
N_j = \int_0^x \left( \frac{dx}{v_i(x)} \right) \big( n_j\sigma_{ij}(\epsilon_i(x))v_i(x) \big) = \int_0^x \left( \frac{dx}{l_{ij}(x)} \right)
\end{align}
That is, $dx / l_{ij}(x)$ particles receive energy $\epsilon_{j0}(x)$ every distance $dx$. Each such particle kicked at $x$ proceeds to deposit its energy in ions over a distance $y$:
\begin{align}
\Delta \epsilon_{j\text{C}}(x,y) = \int_x^{x+y} dy\, \frac{dE_{j\text{C}}}{dy}(\epsilon_{j0}(x))
\end{align}
So the total energy deposited in the ions as a function of $x$ and $y$ in this two-step process is
\begin{align}
\Delta \epsilon_{ij\text{C}}^{(2)}(x,y) = \int_0^x \left( \frac{dx'}{l_{ij}(x')} \right) \int_{x'}^{x'+y(x')} dy'\, \frac{dE_{j\text{C}}}{dy'}(\epsilon_{j0}(x'))
\end{align}
As hinted in the limits here, however, the arguments $x$ and $y$ are not independent, but rather related by the velocities of the particles. After some finite time $t$, the initial $i$ particle will have traveled some distance $x$ and the subsequently created $j$ particles will have traveled some distance $y(x')$ corresponding to when each $j$ particle was kicked. Supposing that all particles are highly relativistic, we can see simply that all these $i$ and $j$ particles will be traveling together, rather than some leading or lagging others. (As soon as a particle is kicked, it proceeds at $c$ collinear with the kicking particle.) Thus it will always be that if the $i$ particle has traveled a distance $x$, each $j$ particle kicked at $x'$ will have traveled from $x'$ to $x' + y(x') = x$. Our two-step deposited energy then simplifies to
\begin{align}
\Delta \epsilon_{ij\text{C}}^{(2)}(x) &= \int_0^x \left( \frac{dx'}{l_{ij}(x')} \right) \int_{x'}^x dy'\, \frac{dE_{j\text{C}}}{dy'}(\epsilon_{j0}(x'))
\end{align}
where again,
\begin{align}
\epsilon_{j0}(x') = l_{ij}(x')\frac{dE_{ij}}{dx'}(\epsilon_{i0}) \tag{\ref{eqn:jInitialEnergy}}
\end{align}
This energy loss is again some subset of the total energy lost in that distance: $\Delta \epsilon_{i\text{C}}^{(2)}(x) \leq \epsilon_{i}(x) - \epsilon_{i0}$. Furthermore, it is separate from the one-step process considered in the previous subsection, so that
\begin{align}
\Delta \epsilon_{i\text{C}}^{(1)}(x) + \Delta \epsilon_{ij\text{C}}^{(2)}(x) \leq \epsilon_{i}(x) - \epsilon_{i0}
\end{align}
Again assuming that energy deposited in ions will thermalize efficiently, we can find a second-order heating length $L_i^{(2)}(\epsilon_{i0})$ by requiring that $x = L_i^{(2)}(\epsilon_{i0})$ sees
\begin{align}
\Delta \epsilon_{i\text{C}}^{(1)}\left( L_i^{(2)}(\epsilon_{i0}) \right) + \Delta \epsilon_{ij\text{C}}^{(2)}\left( L_i^{(2)}(\epsilon_{i0}) \right) \sim \epsilon_{i0}
\end{align}
which translates to
\begin{align}
\int_0^{L_i^{(2)}} dx'\, \frac{dE_{i\text{C}}}{dx'}(\epsilon_{i0}) + \int_0^{L_i^{(2)}} \left( \frac{dx'}{l_{ij}(x')} \right) \int_{x'}^{L_i^{(2)}} dy'\, \frac{dE_{j\text{C}}}{dy'}(\epsilon_{j0}(x')) \sim \epsilon_{i0}
\end{align}

\subsection{$i\to j\to k\to \text{C}$}

In the previous case, the only possibilities were $i\to\text{C}$ and $i\to j\to\text{C}$. When considering two possible non-ion white dwarf species $j$ and $k$, however, we begin to see a combinatorial growth of possibilities:
\begin{align}
i\to \text{C},  && i\to j\to\text{C},  && i\to k\to\text{C},  && i\to j\to k\to\text{C},  && i\to k\to j\to\text{C}
\end{align}
We have already computed the forms of the one- and two-step energy losses, and the three-step process should follow analogously:
\begin{align}
\Delta \epsilon_{ijkC}^{(3)}(x) = \int_0^x \left( \frac{dx'}{l_{ij}(x')} \right) \int_{x'}^x\, \left( \frac{dy'}{l_{jk}(y')} \right) \int_{y'}^x dz'\, \frac{dE_{j\text{C}}}{dz'}(\epsilon_{k0}(x',y'))
\end{align}
where
\begin{align}
\epsilon_{k0}(x',y') = l_{jk}(y')\frac{dE_{jk}}{dy'}\big(\epsilon_{j0}(x')\big)
\end{align}
and $e_{j0}(x')$ is given by \eqref{eqn:jInitialEnergy}. Note that in each of these formulas the last term in parentheses is an argument of $dE/dx$, not a multiplicative factor.

The third order heating length $L_i^{(3)}(\epsilon_{i0})$ can then be calculated by requiring that $x = L_i^{(3)}(\epsilon_{i0})$ sees
\begin{align}
 & \Delta \epsilon_{i\text{C}}^{(1)}\left( L_i^{(3)} \right) \\
  +\  & \Delta \epsilon_{ij\text{C}}^{(2)}\left( L_i^{(3)} \right) + \Delta \epsilon_{ik\text{C}}^{(2)}\left( L_i^{(3)} \right) \\
  +\  & \Delta \epsilon_{ijk\text{C}}^{(3)}\left( L_i^{(3)} \right) + \Delta \epsilon_{ikj\text{C}}^{(3)}\left( L_i^{(3)} \right) \sim \epsilon_{i0}
\end{align}









\end{document}
