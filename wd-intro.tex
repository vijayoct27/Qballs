Identifying the nature of dark matter (DM) remains one of the clearest paths beyond the Standard Model (SM) and it is thus fruitful to study the observable signatures of any yet-allowed DM candidate.
Many direct detection experiments are designed to search for DM, e.g.~\cite{Akerib:2016vxi, Agnese:2017njq}, yet these lose sensitivity to heavier DM due to its diminished number density.
Even for a strongly-interacting candidate, if the DM mass is above $\sim 10^{22}~\GeV$ a terrestrial detector of size $\sim (100~\text{m})^2$ will register fewer than one event per year.
While these masses are large compared to those of fundamental particles, it is reasonable to suppose that DM may exist as composite states just as the SM produces complex structures with mass much larger than fundamental scales (e.g., you, dear reader).
Currently there is a wide range of unexplored parameter space for DM candidates less than $\sim 10^{48}~\GeV$, above which the DM will have observable gravitational microlensing effects~\cite{Griest:2013aaa}.
For such ultra-heavy DM, indirect signatures in astrophysical systems are a natural way forward.
One such signal, proposed by~\cite{Graham:2015apa}, is that DM can trigger runaway fusion and ignite type Ia supernovae (SN) in sub-Chandrasekhar white dwarf (WD) stars.


In addition to constraining the properties of DM, this raises the intriguing possibility that DM-induced runaway fusion is responsible for a fraction of observed astrophysical transients. 
The progenitors of type Ia SN are not fully understood~\cite{Maoz:2012}, and recent observations of sub-Chandrasekhar~\cite{Scalzo:2014sap, Scalzo:2014wxa}, hostless~\cite{McGee:2010}, and unusual type Ias~\cite{Foley:2013} suggest that multiple progenitor systems and ignition mechanisms are operative.
Other suspected WD thermonuclear events, such as the Ca-rich transients~\cite{Kasliwal:2012}, are also poorly understood.
While mechanisms for these events have been proposed~\cite{Woosley1994,Fink:2007fv,Pakmor:2013wia,Sell:2015rfa}, the situation is yet unclear and it is worthwhile to consider new sources of thermonuclear ignition.

Runaway thermonuclear fusion requires both a heating event and the lack of significant cooling which might quench the process.
The WD medium is particularly suited to this as it is dominated by degeneracy pressure and undergoes minimal thermal expansion, which is the mechanism that regulates fusion in main sequence stars.
Thermal diffusion is the primary cooling process in a WD, and it can be thwarted by heating a large enough region.
The properties of a localized heating necessary to trigger runaway fusion were computed in~\cite{Woosley}.
Consequently, it was realized~\cite{Graham:2015apa} that if DM is capable of sufficiently heating a WD in this manner, it will result in a SN with sub-Chandrasekhar mass progenitor.
This was used to constrain primordial black holes which transit a WD and cause heating by dynamical friction, although the authors of~\cite{Graham:2015apa} identify several other heating mechanisms which may be similarly constrained.

In this paper, we examine DM candidates with non-gravitational interactions that cause heating through the production of SM particles.
An essential ingredient in this analysis is understanding the length scales over which SM particles deposit energy in a WD medium.
We find that most high energy particles thermalize rapidly, over distances shorter than or of order the critical size for fusion.
Particle production is thus an effective means of igniting WDs.
Constraints on these DM candidates come from either observing specific, long-lived WDs or by comparing the measured rate of type Ia SN with that expected due to DM.
It is important to note that these constraints are complementary to direct searches---it is more massive DM that is likely to trigger SN, but also more massive DM that has low terrestrial flux.
The WD detector excels in this regime due to its large surface area $\sim (10^4~\text{km})^2$, long lifetime $\sim \text{Gyr}$, and high density. 
We demonstrate these constraints for generic classes of DM models that produce SM particles via DM-SM scattering, DM-DM collisions, or DM decays, and consider the significantly enhanced constraints for DM that is captured in the star.
As a concrete example we consider ultra-heavy Q-ball DM as found in supersymmetric extensions of the SM.

The rest of the paper is organized as follows.
We begin in Section~\ref{sec:boomreview} by reviewing the mechanism of runaway fusion in a WD.
In Section~\ref{sec:smheating} we study the heating of a WD due to the production of high-energy SM particles.
Detailed calculations of the stopping of such particles are provided in Appendix~\ref{sec:wdpdg}.
In Section~\ref{sec:dmignition} we parameterize the explosiveness and event rate for generic classes of DM-WD encounters, and in Section~\ref{sec:constraints} we derive schematic constraints on such models.
The details of DM capture in a WD are reserved for Appendix~\ref{sec:capture}.
Finally we specialize to the case of Q-balls in Section~\ref{sec:qballs}, and conclude in Section~\ref{sec:discussion}.
