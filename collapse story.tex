%\documentclass[10pt, twocolumn]{article}
%\documentclass[11pt]{article}
%\documentclass[twocolumn,showpacs,preprintnumbers,amsmath,amssymb,prl, superscriptaddress]{revtex4}
%\documentclass[twocolumn, preprintnumbers,amsmath,amssymb,prd, superscriptaddress]{revtex4}
\documentclass[preprintnumbers,amsmath,amssymb,prd,superscriptaddress]{revtex4}
%\documentclass[10pt, preprint,showpacs,preprintnumbers,amsmath,amssymb, superscriptaddress]{revtex4}
%\documentclass[11pt, prd,preprintnumbers,amsmath,amssymb, superscriptaddress]{revtex4}
%\documentclass[11pt, prd,preprintnumbers, amsmath,amssymb, superscriptaddress, nofootinbib, hyperref]{revtex4}

\usepackage{latexsym}
\usepackage{amssymb}
\usepackage{epsfig,amsmath,graphics}
\usepackage{epstopdf}
\usepackage{verbatim}
\usepackage{wasysym}
\usepackage{hyperref}
\usepackage{feynmp-auto} % feynman diagrams
%\usepackage{subfig}
\usepackage[utf8]{inputenc}
\usepackage{xpatch}
\usepackage{xcolor}
\usepackage{mathtools}
\hypersetup{
    colorlinks,
    linkcolor={red!80!black},
    citecolor={green!60!black},
    urlcolor={blue!60!black}
}
\usepackage{appendix}

\newcommand{\Ez}{\mathcal{E}_0}
\newcommand{\Eboom}{\mathcal{E}_\text{boom}}
\newcommand{\OO}{\mathcal{O}}
\newcommand{\LL}{\mathcal{L}}
\newcommand{\HH}{\mathcal{H}}
\newcommand{\TeV}{\text{TeV}}
\newcommand{\GeV}{\text{GeV}}
\newcommand{\MeV}{\text{MeV}}
\newcommand{\keV}{\text{keV}}
\newcommand{\rad}{\text{rad}}
\newcommand{\cm}{\text{cm}}
\newcommand{\angstrom}{\buildrel _{\circ} \over {\mathrm{A}}}
\newcommand{\pslash}{p\hspace{-0.070in}/\,}
\newcommand{\Mpl}{M_{\text{pl}}}
\newcommand{\ket}[1]{\ensuremath{\left|#1\right>}}
\newcommand{\bra}[1]{\ensuremath{\left<#1\right|}}
\newcommand{\braket}[2]{\ensuremath{\left<#1|#2\right>}}
%Large Parentheses
\def\r{\right)}
\def\l{\left(}

\begin{document}

Multiple DM-DM collisions or decays in a sufficiently small region can occur rapidly enough to be counted as a single heating event.
This is similar in nature to a transit heating event, where multiple scatters across a transit length $\lambda_T$ can release an energy $\Eboom$ and satisfy \eqref{eq:transitexplosion} even if any individual scatter is not explosive by itself.   
If a single DM-DM collision is unable to ignite the star, the sum total of the energy released in many collisions can still result in a SN if
\begin{equation}
\label{eq:multcolboom}
 m_\chi \gtrsim \frac{\Eboom}{N_\text{mult}} \cdot \text{max} \left \{\frac{L_0}{\lambda_T}, 1 \right \}^3, ~~~~ N_\text{mult} \gtrsim 1,
\end{equation}
We define $N_\text{mult}$ as the number of collisions within a region of size $\text{max}\{\lambda_T,L_0\}^3$ (or smaller) during a diffusion time $\tau_\text{diff}$.
This necessarily depends on additional DM parameters and the evolution of the captured DM in the star. 
These are discussed in detail below. 

For the remainder of this section, all numerical quantities are evaluated assuming a WD lifetime $\tau_\text{WD} \sim 5 ~\text{Gyr}$ and central WD density $n_\text{ion} \sim 10^{31} ~\cm^{-3}$. 
At this density, the relevant WD parameters are approximately: 
\begin{equation}
M_\text{WD} \sim 1.25 ~M_{\astrosun}, ~~~~ R_\text{WD} \sim 4000 ~\text{km}, ~~~~ v_\text{esc} \sim 2 \times 10^{-2}. 
\end{equation}
We also assume a typical WD temperature $T \sim \text{keV}$.
D
M that is not captured traverses the WD in $R_\text{WD}/v_\text{esc} \sim 0.1 ~\text{s}$, and the rate of DM-DM collisions within the WD parameterized by cross-section $\sigma_{\chi \chi}$ is:
\begin{align}
  \Gamma_\text{ann}
  \sim \l \frac{\rho_\chi}{m_\chi} \r^2 \sigma_{\chi \chi} \l \frac{v_\text{esc}}{v_\text{halo}}\r^3 v_\text{halo} R_\text{WD}^3. 
  \label{eq:collisionDM}
\end{align}

For the DM to be captured in a WD, it must lose energy $\sim m_\chi v^2$, where $v$ is the relative DM velocity (in the rest frame of the WD) asymptotically far away.
Properly, this DM velocity is described by a (boosted) Maxwell distribution peaked at the galactic virial velocity $v_\text{halo} \sim 10^{-3}$. 
Since typically $v \ll v_\text{esc}$, the DM has initial velocity $v_\text{esc}$ in the star and must lose a fraction $(v/v_\text{esc})^2$ of its energy to become captured. 
The physics of DM capture can be made more precise for a specific interaction.
Consider a spin-independent, elastic scattering off ions with cross section $\sigma_{\chi A}$. 
Assuming $m_\text{ion} \ll m_\chi$, the typical momentum transfer in an elastic scatter is $q \sim \mu_{A} v_\text{esc} \sim 200 ~\MeV$, where $\mu_{A} \sim m_\text{ion}$ is the reduced mass of the DM-nuclei system. 
This corresponds to an energy transfer $q^2/m_\text{ion} \sim m_\text{ion} v_\text{esc}^2 \sim 10 ~\MeV$. 
The average number of DM scatters during a full transit of the WD is simply a ratio of the mean free path to the size of the WD
\begin{equation}
\bar{N}_\text{scat} \sim n_\text{ion} \sigma_{\chi A} R_\text{WD}.
\end{equation}
If $\bar{N}_\text{scat} < 1$, then $\bar{N}_\text{scat}$ is the probability for a \emph{single} scatter to occur during the transit. 
%Thus, DM with initial velocities less than
%\begin{equation}
%\label{eq:capture}
%v_\text{cap}^2 \sim v_\text{esc}^2 \l \frac{m_\text{ion}}{m_\chi} \r \text{max}\{\bar{N}_\text{scat} ,1\}.
%\end{equation}
%will be captured in the WD. 
The capture of DM in gravitating bodies is very well-studied \cite{Gould}. 
This calculation becomes slightly more involved if a single scatter is ever insufficient to capture the DM.
For incoming DM with asymptotic velocity $\sim v_\text{halo}$ (the peak of the Maxwell distribution), multiple scatters are required if
\begin{equation}
B \equiv \frac{m_\text{ion} v_\text{esc}^2}{m_\chi v_\text{halo}^2} \lesssim 1. 
\end{equation}
It can be shown that the general rate of capture is approximately of the form
\begin{equation}
\Gamma_\text{cap} \sim \Gamma_\text{trans} \cdot \text{min}\left \{1, \bar{N}_\text{scat} \text{min}\{B,1\}\right\}.
\end{equation}
%Evidently, the assumption that the scatters responsible for slowing the DM are not sufficient to blow up the WD \eqref{eq:transitexplosion} is a valid one for cross sections
%\begin{equation}
%\sigma_{\chi A} < \l \frac{\Eboom}{\lambda_T} \r \l \frac{1}{m_\text{ion} v_\text{esc}^2} \r \l \frac{1}{n_\text{ion}} \r \sim 10^{-8} ~\cm^2,
%\end{equation}
Since the momentum transfer $q \sim 100 ~\text{MeV}$ is of order the inverse nuclear size, it is reasonable to expect the DM coherently scatters off all nucleons in any nucleus. 
The average per-nucleon cross section (spin-independent) is given by
\begin{equation}
\sigma_{\chi A} = A^2 \l \frac{\mu_{A}}{\mu_{n}}\r^2 F^2(q) \sigma_{\chi n} \sim A^4 F^2(q) \sigma_{\chi n},
\end{equation}
where $F^2(q)$ is the Helm form factor \cite{LUX thesis}:
\begin{equation}
F^2(q) = \exp (-q^2 s^2) \left(\frac{3 j_1(q R_n)}{q R_n}\right)^2, ~~~~ s \sim \text{fm}, ~ R_n \sim \sqrt{(A^{1/3})^2 - 5s}. 
\end{equation}
For momentum transfers at a DM velocity $v_\text{esc}$, we find that $F^2(q) \sim 0.1$. 
In general, the form factor should range from $F^2 \sim 10^{-2} - 1$, where for sufficiently $q$ the DM is effectively scattering off individual nucleons. 
We will conservatively take $F^2 \sim 0.1$ for all momentum transfers, and this does not affect the results significantly. 
Currently, the bound on spin-independent DM nuclear elastic scatters from XENON 1T \cite{Xenon} is
\begin{equation}
\label{eq:xenon}
\sigma_{\chi n} < 10^{-45} ~\text{cm}^2 \l \frac{m_\chi}{10^3 ~\GeV} \r.
\end{equation}
Interestingly, any DM candidate whose scattering cross section satisfies the direct detection constraint is necessarily capturing only a fraction $10^{-2}$ or fewer of those DM which transit the star.
With such an interaction, the capture rate of DM is always less than the transit rate. 
Ultimately we will be interested in the self-gravitational collapse of captured DM, for which $m_\chi$ is must be sufficiently large that $B < 1$.
Crucially, in this regime the capture rate has the following parametric dependence on mass and scattering cross section:
\begin{equation}
\Gamma_\text{cap} \propto \frac{\sigma_{\chi A}}{m_\chi^2}. 
\end{equation}

If the DM is captured, it eventually thermalizes to an average velocity
\begin{equation}
v_\text{th} \sim \sqrt{\frac{T}{m_\chi}} \sim 10^{-12} \l \frac{m_\chi}{10^{16} ~\GeV}\r^{-1/2}
\end{equation}
and settles at the thermal radius
\begin{align}
R_\text{th} \sim \l \frac{T}{G m_\chi \rho_\text{WD}}\r^{1/2} \sim 0.1 ~\cm \l \frac{m_\chi}{10^{16} ~\GeV}\r^{-1/2},
\end{align}
where its kinetic energy balances against the gravitational potential energy of the (enclosed) WD mass. 
For simplicity we take a constant WD density $\rho_\text{WD} \sim n_\text{ion} m_\text{ion}$ within $R_\text{th}$.
Of course, the timescale to reach thermalization depends on the nature of the DM-SM interaction.
This has been explicitly calculated in the case that the DM loses energy via elastic nuclear scatters, see \cite{Tinyakov}. 
First, the DM passes through the WD many times before the size of its orbit becomes fully contained within the star.
This occurs after a time
\begin{equation}
t_1 \sim \l \frac{m_\chi}{m_\text{ion}} \r^{3/2} \frac{R_\text{WD}}{v_\text{esc}} \frac{1}{\bar{N}_\text{scat}} \frac{1}{\text{max}\{\bar{N}_\text{scat}, 1\}^{1/2}} \sim 2 \times 10^{3} ~\text{yr} \l \frac{m_\chi}{10^{10} ~\GeV} \r^{3/2} \l \frac{\sigma_{\chi A}}{10^{-38} ~\cm^2} \r^{-3/2}. 
\end{equation}
Note that the time to complete a single orbit is simply the free-fall timescale:
\begin{equation}
\label{eq:freefalltime}
t_\text{ff} \sim \sqrt{\frac{1}{G \rho_\text{WD}}} \sim 0.1 ~\text{s}.
\end{equation}
This stage is relevant only if the energy loss after a single transit  does not exceed $\sim m_\chi v_\text{esc}^2$:
\begin{equation}
\l \frac{m_\text{ion}}{m_\chi} \r \text{max}\{\bar{N}_\text{scat},1\} < 1. 
\end{equation}
This is the case for any cross sections which satisfy the XENON bound \eqref{eq:xenon}. 
Subsequently, the DM completes many orbits within the star until dissipation from elastic scatters reduces the orbital size to the thermal radius.
This occurs after a time
\begin{equation}
t_2 \sim \l \frac{m_\chi}{m_\text{ion}} \r \frac{1}{n_\text{ion} \sigma_{\chi A}} \frac{1}{v_\text{ion}} \left \{ 1+\log \l \frac{m_\chi}{m_\text{ion}}\r \right \} \sim 10^2 ~\text{yr} \l \frac{m_\chi}{10^{10} ~\GeV} \r \l \frac{\sigma_{\chi A}}{10^{-38} ~\cm^2} \r^{-1}. 
\end{equation}
where $v_\text{ion} \sim \sqrt{\frac{T}{m_\text{ion}}}$.  
Thus, DM will only settle at the thermal radius if the total thermalization time is shorter than the age of the WD:
\begin{equation}
t_1 + t_2 < \tau_\text{WD}.
\end{equation}

We now turn towards the rate of DM-DM collisions for captured DM. 
Ultimately, the most interesting constraint in the capture scenario will be from the focusing of annihilations during self-gravitational collapse of a DM sphere at the thermal radius.
However, we first carefully examine the effect of annihilations on the evolution of captured DM. 
To begin, the settling DM constitutes a number density of DM throughout the WD volume as well as outside the star. 
\textcolor{blue}{discuss annihilations of infall?}
After a settling time has passed, DM will begin steadily accumulating at $R_\text{th}$.
If \eqref{eq:steadycollect} is satisfied, the accumulation rate is roughly the same as the capture rate. 
However, this density of accumulating DM is also depleting due to annihilations. 
Eventually, these two rates become comparable and there is an equilibrium number of DM particles
\begin{align}
N_\text{eq} \sim \l \frac{\Gamma_\text{cap} R_\text{th}^3}{\sigma_{\chi \chi} v_\text{th}} \r^{1/2} \sim 5 \times 10^{27} \l \frac{m_\chi}{10^{10} ~\GeV} \r^{-3/2} \l \frac{\sigma_{\chi \chi}}{10^{-40} ~\cm^2} \r^{-1/2}  \l \frac{\sigma_{\chi A}}{10^{-38} ~\cm^2} \r^{1/2} \l \frac{\rho_\chi}{0.4 ~\GeV/\cm^3} \r^{1/2}.
\end{align}
Of course, there is no guarantee that this equilibrium is achieved within the age of the WD. 
In that case, annihilations can be ignored and the total number of DM particles accumulated is simply
\begin{align}
N_\text{life} &\sim \Gamma_\text{cap} \tau_\text{WD} \sim 10^{30}  \l \frac{m_\chi}{10^{10} ~\GeV} \r^{-2}  \l \frac{\sigma_{\chi A}}{10^{-38} ~\cm^2} \r \l \frac{\rho_\chi}{0.4 ~\GeV/\cm^3} \r
\end{align}
However, if the collected mass of DM at the thermal radius ever exceeds the WD mass within this volume, then there is the possibility of self-gravitational collapse of the DM.
The critical number of DM for self-gravitation is given by
\begin{align}
\label{eq:Ncore}
    N_\text{sg} \sim \frac{\rho_\text{WD} R^3_\text{th}}{m_\chi} \sim 10^{27} \l \frac{m_\chi}{10^{10} ~\GeV} \r^{-5/2}.
\end{align}
This can only be achieved if the time to collect a critical mass of DM is shorter than the time for annihilations to deplete this mass sufficiently \emph{and} shorter than the WD lifetime. 
Thus the condition for collapse is:
\begin{equation}
\label{eq:collapsecondition}
N_\text{sg} < N_\text{eq}, ~~~~ N_\text{sg} < N_\text{life}. 
\end{equation}
DM masses less than $\sim 10^{6} ~\GeV$ do not have enough time within the age of the WD to collect a number $N_\text{sg}$ and begin a collapse.
Note that the collecting DM at $R_\text{th}$ is adequately described by Maxwell-Boltzmann statistics so long as the number of particles is less than
\begin{equation}
N_\text{QM} \sim R_\text{th}^3 (m_\chi T)^{3/2} \sim 10^{52}. 
\end{equation}
This number is well beyond the critical number needed for self-gravitation if $m_\chi \gg \GeV$. 

Once the DM sphere becomes self-gravitating, it will start to collapse. 
Although the evolution of the collapsing DM is initially set by the dynamical free-fall time, it is ultimately slowed by the need to release gravitational potential energy and happens on a timescale set by DM cooling. 
In this sense the nature of the collapsing DM sphere shares similarities with the formation of DM halos, with an essential difference being dissipation. 
If the dominant source of dissipation is elastic nuclear scatters, the timescale for cooling is initially independent of DM velocity but decreases once the DM velocity exceeds the velocity of ions $v_\text{ion}$:
\begin{equation}
t_\text{cool}(r) \sim \frac{m_\chi}{\rho_\text{WD} \sigma_{\chi A}} \frac{1}{\text{max}\{{v_\text{ion},v_\chi\}}},
\end{equation}
where the velocity of DM is simply set by the virial relation 
\begin{equation}
v_\chi \sim \sqrt{\frac{G N_\text{sg} m_\chi}{r}}. 
\end{equation}
It is straightforward to see that the dynamical free-fall time $\propto r^{3/2}$ is always smaller the collapse time, so that the DM requires many orbits before dissipation affects its trajectory. 
The timescale for $\OO(1)$ of the collapsing DM at a radius $r$ and number $N_\text{sg}$ to deplete by annihilations is simply:
\begin{equation}
t_\text{ann}(r) \sim \frac{r^3}{N_\text{sg} \sigma_{\chi \chi} v_\chi}. 
\end{equation}
The collapse of the entire DM sphere proceeds until the timescale for annihilations becomes of order the timescale for cooling.
If this occurs while the DM velocity is less than $v_\text{ion}$, the radius at which annihilations become relevant is:
\begin{align}
R_1 \sim R_\text{th} \l \frac{\sigma_{\chi \chi}}{\sigma_{\chi A}}\r^{2/7} \l \frac{m_\text{ion}}{m_\chi} \r^{1/7}.
\end{align}
Otherwise, this radius is
\begin{align}
R_2 \sim R_\text{th} \l \frac{\sigma_{\chi \chi}}{\sigma_{\chi A}}\r^{1/3}. 
\end{align}
In other words, the number of collapsing DM particles is depleting by an $\OO(1)$ fraction once the DM sphere has focused to within the minimum of these two scales:
\begin{equation}
R_{\chi \chi} =  \text{min}\{R_1, R_2\}. 
\end{equation}
Of course, such a collapse of the accumulated DM in the WD is only sensible if
\begin{align}
\label{eq:xicondition}
R_{\chi \chi} < R_\text{th}.
\end{align}
It turns out this condition is ultimately satisfied for any DM mass and annihilation cross section which satisfy the ignition condition \eqref{eq:multcolboom} during a collapse. 
%Note, the enclosed WD mass is also dropping by $M_\text{WD}(r) \propto r^3$ during the collapse, so if the DM sphere depletes as a stronger function of radius then the collapse will halt below $R_{\chi \chi}$. 

We also briefly mention the possibility that the number of DM particles initially collapsing can be much larger than $N_\text{sg}$. 
\textcolor{blue}{elaborate}. 
Once the accumulated DM becomes self-gravitating, evolution of the DM sphere can either be collapse (by cooling) or further collection.
The later occurs if the timescale for collapse is greater than the timescale of collection:
\begin{equation}
\frac{N_\text{sg}}{\Gamma_\text{cap}} < \frac{m_\chi}{\rho_\text{WD} \sigma_{\chi A} v_\text{ion}},
\end{equation}
which occurs for DM masses above $\sim 10^{14} ~\GeV$, independent of $\sigma_{\chi A}$. 
In this case, the number of collapsing DM particles saturates to $\Gamma_\text{cap} t_\text{cool}$, with the the cooling time set to match the collection time.  
If in addition the collection time is also initially shorter than the dynamical free-fall time, then the DM sphere at $R_\text{th}$ will simply continue to collect more DM until a saturation number $\sim N_\text{sg}^{1/3} (\Gamma_\text{cap} t_\text{ff})^{2/3}$ greater than $N_\text{sg}$ is reached. 

There are two potential evolutions of the captured DM: either the DM collapses or it does not. 
In the later case, the DM has either reached its equilibrium number at the thermal radius or is still continuing to accumulate, not yet having the critical mass necessary for collapse within its lifetime.
We have checked that this scenario does not yield any constraints on DM parameters which ignite the star. 
%The number of collisions that can be counted as a single heating event is roughly
%\begin{equation}
%\label{eq:nocollapse}
%N_\text{mult} \sim \l \frac{\text{min}\{N_\text{eq}, N_\text{life}\}}{R_\text{th}^3} \r^2 \sigma_{\chi \chi} v_\text{th} L_\text{heat}^3 \tau_\text{diff}, ~~~~ L_\text{heat} \equiv \text{max}\{\lambda_T, L_0\}. 
%\end{equation}
%Even in the ``best-case" scenario of efficient capture and $L_0 \sim \lambda_T$, we find there is no parameter space $\{m_\chi, \sigma_{\chi \chi}\}$ where both \eqref{eq:nocollapseann} and \eqref{eq:multcolboom}---with $N_\text{mult}$ given by \eqref{eq:nocollapse}---are simultaneously satisfied. 
Thus we turn our attention to collapsing DM, characterized by \eqref{eq:collapsecondition}. 
The number of collisions $N_\text{mult}$ that are counted as a single heating event is general given by an integral of the annihilation rate over a diffusion time $\tau_\text{diff}$.
This of course changes with the radius of the collapsing DM sphere.
Since we are interested in releasing $\Eboom$ of energy, we focus on DM collapse at an optimum radius $R_*$ at which point the annihilation rate is maximal.
Assuming the diffusion time is the shortest timescale during the collapse and the number of particles collapsing is $N_\text{sg}$, this number is of order:
\begin{equation}
\label{eq:nmulti}
N_\text{mult} \sim \l \frac{N_\text{sg}}{R_*^3}\r^2  \sigma_{\chi \chi} v_\chi ~\text{min}\{\lambda_T, R_*\}^3 \tau_\text{diff},
\end{equation}
Naively, we expect that the rate of annihilations increases as the radius decreases but only up until the DM sphere collapses to $R_{\chi \chi}$.
Of course, the process of cooling via elastic nuclear scatters becomes more involved if the DM sphere ever falls below the typical spacing of ions $n_\text{ion}^{-1/3} \sim 10^{-10} ~\text{cm}$. 
Moreover, there may be some stabilizing pressure which prevents the DM from ever collapsing or annihilating below a certain radius.
This is to be expected for heavy, composite DM, although such a stable radius would depend on unknown physics. 
Famously, gravity itself provides such a ``pressure", arresting collapses below the Schwarzschild radius by the formation of a black hole:
\begin{equation}
R_\text{BH} \sim G N_\text{sg} m_\chi \sim 5 \times 10^{-15} ~\cm \l \frac{m_\chi}{10^{10} ~\GeV} \r^{-3/2}.
\end{equation}

If the DM is a fermion, degeneracy pressure may also stall the collapse (although collisions are still allowed). 
Even more famously, this only takes place if the total mass of collapsing DM $N_\text{sg} m_\chi$ is below the Chandrasekhar number:
\begin{equation}
N_\text{Cha} \sim \frac{1}{G^{3/2} m_\chi^3} \sim 10^{27} \l \frac{10^{10} ~\GeV}{m_\chi}\r^{-3}.
\end{equation}
Thus, degeneracy pressure is relevant for DM masses $m_\chi \lesssim 10^{9} ~\GeV$.
If this is the case, the DM will temporarily stabilize at a radius
\begin{equation}
\label{eq:deg}
R_\text{deg} \sim \frac{1}{G m_\chi^3 N_\text{sg}^{1/3}} \sim 3 \times 10^{-11} ~\cm \l\frac{m_\chi}{10^{8} ~\GeV}\r^{-13/6},
\end{equation}
at which point the DM sphere will either continue collecting until it reaches the Chandrasekhar mass or annihilate beforehand. 
If the timescale for annihilations is less than the time to collect $\sim N_\text{sg}$ DM particles, than $\OO(1)$ of the DM sphere annihilates while still supported by degeneracy pressure at the initial radius $R_\text{deg}$. 
If the timescale for annihilations matches the collection time for a number great than $N_\text{sg}$ but less than the Chandrasekhar number $M_\text{ch}/m_\chi$, we have checked that the energy released is not sufficient to satisfy\eqref{eq:multcolboom}. 
As a result, for fermions we take
\begin{equation}
R_* = \text{max}\left \{R_{\chi \chi}, \text{max}\{R_\text{BH}, R_\text{deg}\} \right \}. 
\end{equation}

If the DM is a boson, the pressure induced by the uncertainty principle is insufficient to stall the collapse as the self-gravitating DM number is far greater than the ``bosonic" Chandrasekhar number
\begin{equation}
N^\text{bos}_\text{Cha} \sim \frac{1}{G m_\chi^2} \sim 10^{18}  \l \frac{10^{10} ~\GeV}{m_\chi}\r^{-2}.
\end{equation}
However, the collapsing DM sphere will still form a BEC once it reaches critical density at the radius \eqref{eq:deg}. 
At this radius, further collapse of the DM sphere goes towards populating the newly formed BEC (while the density of the non-condensed DM remains roughly constant). 

%If the DM stabilizes into a BH, the condition that not even a \emph{single} collision occurs during the collapse to $R_\text{BH}$ is simply
%\begin{equation}
%\l \frac{N_\text{sg}}{R_\text{BH}^3}\r \sigma_{\chi \chi} \frac{m_\chi}{\rho_\text{WD} \sigma_{\chi A}} \lesssim 1. 
%\end{equation}
%This is the most stringent bound that can be placed on DM masses greater than $\Eboom$, for which a single collision is capable of igniting the star. 

\subsection{Constraints}
In Figure \ref{fig:multicapture}, we derive the constraints on $\sigma_{\chi \chi}$ from the focusing of multiple collisions during gravitational collapse due of DM in the star.
Bounds are from the observation of a single $1.25 ~ M_{\astrosun}$ in our local DM density. 
The results of Figure \ref{fig:multicapture} are valid for any SM annihilation products which deposit their energy compactly upon release within the trigger size $\lambda_T$.
\begin{figure}
\includegraphics[scale=.35]{multicollision.pdf}
\caption{Constraints on DM-DM annihilation cross-section into SM particles which deposit their energy compactly within a trigger size $\lambda_T$ during self-gravitational collapse in a WD. Bounds come from observation of a single $1.25~M_{\astrosun}$ WD. Here we assume a value of $\sigma_{\chi A}$ saturated at the direct detection bounds.}
\label{fig:multicapture}
\end{figure}
\end{document}