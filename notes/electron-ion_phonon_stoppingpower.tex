\documentclass{article}

\usepackage[colorlinks=true]{hyperref}
\usepackage{amsmath}
\usepackage{amssymb}
\usepackage{accents}
\usepackage{parskip}
\usepackage{bm}
\usepackage{xcolor}
\usepackage{mathtools}
\usepackage{wasysym}

%%%%%%
\usepackage[framemethod=TikZ]{mdframed}
\mdfdefinestyle{MyFrame}{%
    linecolor=black,
    outerlinewidth=1pt,
    roundcorner=10pt,
    innertopmargin=\baselineskip,
    innerbottommargin=\baselineskip,
    skipabove=\baselineskip,
    skipbelow=\baselineskip,
    innerrightmargin=20pt,
    innerleftmargin=20pt,
    backgroundcolor=gray!20!white}
\usepackage{xpatch}
\makeatletter
\xpatchcmd{\endmdframed}
  {\aftergroup\endmdf@trivlist\color@endgroup}
  {\endmdf@trivlist\color@endgroup\@doendpe}
  {}{}
\makeatother
%%%%%%

\newcommand{\OO}{\mathcal{O}}
\newcommand{\LL}{\mathcal{L}}
\newcommand{\HH}{\mathcal{H}}
\newcommand{\NN}{\mathcal{N}}
\newcommand{\DD}{\mathcal{D}}
\newcommand{\hc}{\text{ h.c.}}

\newcommand{\angstrom}{\buildrel _{\circ} \over {\mathrm{A}}}
\newcommand{\GeV}{\text{GeV}}
\newcommand{\Hz}{\text{Hz}}
\newcommand{\eV}{\text{eV}}
\newcommand{\MeV}{\text{MeV}}

\newcommand{\ket}[1]{\ensuremath{\left|#1\right>}}
\newcommand{\bra}[1]{\ensuremath{\left<#1\right|}}
\newcommand{\braket}[2]{\ensuremath{\left<#1|#2\right>}}


\title{Electron stopping power via phonon production in the ion lattice}
\author{Paul Riggins}
\begin{document}
\maketitle{}



\section{Goal and Setup}

Our ultimate goal is to compute the stopping power
\begin{align}
\frac{d E}{d x} = \int dE' \frac{d \sigma}{dE'}E'
\end{align}
for an electron scattering off an ion coulomb lattice. Here $E'$ is the energy transfer. For energy transfers below the lattice binding energy, we expect phonon production to become relevant, and the cross-section must take this into account. To capture this behavior we will use the structure factor formalism developed, for instance, in \cite{quantum} Chapter~19.

The double differential cross-section for a particle to scatter into solid angle $\Omega$ while transferring energy $\omega$ and momentum $\textbf{q}$ is given by (\cite{quantum} 19.13)
\begin{align}
\frac{d \sigma}{d \Omega d \omega} = A_\textbf{q} S(\omega, \textbf{q})
\label{eq:DoubleDifferential}
\end{align}
where (\cite{quantum} 19.14)
\begin{align}
A_\textbf{q} = \frac{k'}{k} \left( \frac{m}{2 \pi} \right)^2 |V_\textbf{q}|^2
\end{align}
Here $m$ is the incident particle mass, and $k$ and $k'$ are the magnitudes of the incident particle's initial and final momentum, respectively. These momenta are related to the momentum transfer by $\textbf{k}' = \textbf{k} - \textbf{q}$ (\cite{quantum} 19.3). The fourier transform of the interaction potential is (\cite{quantum} 19.6)
\begin{align}
V_\textbf{q} = \int d^3 x\, e^{i\textbf{q}\cdot \textbf{x}} V(\textbf{x})
\end{align}
For momentum and energy transfers much smaller than the Fermi momentum and energy, we should be able to use a simple coulomb potential for $V(\textbf{x})$. For small transfers, however, we must take into account screening by the Fermi sea, for instance via a dielectric constant (see \cite{dwarf} eq. 9 and \cite{french} 5.67). We will return to this later after we have developed the formalism further.

The structure factor $S(\omega, q)$ is the space-time fourier transform of the density correlation function (\cite{quantum} 19.15-17, 19.40)
\begin{align}
S(\omega, \textbf{q}) &= \frac{1}{2 \pi} \int_{-\infty}^{\infty} dt \, e^{-i\omega t} F(\textbf{q}, t) \\
   &= \frac{1}{2 \pi} \int_{-\infty}^{\infty} dt\, e^{-i \omega t} \sum_{jl} \left\langle e^{-i\textbf{q}\cdot \textbf{x}_j(0)}e^{i\textbf{q}\cdot \textbf{x}_l(t)} \right\rangle_T \\
   &= \frac{1}{2 \pi} \int d^3 x \, d^3 x'\, e^{i \textbf{q}\cdot(\textbf{x} - \textbf{x'})} \int dt \, e^{-i \omega t} \langle \rho(\textbf{x}', 0) \rho(\textbf{x}, t) \rangle_T
\end{align}
where the intermediate scattering function $F(\textbf{q}, t)$ is given by
\begin{align}
F(\textbf{q}, t) &= \sum_{jl} \left\langle e^{-i\textbf{q}\cdot \textbf{x}_j(0)}e^{i\textbf{q}\cdot \textbf{x}_l(t)} \right\rangle_T
\end{align}
and the density operator is
\begin{align}
\rho(\textbf{x}, t) = \sum_j \delta[\textbf{x} - \textbf{x}_j(t)]
\end{align}
The expectation values are taken over the thermal phonon distribution at temperature $T$. In the one-phonon approximation, the elastic and inelastic parts of $F(\textbf{q}, t)$ are given by (\cite{quantum} 19.40, 19.51)
\begin{align}
F_\text{el}(\textbf{q}, t) &= (2 \pi)^3 n_i e^{-2 W} \sum_\textbf{G} \delta^{(3)}(\textbf{q} - \textbf{G}) \\
F_\text{in}(\textbf{q}, t) &= (2 \pi)^3 n_i e^{-2 W} \sum_\nu \frac{q^2}{2 NM \omega_\nu} \Bigg( \langle n_\nu + 1 \rangle e^{i \omega_\nu t} \sum_\textbf{G} \delta^{(3)}(\textbf{q} - \textbf{k}_\nu - \textbf{G}) \nonumber\\  & \hspace{4 cm} + \langle n_\nu \rangle e^{-i \omega_\nu t} \sum_\textbf{G} \delta^{(3)}(\textbf{q} + \textbf{k}_\nu - \textbf{G})  \Bigg)
\label{eq:Fin}
\end{align}
where $N$ is the number of ions in the lattice, $M$ is the ion mass, $n_i$ is the ion number density, $\nu$ indexes the phonon modes of energy $\omega_\nu$ and quasimomentum $\textbf{k}_\nu$, $\textbf{G}$ is a reciprocal lattice factor, and $\langle n_\nu \rangle$ is the expected number of phonons of index $\nu$ at temperature $T$. This last quantity is given by the usual boson statistics:
\begin{align}
\langle n_\nu \rangle = \frac{e^{-\omega_\nu/T}}{1-e^{-\omega_\nu/T}} = \frac{1}{e^{\omega_\nu/T} - 1}
\end{align}
The quantity $W = \frac16 q^2 \langle u_j^2 \rangle$ is the ``Debye-Waller factor'', reflecting decoherence that arises from zero-point and finite temperature effects.

\subsection{Debye approximation}

For simplicity of calculation, we will work in the Debye approximation. In the Debye approximation the Debye-Waller factor is given by (\cite{quantum} 19.56)
\begin{align}
W = \frac16 q^2 \langle u_j^2 \rangle = \frac{3 q^2}{8 M \Theta}\left\{ 1 + \frac{2 \pi^2}{3}\left( \frac{T}{\Theta} \right)^2 + \cdots \right\}
\end{align}
Here $\Theta$ there is the Debye temperature. The Debye approximation describes a phonon distribution of $N$ phonon modes with energies between $ \sim \omega_D / N$ and $\omega_D$, where the maximum photon energy is given by (\cite{intro} 5.21-24)
\begin{align}
\omega_D = c_s(6 \pi^2 n_i)^{1/3}
\label{eq:DebyeEnergy}
\end{align}
and $c_s$ is the sound speed. The Debye temperature is given by $\Theta = \omega_D$ (\cite{intro} 5.28). The dispersion relation for phonons in this approximation is given by
\begin{align}
\omega_\nu = c_sk_\nu
\end{align}
and the density of states is given by
\begin{align}
D(\omega_\nu) = \frac{V\omega_\nu^2}{2 \pi^2 c_s^3}
\end{align}
where $V$ is the lattice volume.

\section{Putting it all together}

Let us now put all this together to calculate the double differential cross-section for inelastic scattering leading to phonon production in the medium. We neglect elastic scattering, since it does not contribute to the stopping power, and we will ignore the possibility of phonon absorption on the assumption that it will be negligible. (We can check this later.)

The phonon production term in $F(\textbf{q}, t)$ is $ \propto \langle n_\nu + 1 \rangle$. Selecting this term, we find the structure factor to be
\begin{align}
S(\omega, \textbf{q}) &= \frac{1}{2 \pi} \int_{-\infty}^\infty dt \, e^{-i \omega t} (2 \pi)^3 n_i e^{-2 W}  \nonumber\\
 &\qquad\qquad  \sum_\nu\frac{q^2}{2 NM \omega_\nu} \langle n_\nu + 1 \rangle e^{i \omega_\nu t} \sum_\textbf{G} \delta^{(3)}(\textbf{q} - \textbf{k}_\nu - \textbf{G})
\end{align}
Evaluating the time integral yields a delta function in energy
\begin{align}
S(\omega, \textbf{q}) &= \frac{1}{2 \pi} (2 \pi)^3 n_i e^{-2 W}  \nonumber\\
&\qquad\qquad  \sum_\nu\frac{q^2}{2 NM \omega_\nu} \langle n_\nu + 1 \rangle \delta(\omega - \omega_\nu) \sum_\textbf{G} \delta^{(3)}(\textbf{q} - \textbf{k}_\nu - \textbf{G})
\label{eq:1PhononStructureFactor}
\end{align}
We know now that we have two delta functions that are not independent of one another. The momentum transfer $\textbf{q}$ and energy transfer $\omega$ are related to each other by the kinematics of the incident particle, and the phonon energy $\omega_\nu$ and crystal momentum $\textbf{k}_\nu$ related to each other by the dispersion relation. Specifically, we have the relations
\begin{align}
\omega &= \sqrt{m^2 + k^2} - \sqrt{m^2 + |\textbf{k} - \textbf{q}|^2} \label{eq:EnergyTransfer} \\
\omega_\nu &= c_sk_\nu \label{eq:DispersionRelation}
\end{align}
where $\textbf{k}$ is the initial momentum of the incident particle of mass $m$. Let's begin by considering a highly relativistic particle with small momentum transfer, so that
\begin{align}
\omega &\approx k - |\textbf{k} - \textbf{q}| \\
   &= k - \sqrt{k^2 + q^2 - 2 kq\cos\theta_{q}} \\
   &\approx k - k \left( 1 + \frac12 \frac{q^2}{k^2} - \frac{q}{k}\cos\theta_{q} \right) \\
   &\approx q\cos\theta_{q} \label{eq:EnergyMomentumRelation}
\end{align}
where $\theta_q$ is the angle of the momentum transfer vector relative to the incident momentum $\textbf{k}$.
To simplify, let us also suppose that we are only interested in momentum transfers within first Brillioun zone $q\lesssim k_D$ and energy transfers $\omega \lesssim \omega_D$. Then we may take only the $\textbf{G} = 0$ term in the reciprocal lattice sum. We may then rewrite the structure factor in terms of momenta as
\begin{align}
S(\omega, \textbf{q}) &= \frac{1}{2 \pi} (2 \pi)^3 n_i e^{-2 W}  \nonumber\\
&\qquad \sum_\nu\frac{q^2}{2 NM (c_sk_\nu)} \langle n_\nu + 1 \rangle \delta(q \cos\theta_{q} - c_sk_\nu) \delta^{(3)}(\textbf{q} - \textbf{k}_\nu)
\end{align}
We can approximate the sum over phonons as an integral over the first Brillioun zone, including a factor of 3 for the various polarizations:
\begin{align}
\sum_\nu \to 3 \int^{k_D} d^3 k_\nu\frac{V}{(2 \pi)^3}  % = 3 \int^{k_D} k_\nu^2 \sin\theta_{k_\nu}\,\frac{V}{(2 \pi)^2}\, dk_\nu \, d \theta_{k_\nu} \, d \phi_{k_\nu}
\end{align}
We use this to collapse the three-dimensional delta function (remember we are assuming $q$ lies within the first Brillioun zone):
\begin{align}
S(\omega, \textbf{q}) &= \frac{1}{2 \pi} (2 \pi)^3 n_i e^{-2 W}  \nonumber\\
&\qquad \left( \frac{3V}{(2 \pi)^3} \right)\frac{q^2}{2 NM (c_sq)} \left[ \frac{1}{e^{c_sq / T} - 1} + 1 \right] \delta(q\cos\theta_{q} - qc_s)
\end{align}
We note that $V / N = n_i^{-1}$. Simplifying this expression, we find
\begin{align}
S(\omega, \textbf{q}) &= \frac{3}{4 \pi M c_s} \left[ \frac{1}{e^{c_sq / T} - 1} +1 \right] \delta(\cos\theta_{q} - c_s)
\end{align}
Note that we have extracted the momenta factor $q$ from the remaining delta function. We may do this because we are only considering processes with non-zero momentum transfer. We have also neglected the Debye-Waller factor, since it is small and the exponential is $\OO(1)$.

\section{Calculating the stopping power}

In order to find the singly-differential cross-section $d \sigma / d \omega$ and the stopping power, we are about to integrate over $\cos\theta_q$, $\phi_q$, and $\omega$. It will be convenient therefore change variables so our structure factor is in terms of these. Recall that $\omega \approx q \cos\theta_q$ \eqref{eq:EnergyMomentumRelation}. Then the structure factor becomes
\begin{align}
\label{eq:StructureFactorVariableChange}
S(\omega, \cos\theta_q, \phi_q) &= \frac{3}{4 \pi M c_s} \left[ \frac{1}{e^{c_s\omega / T\cos\theta_q} - 1} + 1 \right] \delta(\cos\theta_{q} - c_s)
\end{align}
The singly-differential cross-section is given by (see \eqref{eq:DoubleDifferential} and following)
\begin{align}
\frac{d \sigma}{d \omega} = \int d \Omega\, \frac{k'}{k}\left( \frac{m}{2 \pi} \right)^2 |V_\textbf{q}|^2 S(\omega, \cos\theta_q, \phi_q)
\label{eq:SinglyDifferential}
\end{align}
We must now calculate the interaction potential and the differential $d \Omega$.

\subsection{Interaction potential}

We will use Heaviside-Lorentz units for electromagnetism, so that $\epsilon_0 = 1$. Then the coulomb force is given by $V(\textbf{r}) = e^2 / 4 \pi r = \alpha / r$. Accounting for screening, the fourier transform of this interaction potential is given by \cite{french}
\begin{align}
V_\textbf{q} = \frac{4 \pi Z\alpha}{q^2 \epsilon^l(q, 0)}
\end{align}
where the electron longitudinal static dielectric function relative to the vacuum $\epsilon^l(q, 0)$ is given by
\begin{align}
\epsilon^l(q, 0) = 1 + \frac{4 e^2 E_F \sqrt{E_F^2 + m^2}}{\pi q^2}
\end{align}
so we can write
\begin{align}
V_\textbf{q} = \frac{4 \pi Z\alpha}{q^2 + \lambda^2} = \frac{4 \pi Z\alpha}{(\omega / \cos\theta_q)^2+ \lambda^2}
\label{eq:ScreenedInteractionPotential}
\end{align}
with the screening momentum
\begin{align}
\lambda = \left( \frac{\pi}{4 \alpha E_F \sqrt{E_F^2 + m^2}} \right)^{-1/2} \approx \left( \frac{\pi}{4 \alpha E_F m} \right)^{-1/2}
\end{align}
We cannot ignore the dielectric function because our momentum transfers are coincidentally smaller than the Fermi energy by virtue of being smaller than the lattice binding energy.

\subsection{Solid angle $d \Omega$}
\label{sec:SolidAngle}

We wish to rewrite the solid angle differential $d \Omega = d(\cos\theta_{k'})d \phi_{k'}$ in terms of $d(\cos\theta_q)$ and $d\phi_q$. To begin, we recall that $\textbf{k}' = \textbf{k} - \textbf{q}$. The momentum transfer vector $\textbf{q}$ entirely determines the transverse component of $\textbf{k}'$, so we may relate
\begin{align}
q \sin\theta_q = k' \sin\theta_{k'} \approx k\sin\theta_{k'}
\end{align}
where in the last step we used the fact that $q\ll k$. Manipulating and using the trig identity $\sin\theta = \sqrt{1 - \cos^2 \theta}$, we find
\begin{align}
\cos\theta_{k'} &\approx\sqrt{1 - \left( \frac{q}{k} \right)^2 \sin^2 \theta_q} \label{eq:coskp}\\
  &\approx 1 - \frac12\left( \frac{q}{k} \right)^2 \sin^2 \theta_q \\
  &\approx 1 - \frac12\left( \frac{q}{k} \right)^2 (1 - \cos^2 \theta_q) \\
  &\approx 1 - \frac{\omega^2}{2 k^2} \left( \frac{1}{\cos^2 \theta_q} - 1 \right) \\
\end{align}
Thus we have the differential
\begin{align}
d(\cos\theta_{k'}) \approx \frac{\omega^2}{k^2}\frac{d(\cos\theta_q)}{\cos^3 \theta_q}
\end{align}
The azimuthal angles are simply the same, so the solid angle differential becomes
\begin{align}
d \Omega &= d(\cos\theta_{k'})\,d \phi_{k'} \\
 &\approx d(\cos\theta_q)\,d \phi_q\,\frac{\omega^2}{k^2}\frac{1}{\cos^3 \theta_q}
 \label{eq:SolidAngle}
\end{align}

\subsection{Singly-differential cross-section}

We are now ready to calculate the singly-differential cross-section \eqref{eq:SinglyDifferential}. Approximating $k' \approx k$ and inserting the results \eqref{eq:StructureFactorVariableChange}, \eqref{eq:ScreenedInteractionPotential}, and \eqref{eq:SolidAngle}, we can easily evaluate the integrals (with the help of the delta function) to find
\begin{align}
\frac{d \sigma}{d \omega} &= \int d \Omega\, \left( \frac{m}{2 \pi} \right)^2 |V_\textbf{q}|^2 S(\omega, \cos\theta_q, \phi_q) \\
 &\approx \int \left[ d(\cos\theta_q)\,d \phi_q\,\frac{\omega^2}{k^2}\frac{1}{\cos^3 \theta_q} \right] \left( \frac{m}{2 \pi} \right)^2 \left[ \frac{4 \pi Z\alpha}{(\omega / \cos\theta_q)^2+ \lambda^2} \right]^2 \\
 &\qquad \times\left\{ \frac{3}{4 \pi M c_s} \left[ \frac{1}{e^{c_s\omega / T\cos\theta_q} - 1} + 1 \right] \delta(\cos\theta_{q} - c_s) \right\} \nonumber \\
 &= 2 \pi \left[\frac{\omega^2}{k^2}\frac{1}{c_s^3} \right] \left( \frac{m}{2 \pi} \right)^2 \left[ \frac{4 \pi Z\alpha}{(\omega / c_s)^2+ \lambda^2} \right]^2 \frac{3}{4 \pi M c_s} \left[ \frac{1}{e^{\omega / T} - 1} + 1 \right]
\end{align}
We recall that the speed of sound $c_s \sim 10^{-2}$.

\subsection{Stopping power}

All that remains is to integrate this overall possible energy transfers within the first Brillioun zone to find the stopping power. The result is
\begin{align}
\frac{dE}{d x} &= \int d\omega\, \frac{d \sigma}{d \omega} n_i \omega \\
  &\approx \frac{3 \alpha^2 m^2 n_i Z^2}{k^2 M} \left( \frac{\omega_D^2}{c_s^2 \lambda^2 + \omega_D^2} + \ln \left[ \frac{c_s^2 \lambda^2}{c_s^2 \lambda^2 + \omega_D^2} \right]  \right) \\
  &\sim \frac{3 \alpha^2 m^2 n_i Z^2}{k^2 M} (10)
\end{align}

\section{Beyond the first Brillioun zone}

When we restricted to the first Brillioun zone above, we found a phase space suppression of $ \sim q^2 / k^2$ resulting from the restriction to highly forward scatters. In actuality, however, we can transfer a nearly arbitrary amount of momentum to the let us while still in the transferring a minuscule amount of energy. Any large amount of momentum that we transfer must simply be shifted to within the first Brillioun zone to determine how much energy the corresponding phonon carries. This is the meaning of the sum over reciprocal lattice factors $\sum_\textbf{G}$ that we neglected. Standard white dwarf lore, however, suggests that the dominant contributions to scatters come from distant Brillioun zones, because the typical momentum transfer $k_F > k_D$ is outside the first Brillioun zone, and there is more phase space available for larger momentum transfers. The same is true for us, except that our momentum transfers are $\OO(k)$ instead of $\OO(k_F)$. We will now explore how this modifies our results, sticking for now to the single-phonon approximation.

\subsection{Structure factor}

Before dropping the sum over reciprocal lattice vectors, the one-phonon structure factor is given by
\begin{align}
S(\omega, \textbf{q}) &= \frac{1}{2 \pi} (2 \pi)^3 n_i e^{-2 W}  \nonumber\\
&\qquad\qquad  \sum_\nu\frac{q^2}{2 NM \omega_\nu} \langle n_\nu + 1 \rangle \delta(\omega - \omega_\nu) \sum_\textbf{G} \delta^{(3)}(\textbf{q} - \textbf{k}_\nu - \textbf{G})
\tag{\ref{eq:1PhononStructureFactor}}
\end{align}
In order to compute the sum over $\textbf{G}$, we convert this into an integral:
\begin{align}
\sum_\textbf{G} \to \int \frac{d^3 G}{\left( \frac{2 \pi}{a} \right)^3} = \int \frac{d^3 G}{(2 \pi)^3 n_i}
\end{align}
where $a$ is the lattice spacing. This integral serves to remove the delta function involving $\textbf{G}$, leaving
\begin{align}
S(\omega, \textbf{q}) &= \frac{1}{2 \pi}  \sum_\nu\frac{q^2}{2 NM \omega_\nu} \langle n_\nu + 1 \rangle \delta(\omega - \omega_\nu)
\end{align}
where we have also dropped the Debye-Waller factor because the temperature is small. Also converting the phonon sum to integral and evaluating, we find
\begin{align}
   S(\omega, \textbf{q}) &= \frac{1}{2 \pi}  3 \int^{\textbf{k}_D} d^3 k_\nu \frac{V}{(2 \pi)^3} \frac{q^2}{2 NM \omega_\nu} \langle n_\nu + 1 \rangle \delta(\omega - \omega_\nu) \nonumber\\
   &= \frac{3}{(2 \pi)^4 n_i}  \int^{\textbf{k}_D} (k_\nu^2\, dk_\nu \, d \Omega_\nu) \frac{q^2}{2 M \omega_\nu} \langle n_\nu + 1 \rangle \delta(\omega - \omega_\nu) \nonumber\\
   &= \frac{3(4 \pi)}{(2 \pi)^4 n_i}  \int_0^{\omega_D} \left( \frac{\omega_\nu}{c_s} \right)^2 d \omega_\nu \frac{q^2}{2 M \omega_\nu} \left[ \frac{1}{e^{\omega_\nu/T} - 1} + 1 \right] \delta(\omega - \omega_\nu) \nonumber\\
   &= \frac{6}{(2 \pi)^3 n_i} \frac{q^2\omega}{2 M c_s^2} \left[ \frac{1}{e^{\omega/T} - 1} + 1 \right] \theta(\omega_D - \omega)
\end{align}
where in the last step a heavyside function emerges, enforcing the requirement that $\omega \leq \omega_D$ so that we do not transfer more energy than a single phonon could carry. Note we used the dispersion relation $k_\nu = \omega_\nu / c_s$ to change variables in the integral.

\subsection{Energy transfer, solid angle, interaction potential}

The energy transfer $\omega$ is still given by \eqref{eq:EnergyTransfer}, but we can no longer assume that $q \ll k$. It is still true, however, that $|\textbf{k} - \textbf{q}| \sim |\textbf{k}|$, because the energies of the incident and outgoing particle are only allowed to differ by the energy of a single phonon, which is small compared to $k$. Thus the energy transfer is
\begin{align}
\omega \approx k - |\textbf{k} - \textbf{q}| = k - \sqrt{k^2 + q^2 - 2 kq \cos\theta_q}
\end{align}
or, solving for the momentum transfer in terms of $\omega$ and $\cos\theta_q$, we find
\begin{align}
q &= k \cos\theta_q + \sqrt{\omega(\omega - 2 k) + k^2 \cos^2 \theta_q} \\
  &\approx 2k \cos\theta_q - \omega \sec \theta_q \\
  &\approx 2k \cos\theta_q \label{eq:DistantBrilliounZone1PhononMomentumTransfer}
\end{align}
where in the latter two approximations we assumed $\omega\ll k$. Note that the angle of the momentum transfer relative to the incident momentum is only allowed to range between $0$ and $\cos^{-1}\sqrt{2 \omega / k} \approx \pi / 2$. ($q$ must be real and positive.) This results from the kinematic requirement that the final momentum $k'$ be smaller in magnitude than the incident momentum $k$ by only as much as the energy transfer $\omega$. In the last approximation we have ignored this small energy transfer, allowing the momentum transfer angle to vary all the way up to $\pi / 2$.

In a moment we will also need the solid angle differential corresponding to the direction of the outgoing $\textbf{k}'$ (just as we calculated before in Section~\ref{sec:SolidAngle}). We start with equation~\eqref{eq:coskp}, but this time manipulate without assuming $q\ll k$:
\begin{align}
\cos\theta_{k'} &\approx\sqrt{1 - \left( \frac{q}{k} \right)^2 \sin^2 \theta_q} \tag{\ref{eq:coskp}}\\
  &\approx \sqrt{1 - \left( \frac{2 k \cos\theta_q}{k} \right)^2 \sin^2 \theta_q} \\
  &= \cos 2 \theta_q \\
  &= 2 \cos^2 \theta_q - 1
\end{align}
Thus the solid angle becomes
\begin{align}
d \Omega &= d(\cos\theta_{k'})d \phi_{k'} \\
   &= 4\cos\theta_q\, d(\cos\theta_q)d \phi_q
\end{align}

We will similarly need the interaction potential, which is given by equation~\eqref{eq:ScreenedInteractionPotential} except with $q$ given by \eqref{eq:DistantBrilliounZone1PhononMomentumTransfer}:
\begin{align}
V_\textbf{q} = \frac{4 \pi Z\alpha}{q^2 + \lambda^2} = \frac{4 \pi Z\alpha}{(2 k \cos\theta_q)^2+ \lambda^2}
\end{align}

\subsection{Singly differential cross-section}

The singly differential cross-section \eqref{eq:SinglyDifferential} (approximating $k' \approx k$) is now given by
\begin{align}
\frac{d \sigma}{d \omega} &= \int d \Omega\, \left( \frac{m}{2 \pi} \right)^2 |V_\textbf{q}|^2 S(\omega, \cos\theta_q, \phi_q) \\
   &\approx \int \left[ 4\cos\theta_q\, d(\cos\theta_q)d \phi_q \right] \left( \frac{m}{2 \pi} \right)^2 \left[ \frac{4 \pi Z\alpha}{(2 k \cos\theta_q)^2+ \lambda^2} \right]^2 \nonumber\\
   &\qquad\qquad \left[ \frac{6}{(2 \pi)^3 n_i} \frac{(2 k\cos\theta_q)^2\omega}{2 M c_s^2} \left[ \frac{1}{e^{\omega/T} - 1} + 1 \right] \theta(\omega_D - \omega) \right]
\end{align}
Recall that the only allowed momentum transfer angles are between 0 and $\pi/2$, so the integration over $d(\cos\theta_q)$ ranges from $0$ to $1$. Performing the angular integrals, we find
\begin{align}
\frac{d \sigma}{d \omega} &= \frac{3 m^2 Z^2 \alpha^2 \omega}{2\pi^2 c_s^2 k^2 M n_i} \left[ \ln \left( \frac{4 k^2 + \lambda^2}{\lambda^2} \right) - \frac{4k^2}{4 k^2 + \lambda^2} \right] \left[ \frac{1}{e^{\omega/T} - 1} + 1 \right] \theta(\omega_D - \omega)
\end{align}
The screening momentum $\lambda \sim 10^{-2}~\MeV \ll k$, so this reduces to
\begin{align}
\frac{d \sigma}{d \omega} &= \frac{3 m^2 Z^2 \alpha^2 \omega}{2\pi^2 c_s^2 k^2 M n_i} \ln \left( \frac{4 k^2}{\lambda^2} \right) \left[ \frac{1}{e^{\omega/T} - 1} + 1 \right] \theta(\omega_D - \omega)
\end{align}

\subsection{Stopping power}

We are now ready to compute the stopping power. Before we perform the integral, we note that the Bose enhancement for small $\omega$ goes like $ \sim T / \omega$. This is therefore counterbalanced by the factor of $\omega^2$ in the stopping power integral, and the bose enhancement plays no appreciable role unless $T \sim \omega_D$. (This has been confirmed numerically.) We may thus neglect that term, in which case the stopping power becomes
\begin{align}
\frac{dE}{d x} &= \int d \omega \,\frac{d \sigma}{d \omega} n_i \omega \\
  &\approx \int d \omega \,n_i \omega \left\{ \frac{3 m^2 Z^2 \alpha^2 \omega}{2\pi^2 c_s^2 k^2 M n_i} \ln \left( \frac{4 k^2}{\lambda^2} \right) \left[ 1 \right] \theta(\omega_D - \omega) \right\} \\
   &= \frac{m^2 Z^2 \alpha^2 \omega_D^3}{2 \pi^2 c_s^2 k^2 M} \ln\left( \frac{4 k^2}{\lambda^2} \right)
\end{align}
Inserting the Debye energy~\eqref{eq:DebyeEnergy}, we find the stopping power is
\begin{align}
\frac{d E}{d x} &= \frac{3 m^2 Z^2 \alpha^2 c_s n_i}{k^2 M} \ln\left( \frac{4 k^2}{\lambda^2} \right)
\end{align}
This result should be valid for all incident momenta $k\gg k_D = (6 \pi^2 n_i)^{1/3} \sim 4~\MeV$. 

\section{Beyond the one-phonon approximation}

In this section we will attempt to go beyond the one-phonon approximation to allow for larger transfers of energy and momentum from the electron to the lattice. If we do not employ the one-phonon approximation, the inelastic intermediate scattering function \eqref{eq:Fin} becomes
\begin{align}
  F_\text{in}(\textbf{q}, t) &= \sum_{ij} e^{i\textbf{q}\cdot(\textbf{R}_i - \textbf{R}_j)} \sum_{n = 1}^\infty \frac{1}{n!}\left\{ \frac{1}{2 MN} \sum_\nu \frac{q^2}{\omega_\nu} \langle n_\nu + 1 \rangle e^{-i \textbf{k}_\nu \cdot(\textbf{R}_i - \textbf{R}_j) + i \omega_\nu t} \right\}^n
\end{align}
Note we have dropped the Debye-Waller factor $e^{-2 W}$ and the phonon-absorption terms because of the low temperature of the star relative to the lattice binding energy. The $n$th term in this expansion corresponds to the creation of $n$ phonons.

The reciprocal lattice vectors in \eqref{eq:Fin} result from the sum
\begin{align}
\sum_{ij} e^{i(\textbf{q} - \textbf{k}_\nu)\cdot(\textbf{R}_i - \textbf{R}_j)} = (2 \pi)^3 n_i\sum_\textbf{G} \delta^{(3)} (\textbf{q} - \textbf{k}_\nu - \textbf{G})
\end{align}
The sum over $ij$ here will similarly serve to relate the momentum transfer $\textbf{q}$ to the total momentum of the created phonons in each of the higher phonon terms. For simplicity (and because it is a good approximation), we will take the star to be at zero temperature so that $\langle n_\nu + 1 \rangle = 1$ for all phonon modes. Let's see how this plays out in the 1- and 2-phonon terms $F_\text{in}^{(1)}$ and $F_\text{in}^{(2)}$:
\begin{align}
F_\text{in}^{(1)} &= \sum_{ij} e^{i\textbf{q}\cdot(\textbf{R}_i - \textbf{R}_j)} \frac{1}{1!} \frac{1}{(2 MN)} \sum_{\nu_1} \frac{q^2}{\omega_{\nu_1}} e^{-i \textbf{k}_{\nu_1} \cdot(\textbf{R}_i - \textbf{R}_j) + i \omega_{\nu_1} t} \nonumber\\
    &= \frac{1}{1!(2 MN)} \sum_{\nu_1} \frac{q^2}{\omega_{\nu_1}} (2 \pi)^3 n_i \sum_\textbf{G} \delta^{(3)}(\textbf{q} - \textbf{k}_{\nu_1} - \textbf{G}) e^{i \omega_{\nu_1} t} \nonumber\\
    &= \frac{3}{1!(2 M n_i)} \int\frac{d^3 k_{\nu_1}}{(2 \pi)^3} \frac{q^2}{\omega_{\nu_1}} (2 \pi)^3 n_i \sum_\textbf{G} \delta^{(3)}(\textbf{q} - \textbf{k}_{\nu_1} - \textbf{G}) e^{i \omega_{\nu_1} t} \\
    % &= \frac{3n_i}{1!(2 M n_i)} \sum_\textbf{G} \frac{q^2}{\omega_{\nu_1}}  e^{i \omega_{\nu_1} t} \bigg|_{\omega_{\nu_1} = c_s|\textbf{q} - \textbf{G}|} \\
 %
  \nonumber\\
 %
F_\text{in}^{(2)} &= \sum_{ij} e^{i\textbf{q}\cdot(\textbf{R}_i - \textbf{R}_j)} \frac{1}{2!} \frac{1}{(2 MN)^2} \sum_{\nu_1} \sum_{\nu_2} \frac{q^4}{\omega_{\nu_1}\omega_{\nu_2}} \nonumber\\
&\qquad\qquad \times e^{-i (\textbf{k}_{\nu_1} + \textbf{k}_{\nu_2}) \cdot(\textbf{R}_i - \textbf{R}_j) + i (\omega_{\nu_1} + \omega_{\nu_2}) t} \nonumber\\
   &= \frac{1}{2!(2 MN)^2} \sum_{\nu_1} \sum_{\nu_2} \frac{q^4}{\omega_{\nu_1}\omega_{\nu_2}} (2 \pi)^3 n_i\sum_\textbf{G} \delta^{(3)}(\textbf{q} - \textbf{k}_{\nu_1} - \textbf{k}_{\nu_2} - \textbf{G}) e^{i(\omega_{\nu_1} + \omega_{\nu_2})t} \nonumber\\
   &= \frac{3^2}{2!(2 Mn_i)^2} \int\frac{d^3 k_{\nu_1}}{(2 \pi)^3} \int\frac{d^3 k_{\nu_2}}{(2 \pi)^3}  \frac{q^4}{\omega_{\nu_1}\omega_{\nu_2}}  \nonumber\\
   & \qquad\qquad \times (2 \pi)^3 n_i\sum_\textbf{G} \delta^{(3)}(\textbf{q} - \textbf{k}_{\nu_1} - \textbf{k}_{\nu_2} - \textbf{G}) e^{i(\omega_{\nu_1} + \omega_{\nu_2})t}
  %\\  &= \frac{3^2 n_i}{2!(2 Mn_i)^2} \sum_\textbf{G} \int\frac{d^3 k_{\nu_2}}{(2 \pi)^3}  \frac{q^4}{\omega_{\nu_1}\omega_{\nu_2}} e^{i(\omega_{\nu_1} + \omega_{\nu_2})t} \bigg|_{\omega_{\nu_1} = c_s|\textbf{q} - \textbf{k}_{\nu_2} - \textbf{G}|}
\end{align}
At this point we can infer the pattern and write down the result for the $n$-phonon process:
\begin{align}
F_\text{in}^{(n)} &= \frac{3^n (2 \pi)^3 n_i}{n!(2 Mn_i)^n} \sum_\textbf{G} q^{2 n} \prod_{j = 1}^n \left( \int\frac{d^3 k_{\nu_i}}{(2 \pi)^3} \frac{e^{i\omega_{\nu_i} t}}{\omega_{\nu_i}}  \right) \delta^{(3)}\bigg(\textbf{q} - \textbf{G} - \sum_{j = 1}^n \textbf{k}_{\nu_j} \bigg)
%\\ F_\text{in}^{(n)} &= \frac{3^n n_i}{n!(2 Mn_i)^n} \sum_\textbf{G} q^{2 n} \prod_{j = 2}^n \left( \int\frac{d^3 k_{\nu_i}}{(2 \pi)^3} \frac{1}{\omega_{\nu_i}} e^{i\omega_{\nu_i} t} \right) \frac{e^{i\omega_{\nu_1}t}}{\omega_{\nu_1}} \bigg|_{\omega_{\nu_1} = c_s|\textbf{q} - \textbf{G} - \sum_{j = 2}^{n}\textbf{k}_{\nu_i}|}
\end{align}
We will begin by considering only phonons created within the first Brillioun zone, so that we may take $\textbf{G} = 0$. Once we have explored this we may try a non-zero $\textbf{G}$ to see what the effects might be. With this simplification, the intermediate scattering function becomes
\begin{align}
F_\text{in}^{(n)} &= \frac{3^n (2 \pi)^3 n_i}{n!(2 Mn_i)^n} q^{2 n} \prod_{j = 1}^n \left( \int\frac{d^3 k_{\nu_j}}{(2 \pi)^3} \frac{e^{i\omega_{\nu_j} t}}{\omega_{\nu_j}}  \right) \delta^{(3)}\bigg(\textbf{q} - \sum_{j = 1}^n \textbf{k}_{\nu_j} \bigg) \nonumber\\
&= \frac{3^n n_i}{n!(2 Mn_i)^n} q^{2 n} \prod_{j = 2}^n \left( \int\frac{d^3 k_{\nu_j}}{(2 \pi)^3} \frac{e^{i\omega_{\nu_j} t}}{\omega_{\nu_j}}  \right) \frac{e^{i\omega_{\nu_1}t}}{\omega_{\nu_1}} \bigg|_{\omega_{\nu_1} = c_s|\textbf{q} - \sum_{j = 2}^{n}\textbf{k}_{\nu_j}|}
\end{align}

\begin{mdframed}[style=MyFrame]
The assumption that $\textbf{G} = 0$ is probably a bad and overcomplicating one. Various papers (like \href{http://articles.adsabs.harvard.edu/cgi-bin/nph-iarticle_query?1980SvA....24..303Y&defaultprint=YES&filetype=.pdf}{this one}) argue that umklapp processes dominate in a coulomb lattice, such that it is only the distant Brillioun zones that contribute. This seems to be because the typical (maximum?) momentum transfer $ \sim k_F$ is large compared to the size of the Brillioun zone. (This is probably the same for us? Review the arguments that lead to their conclusion.) In these distant zones, one can approximate $\textbf{q} \sim \textbf{G}$, and then we simply need $\cos\theta_{kq} \ll1$ so that $\omega \sim G \cos\theta_{kq}$ is small enough to match a phonon energy. (Recall $\theta_{kq}$ is the angle between the momentum transfer and the incident momentum.)

So perhaps I can integrate over $\textbf{G}$, simply using the delta function to set $\textbf{q} \approx \textbf{G}$. Then the integration over time will match the energy to a phonon energy, and the integral over phonon momentum will \ldots restrict the allowed angles of $\textbf{q}$?

I expect the end result of this to be a larger cross-section and stopping power than the one I first calculated. The energy transfers aren't any larger, but there is a much larger phase space that the incident particle could be scattered into.

These same references say that multi-phonon excitations are only relevant near the melting point. I should review the argument behind this and see whether it still holds for high-energy incident particles.
\end{mdframed}



% $\sum_\textbf{G}\sim\frac{4\pi G^2 dG}{\left(\frac{2\pi}{a}\right)^3}$ for large $|\textbf{q}-\textbf{k}_\nu|$?


\begin{thebibliography}{1}
  \bibitem{quantum} Kittel, \emph{Quantum Theory of Solids}
  \bibitem{dwarf} Baiko, \emph{Ion Structure Factors and Electron Transport in Dense Coulomb Plasmas}
  \bibitem{french} Jancovici, \emph{On the Relativistic Degenerate Electron Gas}
  \bibitem{intro} Kittel, \emph{Introduction to Solid State Physics}
\end{thebibliography}

\end{document}
