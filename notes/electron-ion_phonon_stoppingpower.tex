\documentclass{article}

\usepackage[colorlinks=true]{hyperref}
\usepackage{amsmath}
\usepackage{amssymb}
\usepackage{accents}
\usepackage{parskip}
\usepackage{bm}
\usepackage{xcolor}
\usepackage{mathtools}
\usepackage{wasysym}

%%%%%%
\usepackage[framemethod=TikZ]{mdframed}
\mdfdefinestyle{MyFrame}{%
    linecolor=black,
    outerlinewidth=1pt,
    roundcorner=10pt,
    innertopmargin=\baselineskip,
    innerbottommargin=\baselineskip,
    skipabove=\baselineskip,
    skipbelow=\baselineskip,
    innerrightmargin=20pt,
    innerleftmargin=20pt,
    backgroundcolor=gray!20!white}
\usepackage{xpatch}
\makeatletter
\xpatchcmd{\endmdframed}
  {\aftergroup\endmdf@trivlist\color@endgroup}
  {\endmdf@trivlist\color@endgroup\@doendpe}
  {}{}
\makeatother
%%%%%%

\newcommand{\OO}{\mathcal{O}}
\newcommand{\LL}{\mathcal{L}}
\newcommand{\HH}{\mathcal{H}}
\newcommand{\NN}{\mathcal{N}}
\newcommand{\DD}{\mathcal{D}}
\newcommand{\hc}{\text{ h.c.}}

\newcommand{\angstrom}{\buildrel _{\circ} \over {\mathrm{A}}}
\newcommand{\GeV}{\text{GeV}}
\newcommand{\Hz}{\text{Hz}}
\newcommand{\eV}{\text{eV}}
\newcommand{\MeV}{\text{MeV}}

\newcommand{\ket}[1]{\ensuremath{\left|#1\right>}}
\newcommand{\bra}[1]{\ensuremath{\left<#1\right|}}
\newcommand{\braket}[2]{\ensuremath{\left<#1|#2\right>}}


\title{Electron stopping power via phonon production in the ion lattice}
\author{Paul Riggins}
\begin{document}
\maketitle{}



\section{Goal and Setup}

Our ultimate goal is to compute the stopping power
\begin{align}
\frac{d E}{d x} = \int dE' \frac{d \sigma}{dE'}E'
\end{align}
for an electron scattering off an ion coulomb lattice. Here $E'$ is the energy transfer. For energy transfers below the lattice binding energy, we expect phonon production to become relevant, and the cross-section must take this into account. To capture this behavior we will use the structure factor formalism developed, for instance, in \cite{quantum} Chapter~19.

The double differential cross-section for a particle to scatter into solid angle $\Omega$ while transferring energy $\omega$ and momentum $\textbf{q}$ is given by (\cite{quantum} 19.13)
\begin{align}
\frac{d \sigma}{d \Omega d \omega} = A_\textbf{q} S(\omega, \textbf{q})
\label{eq:DoubleDifferential}
\end{align}
where (\cite{quantum} 19.14)
\begin{align}
A_\textbf{q} = \frac{k'}{k} \left( \frac{m}{2 \pi} \right)^2 |V_\textbf{q}|^2
\end{align}
Here $m$ is the incident particle mass, and $k$ and $k'$ are the magnitudes of the incident particle's initial and final momentum, respectively. These momenta are related to the momentum transfer by $\textbf{k}' = \textbf{k} - \textbf{q}$ (\cite{quantum} 19.3). The fourier transform of the interaction potential is (\cite{quantum} 19.6)
\begin{align}
V_\textbf{q} = \int d^3 x\, e^{i\textbf{q}\cdot \textbf{x}} V(\textbf{x})
\end{align}
For momentum and energy transfers much smaller than the Fermi momentum and energy, we should be able to use a simple coulomb potential for $V(\textbf{x})$. For small transfers, however, we must take into account screening by the Fermi sea, for instance via a dielectric constant (see \cite{dwarf} eq. 9 and \cite{french} 5.67). We will return to this later after we have developed the formalism further.

The structure factor $S(\omega, q)$ is the space-time fourier transform of the density correlation function (\cite{quantum} 19.15-17, 19.40)
\begin{align}
S(\omega, \textbf{q}) &= \frac{1}{2 \pi} \int_{-\infty}^{\infty} dt \, e^{-i\omega t} F(\textbf{q}, t) \\
   &= \frac{1}{2 \pi} \int_{-\infty}^{\infty} dt\, e^{-i \omega t} \sum_{jl} \left\langle e^{-i\textbf{q}\cdot \textbf{x}_j(0)}e^{i\textbf{q}\cdot \textbf{x}_l(t)} \right\rangle_T \\
   &= \frac{1}{2 \pi} \int d^3 x \, d^3 x'\, e^{i \textbf{q}\cdot(\textbf{x} - \textbf{x'})} \int dt \, e^{-i \omega t} \langle \rho(\textbf{x}', 0) \rho(\textbf{x}, t) \rangle_T
\end{align}
where the intermediate scattering function $F(\textbf{q}, t)$ is given by
\begin{align}
F(\textbf{q}, t) &= \sum_{jl} \left\langle e^{-i\textbf{q}\cdot \textbf{x}_j(0)}e^{i\textbf{q}\cdot \textbf{x}_l(t)} \right\rangle_T
\end{align}
and the density operator is
\begin{align}
\rho(\textbf{x}, t) = \sum_j \delta[\textbf{x} - \textbf{x}_j(t)]
\end{align}
The expectation values are taken over the thermal phonon distribution at temperature $T$. In the one-phonon approximation, the elastic and inelastic parts of $F(\textbf{q}, t)$ are given by (\cite{quantum} 19.40, 19.51)
\begin{align}
F_\text{el}(\textbf{q}, t) &= (2 \pi)^3 n_i e^{-2 W} \sum_\textbf{G} \delta^{(3)}(\textbf{q} - \textbf{G}) \\
F_\text{in}(\textbf{q}, t) &= (2 \pi)^3 n_i e^{-2 W} \sum_\nu \frac{q^2}{2 NM \omega_\nu} \Bigg( \langle n_\nu + 1 \rangle e^{i \omega_\nu t} \sum_\textbf{G} \delta^{(3)}(\textbf{q} - \textbf{k}_\nu - \textbf{G}) \nonumber\\  & \hspace{4 cm} + \langle n_\nu \rangle e^{-i \omega_\nu t} \sum_\textbf{G} \delta^{(3)}(\textbf{q} + \textbf{k}_\nu - \textbf{G})  \Bigg)
\label{eq:Fin}
\end{align}
where $N$ is the number of ions in the lattice, $M$ is the ion mass, $n_i$ is the ion number density, $\nu$ indexes the phonon modes of energy $\omega_\nu$ and quasimomentum $\textbf{k}_\nu$, $\textbf{G}$ is a reciprocal lattice factor, and $\langle n_\nu \rangle$ is the expected number of phonons of index $\nu$ at temperature $T$. This last quantity is given by the usual boson statistics:
\begin{align}
\langle n_\nu \rangle = \frac{e^{-\omega_\nu/T}}{1-e^{-\omega_\nu/T}} = \frac{1}{e^{\omega_\nu/T} - 1}
\end{align}
The quantity $W = \frac16 q^2 \langle u_j^2 \rangle$ is the ``Debye-Waller factor'', reflecting decoherence that arises from zero-point and finite temperature effects.

\subsection{Debye approximation}

For simplicity of calculation, we will work in the Debye approximation. In the Debye approximation the Debye-Waller factor is given by (\cite{quantum} 19.56)
\begin{align}
W = \frac16 q^2 \langle u_j^2 \rangle = \frac{3 q^2}{8 M \Theta}\left\{ 1 + \frac{2 \pi^2}{3}\left( \frac{T}{\Theta} \right)^2 + \cdots \right\}
\end{align}
Here $\Theta$ there is the Debye temperature. The Debye approximation describes a phonon distribution of $N$ phonon modes with energies between $ \sim \omega_D / N$ and $\omega_D$, where the maximum photon energy is given by (\cite{intro} 5.21-24)
\begin{align}
\omega_D = c_s(6 \pi^2 n_i)^{1/3}
\label{eq:DebyeEnergy}
\end{align}
and $c_s$ is the sound speed. The Debye temperature is given by $\Theta = \omega_D$ (\cite{intro} 5.28). The dispersion relation for phonons in this approximation is given by
\begin{align}
\omega_\nu = c_sk_\nu
\end{align}
and the density of states is given by
\begin{align}
D(\omega_\nu) = \frac{V\omega_\nu^2}{2 \pi^2 c_s^3}
\end{align}
where $V$ is the lattice volume.

\section{Putting it all together}

Let us now put all this together to calculate the double differential cross-section for inelastic scattering leading to phonon production in the medium. We neglect elastic scattering, since it does not contribute to the stopping power, and we will ignore the possibility of phonon absorption on the assumption that it will be negligible. (We can check this later.)

The phonon production term in $F(\textbf{q}, t)$ is $ \propto \langle n_\nu + 1 \rangle$. Selecting this term, we find the structure factor to be
\begin{align}
S(\omega, \textbf{q}) &= \frac{1}{2 \pi} \int_{-\infty}^\infty dt \, e^{-i \omega t} (2 \pi)^3 n_i e^{-2 W}  \nonumber\\
 &\qquad\qquad  \sum_\nu\frac{q^2}{2 NM \omega_\nu} \langle n_\nu + 1 \rangle e^{i \omega_\nu t} \sum_\textbf{G} \delta^{(3)}(\textbf{q} - \textbf{k}_\nu - \textbf{G})
\end{align}
Evaluating the time integral yields a delta function in energy
\begin{align}
S(\omega, \textbf{q}) &= \frac{1}{2 \pi} (2 \pi)^3 n_i e^{-2 W}  \nonumber\\
&\qquad\qquad  \sum_\nu\frac{q^2}{2 NM \omega_\nu} \langle n_\nu + 1 \rangle \delta(\omega - \omega_\nu) \sum_\textbf{G} \delta^{(3)}(\textbf{q} - \textbf{k}_\nu - \textbf{G})
\label{eq:1PhononStructureFactor}
\end{align}
We know now that we have two delta functions that are not independent of one another. The momentum transfer $\textbf{q}$ and energy transfer $\omega$ are related to each other by the kinematics of the incident particle, and the phonon energy $\omega_\nu$ and crystal momentum $\textbf{k}_\nu$ related to each other by the dispersion relation. Specifically, we have the relations
\begin{align}
\omega &= \sqrt{m^2 + k^2} - \sqrt{m^2 + |\textbf{k} - \textbf{q}|^2} \label{eq:EnergyTransfer} \\
\omega_\nu &= c_sk_\nu \label{eq:DispersionRelation}
\end{align}
where $\textbf{k}$ is the initial momentum of the incident particle of mass $m$. Let's begin by considering a highly relativistic particle with small momentum transfer, so that
\begin{align}
\omega &\approx k - |\textbf{k} - \textbf{q}| \\
   &= k - \sqrt{k^2 + q^2 - 2 kq\cos\theta_{q}} \\
   &\approx k - k \left( 1 + \frac12 \frac{q^2}{k^2} - \frac{q}{k}\cos\theta_{q} \right) \\
   &\approx q\cos\theta_{q} \label{eq:EnergyMomentumRelation}
\end{align}
where $\theta_q$ is the angle of the momentum transfer vector relative to the incident momentum $\textbf{k}$.
To simplify, let us also suppose that we are only interested in momentum transfers within first Brillioun zone $q\lesssim k_D$ and energy transfers $\omega \lesssim \omega_D$. Then we may take only the $\textbf{G} = 0$ term in the reciprocal lattice sum. We may then rewrite the structure factor in terms of momenta as
\begin{align}
S(\omega, \textbf{q}) &= \frac{1}{2 \pi} (2 \pi)^3 n_i e^{-2 W}  \nonumber\\
&\qquad \sum_\nu\frac{q^2}{2 NM (c_sk_\nu)} \langle n_\nu + 1 \rangle \delta(q \cos\theta_{q} - c_sk_\nu) \delta^{(3)}(\textbf{q} - \textbf{k}_\nu)
\end{align}
We can approximate the sum over phonons as an integral over the first Brillioun zone, including a factor of 3 for the various polarizations:
\begin{align}
\sum_\nu \to 3 \int^{k_D} d^3 k_\nu\frac{V}{(2 \pi)^3}  % = 3 \int^{k_D} k_\nu^2 \sin\theta_{k_\nu}\,\frac{V}{(2 \pi)^2}\, dk_\nu \, d \theta_{k_\nu} \, d \phi_{k_\nu}
\end{align}
We use this to collapse the three-dimensional delta function (remember we are assuming $q$ lies within the first Brillioun zone):
\begin{align}
S(\omega, \textbf{q}) &= \frac{1}{2 \pi} (2 \pi)^3 n_i e^{-2 W}  \nonumber\\
&\qquad \left( \frac{3V}{(2 \pi)^3} \right)\frac{q^2}{2 NM (c_sq)} \left[ \frac{1}{e^{c_sq / T} - 1} + 1 \right] \delta(q\cos\theta_{q} - qc_s)
\end{align}
We note that $V / N = n_i^{-1}$. Simplifying this expression, we find
\begin{align}
S(\omega, \textbf{q}) &= \frac{3}{4 \pi M c_s} \left[ \frac{1}{e^{c_sq / T} - 1} +1 \right] \delta(\cos\theta_{q} - c_s)
\label{eq:FirstBrilliounZoneStructureFactor}
\end{align}
Note that we have extracted the momenta factor $q$ from the remaining delta function. We may do this because we are only considering processes with non-zero momentum transfer. We have also neglected the Debye-Waller factor, since it is small and the exponential is $\OO(1)$.

\section{Calculating the stopping power}

In order to find the singly-differential cross-section $d \sigma / d \omega$ and the stopping power, we are about to integrate over $\cos\theta_q$, $\phi_q$, and $\omega$. It will be convenient therefore change variables so our structure factor is in terms of these. Recall that $\omega \approx q \cos\theta_q$ \eqref{eq:EnergyMomentumRelation}. Then the structure factor becomes
\begin{align}
\label{eq:StructureFactorVariableChange}
S(\omega, \cos\theta_q, \phi_q) &= \frac{3}{4 \pi M c_s} \left[ \frac{1}{e^{c_s\omega / T\cos\theta_q} - 1} + 1 \right] \delta(\cos\theta_{q} - c_s)
\end{align}
The singly-differential cross-section is given by (see \eqref{eq:DoubleDifferential} and following)
\begin{align}
\frac{d \sigma}{d \omega} = \int d \Omega\, \frac{k'}{k}\left( \frac{m}{2 \pi} \right)^2 |V_\textbf{q}|^2 S(\omega, \cos\theta_q, \phi_q)
\label{eq:SinglyDifferential}
\end{align}
We must now calculate the interaction potential and the differential $d \Omega$.

\subsection{Interaction potential}

We will use Heaviside-Lorentz units for electromagnetism, so that $\epsilon_0 = 1$. Then the coulomb force is given by $V(\textbf{r}) = e^2 / 4 \pi r = \alpha / r$. Accounting for screening, the fourier transform of this interaction potential is given by \cite{french}
\begin{align}
V_\textbf{q} = \frac{4 \pi Z\alpha}{q^2 \epsilon^l(q, 0)}
\end{align}
where the electron longitudinal static dielectric function relative to the vacuum $\epsilon^l(q, 0)$ is given by
\begin{align}
\epsilon^l(q, 0) = 1 + \frac{4 e^2 E_F \sqrt{E_F^2 + m^2}}{\pi q^2}
\end{align}
so we can write
\begin{align}
V_\textbf{q} = \frac{4 \pi Z\alpha}{q^2 + \lambda^2} = \frac{4 \pi Z\alpha}{(\omega / \cos\theta_q)^2+ \lambda^2}
\label{eq:ScreenedInteractionPotential}
\end{align}
with the screening momentum
\begin{align}
\lambda = \left( \frac{\pi}{4 \alpha E_F \sqrt{E_F^2 + m^2}} \right)^{-1/2} \approx \left( \frac{\pi}{4 \alpha E_F m} \right)^{-1/2}
\end{align}
We cannot ignore the dielectric function because our momentum transfers are coincidentally smaller than the Fermi energy by virtue of being smaller than the lattice binding energy.

\subsection{Solid angle $d \Omega$}
\label{sec:SolidAngle}

We wish to rewrite the solid angle differential $d \Omega = d(\cos\theta_{k'})d \phi_{k'}$ in terms of $d(\cos\theta_q)$ and $d\phi_q$. To begin, we recall that $\textbf{k}' = \textbf{k} - \textbf{q}$. The momentum transfer vector $\textbf{q}$ entirely determines the transverse component of $\textbf{k}'$, so we may relate
\begin{align}
q \sin\theta_q = k' \sin\theta_{k'} \approx k\sin\theta_{k'}
\label{eq:TransverseComponent}
\end{align}
where in the last step we used the fact that $q\ll k$. Manipulating and using the trig identity $\sin\theta = \sqrt{1 - \cos^2 \theta}$, we find
\begin{align}
\cos\theta_{k'} &\approx\sqrt{1 - \left( \frac{q}{k} \right)^2 \sin^2 \theta_q} \label{eq:coskp}\\
  &\approx 1 - \frac12\left( \frac{q}{k} \right)^2 \sin^2 \theta_q \\
  &\approx 1 - \frac12\left( \frac{q}{k} \right)^2 (1 - \cos^2 \theta_q) \\
  &\approx 1 - \frac{\omega^2}{2 k^2} \left( \frac{1}{\cos^2 \theta_q} - 1 \right) \\
\end{align}
Thus we have the differential
\begin{align}
d(\cos\theta_{k'}) \approx \frac{\omega^2}{k^2}\frac{d(\cos\theta_q)}{\cos^3 \theta_q}
\end{align}
The azimuthal angles are simply the same, so the solid angle differential becomes
\begin{align}
d \Omega &= d(\cos\theta_{k'})\,d \phi_{k'} \\
 &\approx d(\cos\theta_q)\,d \phi_q\,\frac{\omega^2}{k^2}\frac{1}{\cos^3 \theta_q}
 \label{eq:SolidAngle}
\end{align}

\subsection{Singly-differential cross-section}

We are now ready to calculate the singly-differential cross-section \eqref{eq:SinglyDifferential}. Approximating $k' \approx k$ and inserting the results \eqref{eq:StructureFactorVariableChange}, \eqref{eq:ScreenedInteractionPotential}, and \eqref{eq:SolidAngle}, we can easily evaluate the integrals (with the help of the delta function) to find
\begin{align}
\frac{d \sigma}{d \omega} &= \int d \Omega\, \left( \frac{m}{2 \pi} \right)^2 |V_\textbf{q}|^2 S(\omega, \cos\theta_q, \phi_q) \\
 &\approx \int \left[ d(\cos\theta_q)\,d \phi_q\,\frac{\omega^2}{k^2}\frac{1}{\cos^3 \theta_q} \right] \left( \frac{m}{2 \pi} \right)^2 \left[ \frac{4 \pi Z\alpha}{(\omega / \cos\theta_q)^2+ \lambda^2} \right]^2 \\
 &\qquad \times\left\{ \frac{3}{4 \pi M c_s} \left[ \frac{1}{e^{c_s\omega / T\cos\theta_q} - 1} + 1 \right] \delta(\cos\theta_{q} - c_s) \right\} \nonumber \\
 &= 2 \pi \left[\frac{\omega^2}{k^2}\frac{1}{c_s^3} \right] \left( \frac{m}{2 \pi} \right)^2 \left[ \frac{4 \pi Z\alpha}{(\omega / c_s)^2+ \lambda^2} \right]^2 \frac{3}{4 \pi M c_s} \left[ \frac{1}{e^{\omega / T} - 1} + 1 \right]
\end{align}
We recall that the speed of sound $c_s \sim 10^{-2}$.

\subsection{Stopping power}

All that remains is to integrate this overall possible energy transfers within the first Brillioun zone to find the stopping power. The result is
\begin{align}
\frac{dE}{d x} &= \int d\omega\, \frac{d \sigma}{d \omega} n_i \omega \\
  &\approx \frac{3 \alpha^2 m^2 n_i Z^2}{k^2 M} \left( \frac{\omega_D^2}{c_s^2 \lambda^2 + \omega_D^2} + \ln \left[ \frac{c_s^2 \lambda^2}{c_s^2 \lambda^2 + \omega_D^2} \right]  \right) \\
  &\sim \frac{3 \alpha^2 m^2 n_i Z^2}{k^2 M} (10)
\end{align}

\section{Beyond the first Brillioun zone}

When we restricted to the first Brillioun zone above, we found a phase space suppression of $ \sim q^2 / k^2$ resulting from the restriction to highly forward scatters. In actuality, however, we can transfer a nearly arbitrary amount of momentum to the let us while still in the transferring a minuscule amount of energy. Any large amount of momentum that we transfer must simply be shifted to within the first Brillioun zone to determine how much energy the corresponding phonon carries. This is the meaning of the sum over reciprocal lattice factors $\sum_\textbf{G}$ that we neglected. Standard white dwarf lore, however, suggests that the dominant contributions to scatters come from distant Brillioun zones, because the typical momentum transfer $k_F > k_D$ is outside the first Brillioun zone, and there is more phase space available for larger momentum transfers. The same is true for us, except that our momentum transfers are $\OO(k)$ instead of $\OO(k_F)$. We will now explore how this modifies our results, sticking for now to the single-phonon approximation.

\subsection{Structure factor}

Before dropping the sum over reciprocal lattice vectors, the one-phonon structure factor is given by
\begin{align}
S(\omega, \textbf{q}) &= \frac{1}{2 \pi} (2 \pi)^3 n_i e^{-2 W}  \nonumber\\
&\qquad\qquad  \sum_\nu\frac{q^2}{2 NM \omega_\nu} \langle n_\nu + 1 \rangle \delta(\omega - \omega_\nu) \sum_\textbf{G} \delta^{(3)}(\textbf{q} - \textbf{k}_\nu - \textbf{G})
\tag{\ref{eq:1PhononStructureFactor}}
\end{align}
In order to compute the sum over $\textbf{G}$, we convert this into an integral:
\begin{align}
\sum_\textbf{G} \to \int \frac{d^3 G}{\left( \frac{2 \pi}{a} \right)^3} = \int \frac{d^3 G}{(2 \pi)^3 n_i}
\end{align}
where $a$ is the lattice spacing. This integral serves to remove the delta function involving $\textbf{G}$, leaving
\begin{align}
S(\omega, \textbf{q}) &= \frac{1}{2 \pi}  \sum_\nu\frac{q^2}{2 NM \omega_\nu} \langle n_\nu + 1 \rangle \delta(\omega - \omega_\nu)
\end{align}
where we have also dropped the Debye-Waller factor because the temperature is small. Also converting the phonon sum to integral and evaluating, we find
\begin{align}
   S(\omega, \textbf{q}) &= \frac{1}{2 \pi}  3 \int^{\textbf{k}_D} d^3 k_\nu \frac{V}{(2 \pi)^3} \frac{q^2}{2 NM \omega_\nu} \langle n_\nu + 1 \rangle \delta(\omega - \omega_\nu) \nonumber\\
   &= \frac{3}{(2 \pi)^4 n_i}  \int^{\textbf{k}_D} (k_\nu^2\, dk_\nu \, d \Omega_\nu) \frac{q^2}{2 M \omega_\nu} \langle n_\nu + 1 \rangle \delta(\omega - \omega_\nu) \nonumber\\
   &= \frac{3(4 \pi)}{(2 \pi)^4 n_i}  \int_0^{\omega_D} \left( \frac{\omega_\nu}{c_s} \right)^2 \frac{d \omega_\nu}{c_s} \frac{q^2}{2 M \omega_\nu} \left[ \frac{1}{e^{\omega_\nu/T} - 1} + 1 \right] \delta(\omega - \omega_\nu) \nonumber\\
   &= \frac{6}{(2 \pi)^3 n_i} \frac{q^2\omega}{2 M c_s^3} \left[ \frac{1}{e^{\omega/T} - 1} + 1 \right] \theta(\omega_D - \omega)
   \label{eq:UmklappStructureFactor}
\end{align}
where in the last step a Heaviside function emerges, enforcing the requirement that $\omega \leq \omega_D$ so that we do not transfer more energy than a single phonon could carry. Note we used the dispersion relation $k_\nu = \omega_\nu / c_s$ to change variables in the integral.

\subsection{Energy transfer, solid angle, interaction potential}

The energy transfer $\omega$ is still given by \eqref{eq:EnergyTransfer}, but we can no longer assume that $q \ll k$. It is still true, however, that $|\textbf{k} - \textbf{q}| \sim |\textbf{k}|$, because the energies of the incident and outgoing particle are only allowed to differ by the energy of a single phonon, which is small compared to $k$. Thus the energy transfer is
\begin{align}
\omega \approx k - |\textbf{k} - \textbf{q}| = k - \sqrt{k^2 + q^2 - 2 kq \cos\theta_q}
\end{align}
or, solving for the momentum transfer in terms of $\omega$ and $\cos\theta_q$, we find
\begin{align}
q &= k \cos\theta_q + \sqrt{\omega(\omega - 2 k) + k^2 \cos^2 \theta_q} \label{eq:LargeMomentumTransfer}\\
  &\approx 2k \cos\theta_q - \omega \sec \theta_q \\
  &\approx 2k \cos\theta_q \label{eq:DistantBrilliounZone1PhononMomentumTransfer}
\end{align}
where in the latter two approximations we assumed $\omega\ll k$. Note that the angle of the momentum transfer relative to the incident momentum is only allowed to range between $0$ and $\cos^{-1}\sqrt{2 \omega / k} \approx \pi / 2$. ($q$ must be real and positive.) This results from the kinematic requirement that the final momentum $k'$ be smaller in magnitude than the incident momentum $k$ by only as much as the energy transfer $\omega$. In the last approximation we have ignored this small energy transfer, allowing the momentum transfer angle to vary all the way up to $\pi / 2$.

In a moment we will also need the solid angle differential corresponding to the direction of the outgoing $\textbf{k}'$ (just as we calculated before in Section~\ref{sec:SolidAngle}). We start with equation~\eqref{eq:coskp}, but this time manipulate without assuming $q\ll k$:
\begin{align}
\cos\theta_{k'} &\approx\sqrt{1 - \left( \frac{q}{k} \right)^2 \sin^2 \theta_q} \tag{\ref{eq:coskp}}\\
  &\approx \sqrt{1 - \left( \frac{2 k \cos\theta_q}{k} \right)^2 \sin^2 \theta_q} \\
  &= \cos 2 \theta_q \\
  &= 2 \cos^2 \theta_q - 1
\end{align}
Thus the solid angle becomes
\begin{align}
d \Omega &= d(\cos\theta_{k'})d \phi_{k'} \\
   &= 4\cos\theta_q\, d(\cos\theta_q)d \phi_q
\end{align}

We will similarly need the interaction potential, which is given by equation~\eqref{eq:ScreenedInteractionPotential} except with $q$ given by \eqref{eq:DistantBrilliounZone1PhononMomentumTransfer}:
\begin{align}
V_\textbf{q} = \frac{4 \pi Z\alpha}{q^2 + \lambda^2} = \frac{4 \pi Z\alpha}{(2 k \cos\theta_q)^2+ \lambda^2}
\end{align}

\subsection{Singly differential cross-section}

The singly differential cross-section \eqref{eq:SinglyDifferential} (approximating $k' \approx k$) is now given by
\begin{align}
\frac{d \sigma}{d \omega} &= \int d \Omega\, \left( \frac{m}{2 \pi} \right)^2 |V_\textbf{q}|^2 S(\omega, \cos\theta_q, \phi_q) \\
   &\approx \int \left[ 4\cos\theta_q\, d(\cos\theta_q)d \phi_q \right] \left( \frac{m}{2 \pi} \right)^2 \left[ \frac{4 \pi Z\alpha}{(2 k \cos\theta_q)^2+ \lambda^2} \right]^2 \nonumber\\
   &\qquad\qquad \left[ \frac{6}{(2 \pi)^3 n_i} \frac{(2 k\cos\theta_q)^2\omega}{2 M c_s^3} \left[ \frac{1}{e^{\omega/T} - 1} + 1 \right] \theta(\omega_D - \omega) \right]
\end{align}
Recall that the only allowed momentum transfer angles are between 0 and $\pi/2$, so the integration over $d(\cos\theta_q)$ ranges from $0$ to $1$. Performing the angular integrals, we find
\begin{align}
\frac{d \sigma}{d \omega} &= \frac{3 m^2 Z^2 \alpha^2 \omega}{2\pi^2 c_s^3 k^2 M n_i} \left[ \ln \left( \frac{4 k^2 + \lambda^2}{\lambda^2} \right) - \frac{4k^2}{4 k^2 + \lambda^2} \right] \left[ \frac{1}{e^{\omega/T} - 1} + 1 \right] \theta(\omega_D - \omega)
\end{align}
The screening momentum $\lambda \sim 10^{-2}~\MeV \ll k$, so this reduces to
\begin{align}
\frac{d \sigma}{d \omega} &= \frac{3 m^2 Z^2 \alpha^2 \omega}{2\pi^2 c_s^3 k^2 M n_i} \ln \left( \frac{4 k^2}{\lambda^2} \right) \left[ \frac{1}{e^{\omega/T} - 1} + 1 \right] \theta(\omega_D - \omega)
\end{align}

\subsection{Stopping power}

We are now ready to compute the stopping power. Before we perform the integral, we note that the Bose enhancement for small $\omega$ goes like $ \sim T / \omega$. This is therefore counterbalanced by the factor of $\omega^2$ in the stopping power integral, and the bose enhancement plays no appreciable role unless $T \sim \omega_D$. (This has been confirmed numerically.) We may thus neglect that term, in which case the stopping power becomes
\begin{align}
\frac{dE}{d x} &= \int d \omega \,\frac{d \sigma}{d \omega} n_i \omega \\
  &\approx \int d \omega \,n_i \omega \left\{ \frac{3 m^2 Z^2 \alpha^2 \omega}{2\pi^2 c_s^3 k^2 M n_i} \ln \left( \frac{4 k^2}{\lambda^2} \right) \left[ 1 \right] \theta(\omega_D - \omega) \right\} \\
   &= \frac{m^2 Z^2 \alpha^2 \omega_D^3}{2 \pi^2 c_s^3 k^2 M} \ln\left( \frac{4 k^2}{\lambda^2} \right)
\end{align}
Inserting the Debye energy~\eqref{eq:DebyeEnergy}, we find the stopping power is
\begin{align}
\frac{d E}{d x} &= \frac{3 m^2 Z^2 \alpha^2 n_i}{k^2 M} \ln\left( \frac{4 k^2}{\lambda^2} \right)
\end{align}
This result should be valid for all incident momenta $k\gg k_D = (6 \pi^2 n_i)^{1/3} \sim 4~\MeV$.

\section{Beyond the one-phonon approximation}

In this section, we will attempt to characterize the contributions beyond the one-phonon approximation. We will draw chiefly on arguments from \cite{impulse} (note there seem to be some differences in convention there compared to \cite{quantum} in the definition of the intermediate scattering function, but this shouldn't affect the final answer).

\subsection{Multiphonon structure factor in the incoherent approximation}

In \cite{impulse}, we find the intermediate scattering function for a single harmonic oscillator given by
\begin{align}
F_0(\textbf{q}, t) = \exp\left\{ \frac{q^2}{2 M \omega_0} G_0(\omega_0, t) \right\}
\end{align}
where
\begin{align}
G_0(\omega_0, t) &= (\cos\omega_0 t - 1) \coth\left( \frac{\omega_0}{2 T} \right) + i \sin\omega_0 t
\end{align}
and we have defined $n(\omega) = 1 / (e^{\omega/T} - 1)$ as the Bose occupation of the mode. This function $G_0$ arises from the creation and annihilation operators corresponding to the emission and absorption of phonons (see, for instance, \cite{quantum} equations (20.41-49), or \cite{interactions} equations (12.93) and following). The time independent term is the Debye-Waller factor, and the first-order expansion in $q^2$ would yield to the one-phonon approximation.

Similarly, we can describe the lattice with density of states $Z(\omega)$ with characteristic frequency $\omega_l$ with the intermediate scattering function
\begin{align}
F_L(\textbf{q}, t) = \exp\left\{ \frac{q^2}{2 M \omega_l} G_L(\omega_l, t) \right\}
\end{align}
where
\begin{align}
G_L(\omega_l, t) &= \int_0^\infty du \left\{ (\cos(u\omega_l t) - 1) \coth\left( \frac{u \omega_l}{2 T} \right) + i \sin(u\omega_l t) \right\} \frac{Z(u)}{u}
\label{eq:LatticeScatteringFunctionExponent}
\end{align}
and we have defined the dimensionless frequency $u = \omega / \omega_l$ and the corresponding dimensionless density of states $Z(u)\to\omega_l Z(\omega)$. Note this result for the lattice is only valid in the ``incoherent approximation'', treating each photon mode as independent. That is, this is what \cite{interactions} would describe as the ``intermediate self-scattering function'', reflecting correlations of each atom in the lattice with itself, and ignoring correlations of each atom in the lattice with others. This would seem a suspect approximation, except it turns out to be mathematically equivalent to the approximation employed in \cite{dwarf} to capture the fact that umklapp processes dominate in a white dwarf coulomb lattice (see \cite{dwarf} equation (7)). We can reproduce the single harmonic oscillator with a density of states $Z(u) = u\delta(u - 1)$ characterized by frequency $\omega_0$, so in what follows here we will always assume are working with the lattice.

At this point \cite{impulse} introduces dimensionless variables for the momentum transfer, temperature, and time
\begin{align}
Q^2 = \frac{q^2}{2 M \omega_l}, \qquad T^* = \frac{T}{\omega_l}, \qquad t' = \omega_l t
\end{align}
Then we can rewrite $G_L(\omega_l, t)$ as
\begin{align}
G(t') &= \int_0^\infty du' \left\{ (\cos u't' - 1) \coth\left( \frac{u'}{2 T^*} \right) + i \sin u't' \right\} \frac{Z(u')}{u'} \nonumber\\
   &= \int_0^\infty du' \left\{ \cos u't' \coth\left( \frac{u'}{2 T^*} \right) + i \sin u't' \right\} \frac{Z(u')}{u'} - 2 W / Q^2 \nonumber\\
    &= \int_{-\infty}^\infty \frac{du'}{u'}Z(u')n(u')e^{-iu't'} - 2 W / Q^2
\end{align}
where the rewrite in the last line employs a mathematical trick by defining $Z(u)$ as even and using the fact that $n(u) - n(-u) = \coth(u/2 T^*)$. Now the positive half of the integration encodes phonon absorption, while the negative half of the integration encodes phonon emission. We have also defined the Debye-Waller factor
\begin{align}
2 W = Q^2 \int_0^\infty du'\, \frac{Z(u')}{u'} \coth\left( \frac{u'}{2 T^*} \right)
\end{align}
Note we are using the integration variable $u'$ simply to distinguish it from the dimensionless energy transfer $u = \omega / \omega_l$ in the structure factor. We can write this structure factor as
\begin{align}
S(Q, u) &= \frac{1}{2 \pi \omega_l} \int_{-\infty}^\infty dt'\, e^{-iut'} e^{Q^2 G(t')} \nonumber\\
   &= \frac{1}{2 \pi \omega_l} e^{-2 W} \int_{-\infty}^\infty dt'\, e^{-iut'} e^{Q^2 g(t')}
   \label{eq:DimensionlessStructureFactor}
\end{align}
where we have defined the non-Debye-Waller piece of $G(t')$ as
\begin{align}
g(t') = \int_{-\infty}^\infty \frac{du'}{u'}Z(u')n(u')e^{-iu't'}
\end{align}

\subsection{Short-time approximation}

For small transfers of energy and momentum, we expect to be dominated by one-phonon processes. For progressively larger transfers, we expect higher phonon processes to become more appreciable. For very large energy or large momentum transfers, we should expect to reject particles out of the lattice rather than simply exciting lattice vibrations. This last expectation gives rise to the so-called ``impulse approximation'', and the structure factor approaches a delta function corresponding to energy and momentum conservation of a free particle. An expression for this is given in \cite{impulse} (see equation (2.12) there), but we will derive it ourselves here.

Inspired by the suggestions in \cite{impulse}, we will derive this approximation by expanding $g(t)$ for short times about $t = 0$. This expansion is
\begin{align}
g(t) = \alpha_0 - it'\alpha_1 - t'^2 \alpha_2 + \OO(t'^3)
\end{align}
where
\begin{align}
\alpha_0 &= \int_{-\infty}^\infty du'\,\frac{1}{u'} Z(u')n(u') \\
\alpha_1 &= \int_{-\infty}^\infty du'\,Z(u')n(u') \\
\alpha_2 &=  \int_{-\infty}^\infty du'\,\frac{u'}{2} Z(u')n(u')
\end{align}
Note that $Q^2\alpha_0 = 2 W$ is the Debye-Waller factor, exchanging $n(u)$ for the $\coth$ as before. In the Debye approximation, where $Z(u) = 3 u^2\, \theta(1-|u|)$, we find
\begin{align}
\alpha_0 &= \int_{-1}^1 du'\, \frac{1}{u'} (3 u'^2) \frac{1}{e^{u'/T^*}-1} \approx \frac32 + \pi^2 (T^*)^2  \quad \text{for} \quad T^*\lesssim 0.1 \label{eq:alpha-0}\\
\alpha_1 &= \int_{-1}^1 du'\, (3 u'^2) \frac{1}{e^{u'/T^*}-1} = -1 \label{eq:alpha-1}\\
\alpha_2 &= \int_{-1}^1 du'\,\frac{u'}{2} (3 u'^2) \frac{1}{e^{u'/T^*}-1} \approx \frac{3}{8} + \frac{\pi^4}{5}(T^*)^4 \quad \text{for} \quad T^*\lesssim 0.1
  \label{eq:alpha-2}
\end{align}
Then the structure factor is
\begin{align}
S(Q, u) &\approx\frac{1}{2 \pi \omega_l} e^{-2 W} \int_{-\infty}^\infty dt'\, e^{-iut'} e^{Q^2 (\alpha_0 - it'\alpha_1 - t'^2 \alpha_2)} \nonumber\\
   &= \frac{1}{2 \pi \omega_l} e^{-2 W} \sqrt{\frac{\pi}{Q^2 \alpha_2}} e^{Q^2 \alpha_0} \exp\left\{ -\frac{(u + Q^2 \alpha_1)^2}{4 Q^2 \alpha_2} \right\} \nonumber\\
   &= \frac{1}{\sqrt{4 \pi \omega_l^2 Q^2 \alpha_2}} \exp\left\{ -\frac{(u - Q^2)^2}{4 Q^2 \alpha_2} \right\} \\
   &= \frac{1}{\sqrt{4 \pi \omega_l E_r \alpha_2}} \exp\left\{ -\frac{(\omega - E_r)^2}{4 \omega_lE_r \alpha_2} \right\}
   \label{eq:ImpulseApproximationStructureFactor}
\end{align}
where in the last line we have rewritten the result in terms of dimensionful variables and the non-relativistic recoil energy $E_r = q^2 / 2 M$. This is the impulse approximation we were seeking, valid for $u \gg 1$. Note that it is a Gaussian centered at $\omega = E_r$ with characteristic width
\begin{align}
\Gamma^2 = 2 \alpha_2\, \omega_l E_r
\end{align}
 It is strongly peaked for energy transfers equal to the non-relativistic kinetic energy of the ejected particle, and we thus expected to be valid for energy transfers less than the mass of the ejected lattice particle (lest we pass into the relativistic regime).

Compare this to the result for a free particle (see \cite{impulse} equation (2.17))
\begin{align}
S(q, \omega) = \frac{1}{\sqrt{4 \pi T E_r}} \exp \left\{ -\frac{(\omega - E_r)^2}{4 TE_r} \right\}
\end{align}
which has width $\Gamma_\text{free}^2 = 2 TE_r$. This approaches a delta function as $T\to 0$, but the lattice result does not. There is always an effective doppler broadening in the lattice (even at zero temperature), presumably because of zero-point fluctuations.

\subsection{Term-by-term phonon contributions}

\subsubsection{One-phonon contribution}

For small momentum transfers, we can expand the structure factor \eqref{eq:DimensionlessStructureFactor} for small $Q$. Doing this, we can evaluate the time integrals to find energy delta functions:
\begin{align}
S(Q, u) &\approx \frac{1}{2 \pi \omega_l} e^{-2 W} \int_{-\infty}^\infty dt'\, e^{-iut'} (1 + Q^2 g(t')) \nonumber\\
   &= \frac{1}{2 \pi \omega_l} e^{-2 W} \int_{-\infty}^\infty dt'\, e^{-iut'} \left( 1 + Q^2 \int_{-1}^1 \frac{du'}{u'}Z(u')n(u')e^{-iu't'} \right) \nonumber\\
   &= \frac{1}{\omega_l} e^{-2 W} \left( \delta(u) + Q^2 \frac{Z(-u)}{-u}n(-u)\,\theta(1-|u|) \right) \nonumber\\
   &= \frac{1}{\omega_l} e^{-2 W} \left( \delta(u) + Q^2 \frac{Z(u)}{u}[n(u) + 1]\,\theta(1-|u|) \right)
\end{align}
The first term is the elastic scattering term, which we will neglect since we are only interested in inelastic scattering. Note we have assumed that the density of states vanishes beyond the characteristic frequency $\omega_l$, so that a single phonon is bounded on the energy it can carry; the integral over $u'$ in $g(t')$ is then restricted to $-1 < u' < 1$, and results in the Heaviside function seen in the result.

How does the one-phonon term compare with our previous results, which we derived from Kittel \cite{quantum} and massaged with our own approximations? Replacing the dimensionless variables with their definitions, we find
\begin{align}
S_\text{1-phonon}(q, \omega) &= \frac{1}{\omega_l} e^{-2 W} Q^2 \frac{Z(u)}{u}[n(u) + 1]\,\theta(\omega_D-|\omega|)  \label{eq:1PhononContribution}\\
   &= \frac{1}{\omega_l} e^{-2 W} \frac{q^2}{2 m \omega_l} \frac{\omega_l Z(\omega)}{\omega/\omega_l}[n(\omega) + 1]\,\theta(\omega_D-|\omega|)
\end{align}
Using the Debye approximation, the dimensionful density of states is $Z(\omega) = 3 \omega^2 / \omega_D^3$ with the Debye energy $\omega_D = c_s(6 \pi^2 n_i)^{1/3}$, so
\begin{align}
S_\text{1-phonon}(q, \omega) &= e^{-2 W} \frac{q^2}{2 m} \frac{3 \omega^2 / \omega_D^3}{\omega}[n(\omega) + 1]\,\theta(\omega_D-|\omega|) \nonumber\\
 &= e^{-2 W} \frac{1}{(2 \pi)^2 n_i} \frac{q^2\omega}{m c_s^3} [n(\omega) + 1]\,\theta(\omega_D-|\omega|)
\end{align}
This is a factor of 2 larger than \eqref{eq:UmklappStructureFactor} that we computed by approximating all of the umklapp contributions. It seems more different from the structure factor \eqref{eq:FirstBrilliounZoneStructureFactor} that we computed for the first Brillioun zone, but there the energy \emph{and} momentum transfers are restricted to be $\lesssim \omega_D$ or $\lesssim k_D$, respectively. Letting $\omega \sim \omega_D$ and $q \sim k_D$ here cancels the factor of density and then the two results are parametrically similar.

\subsubsection{Gaussian approximation for $n$-phonon processes}

Similar to the one-phonon approximation, we can examine higher-order terms in \eqref{eq:DimensionlessStructureFactor} to understand the contributions of higher-order phonon processes. This is the case even when $Q^2 \gtrsim 1$, although in that case every term may have an appreciable contribution to the final result. We will use the Debye approximation throughout. The $n$-phonon process contribution to the structure factor is given by
\begin{align}
S_{n\text{-ph.}}(Q, u) &= \frac{1}{2 \pi \omega_l} e^{-2 W} \int_{-\infty}^\infty dt'\, e^{-iut'} \frac{(Q^2 g(t'))^n}{n!} \nonumber\\
   &= \frac{1}{2 \pi \omega_l} e^{-2 W} \int_{-\infty}^\infty dt'\, e^{-iut'} \frac{Q^{2 n}}{n!} \left( \int_{-1}^1 \frac{du'}{u'}Z(u')n(u')e^{-iu't'} \right)^n \nonumber\\
   &= \frac{1}{2 \pi \omega_l} e^{-2 W} \int_{-\infty}^\infty dt'\, e^{-iut'} \frac{Q^{2 n}}{n!} \prod_{j = 1}^n \left( \int_{-1}^1 \frac{du_j}{u_j}Z(u_j)n(u_j)e^{-iu_jt'} \right)
\end{align}
Evaluating the time integral enforces conservation of energy, requiring that $u + \sum u_j = 0$. Recall that negative values of $u_j$ correspond to phonon emission. We may thus write
\begin{align}
S_{n\text{-ph.}}(Q, u) &= \frac{1}{\omega_l} e^{-2 W} \frac{Q^{2 n}}{n!} \prod_{j = 1}^n \left( \int_{-1}^1 \frac{du_j}{u_j}Z(u_j)n(u_j) \right)\,\bigg|_{u + \sum u_j = 0}
\end{align}
We will now proceed to employ a series of massages and approximations apparently originally due to Sj\"{o}lander (\cite{sj}) and Schofield and Hassitt (\cite{sch1}, \cite{sch2}). The results are allegedly quoted in \cite{impulse} (see equation (2.12) and following) but they don't seem quite right. The methods used to derive those equations are explained in \cite{multiphonon}, so we will reproduce the results on our own here. (The results are also given in \cite{multiphonon}, but there seem to be some typos, most critically in the definition of the Gaussian width $R$.)

We begin by noting that
\begin{align}
\frac{Z(u_j)}{u_j} n(u_j) = \frac{Z(u_j)}{u_j} \frac{1}{e^{u_j/T^*} - 1} = \frac{Z(u_j)}{u_j} \frac{e^{-u_j/2T^*}}{2\sinh(u_j/2T^*)}
\end{align}
Inserting this, we can then pull out the exponentials in the combined integrands thanks to the condition $u + \sum u_j = 0$. The result is
\begin{align}
S_{n\text{-ph.}}(Q, u) &= \frac{1}{\omega_l} e^{-2 W} \frac{Q^{2 n}}{n!} e^{u/2T^*} \prod_{j = 1}^n \left( \int_{-1}^1 du_j\, \frac{Z(u_j)}{u_j} \frac{1}{2\sinh(u_j/2T^*)} \right)\,\bigg|_{u + \sum u_j = 0}
\end{align}
We now rewrite the integrand as
\begin{align}
G(u_j, T^*) = \frac{1}{2 \beta} \frac{Z(u_j)}{u_j\sinh(u_j / 2 T^*)}
\end{align}
with the normalization constant
\begin{align}
\beta = \int_0^1 du_j\, \frac{Z(u_j)}{u_j\sinh(u_j / 2 T^*)} \approx 3 \pi^2 T^2 \quad\text{for}\quad T^*\lesssim 0.1
\end{align}
chosen so that each integral $\int_{-1}^1 du_j\, G(u_j, T^*) = 1$. The structure factor written this way is
\begin{align}
S_{n\text{-ph.}}(Q, u) &= \frac{1}{\omega_l} e^{-2 W} \frac{Q^{2 n}}{n!} e^{u/2T^*} \beta^n \prod_{j = 1}^n \left( \int_{-1}^1 du_j\, G(u_j, T^*) \right)\,\bigg|_{u + \sum u_j = 0}
\end{align}
We now make two critical approximations that allow us to simplify this result. First, we suppose that the function $G(u_j, T)$ can be approximated by a Gaussian:
\begin{align}
G(u_j, T^*) &\approx \frac{1}{\sqrt{2 \pi R^2}} e^{-u_j^2 / 2 R^2} \quad\text{with}\quad R = \frac{\beta}{3\sqrt{2 \pi}\, T^*}
\end{align}
where we have chosen $R$ to match the value of $G(u_j, T^*)$ at $u_j = 0$. Checking in Mathematica, this seems to be a good approximation to the form of $G$ for a wide range of $T^*$, small and large. However, it drops off too rapidly for small temperatures, and will lead to a vanishing of multi-phonon processes in the final result for $T^*\to 0$.

Secondly, we extend the limits of integration for each $u_j$ out to infinity. This seems to be a good approximation at least for $T^*\lesssim 0.1$, since then the Gaussian is negligible for $|u_j|\gtrsim1$. With these two approximations in hand, we can now use the following useful fact of Gaussian integrals to simplify the structure factor:
\begin{align}
\prod_{j = 1}^n \left( \int_{-\infty}^\infty \frac{1}{\sqrt{2 \pi R^2}} e^{-u_j^2 / 2 R^2} \right) \delta\left(u + \sum_{i = 1}^n u_i \right) = \frac{1}{\sqrt{2 \pi n R^2}} e^{-u^2 / 2 nR^2}
\end{align}
This is described in the various articles mentioned earlier as a central limit theorem result, but can also simply be checked in Mathematica.

Finally, we find the $n$-phonon contribution to the structure factor is given approximately by
\begin{align}
S_{n\text{-ph.}}(Q, u) &\approx \frac{1}{\omega_l} e^{-2 W} \frac{Q^{2 n}}{n!} e^{u/2T^*} \beta^n \frac{1}{\sqrt{2 \pi n R^2}} e^{-u^2 / 2 nR^2}
\label{eq:MultiPhononContribution}
\end{align}
In principle this should also come with a Heaviside function $\theta(n - |u|)$ just like the one-phonon contribution, but we find empirically that the $e^{-u^2 / 2 nR^2}$ term crushes the $n$-phonon contribution well before $u \sim n$.

Unfortunately, we see in this result very undesirable behavior for $T^*\to 0$: the Gaussian width $R\to 0$, forcing the contribution of all $n\geq 2$ processes to zero. This is certainly unphysical. It arises because we extracted an exponential early on from the integrand, which otherwise would have balanced against the exact integrand would give rise to the $n(u) + 1$ that makes it possible for phonons to be created even at zero temperature.

% \subsubsection{Total inelastic structure factor}
%
% We have now calculated the contributions from all available phonon processes in the incoherent approximation. We now sum these contributions to find the total inelastic structure factor, and consider the result. Combining equations \eqref{eq:1PhononContribution} and \eqref{eq:MultiPhononContribution}, we find
% \begin{align}
% S_\text{in}(Q, u) &\approx \frac{1}{\omega_l} e^{-2 W} Q^2 \frac{Z(u)}{u}[n(u) + 1]\,\theta(1-|u|) \nonumber\\ &\qquad + \sum_{n = 2}^\infty \left( \frac{1}{\omega_l} e^{-2 W} \frac{Q^{2 n}}{n!} e^{u/2T^*} \beta^n \frac{1}{\sqrt{2 \pi n R^2}} e^{-u^2 / 2 nR^2} \right) \\
%    &=
% \end{align}


\subsection{Computing singly-differential cross-sections and stopping powers}

\subsubsection{Impulse approximation}

Let us compute the singly-differential cross-section and subsequently the stopping power for the impulse approximation. We recall the structure factor is given by
\begin{align}
S(q, \omega) &\approx \frac{1}{\sqrt{4 \pi \omega_l E_r \alpha_2}} \exp\left\{ -\frac{(\omega - E_r)^2}{4 \omega_lE_r \alpha_2} \right\} \tag{\ref{eq:ImpulseApproximationStructureFactor}}
\end{align}
where $E_r = q^2 / 2 M$ is the non-relativistic recoil energy and $\alpha_2 \approx (3/8) + (\pi^4/5)(T^*)^4$ for $T^*\lesssim 0.1$ \eqref{eq:alpha-2}. This is valid for energy transfers $\omega \gg \omega_l = \omega_D$. An incident electron scattering of the lattice will be relativistic both before and after the scatter, because it cannot dip below the surface of the Fermi sea. Nevertheless, its incident energy may change by $\OO(1)$ in a high-energy scatter of this sort. In terms of the initial and final momenta and energy transfer, the momentum transfer is then
\begin{align}
q^2 &= |\textbf{k} - \textbf{k}'|^2 = k^2 + k'^2 - 2 kk'\cos\theta_{k'} \nonumber\\
  &= k^2 + (k - \omega)^2 - 2 k(k - \omega)\cos\theta_{k'} \label{eq:RelativisticMomentumTransfer}
\end{align}
We can then change variables to the recoil energy, and write
\begin{align}
q^2 &= 2 ME_r \\
d(\cos\theta_{k'}) &= \frac{M}{-k(k - \omega)}\,dE_r
\end{align}
where $E_r$ ranges from $(2 k - \omega)^2 / 2 M$ to $\omega^2 / 2 M$ correspondent to $d(\cos\theta_{k'})$ ranging from $-1$ to $1$.
Writing the incident particle mass as $m$, the singly-differential cross-section for a Fermi-screened Coulomb scatter is then given by
\begin{align}
  \frac{d \sigma}{d \omega} &= \int d \Omega\, \frac{k'}{k} \left( \frac{m}{2 \pi} \right)^2 |V_\textbf{q}|^2 S(q, \omega)  \nonumber\\
     &\approx \int d(\cos\theta_{k'})\, d \phi_{k'}\, \frac{k'}{k} \left( \frac{m}{2 \pi} \right)^2 \left[ \frac{4 \pi Z\alpha}{q^2+ \lambda^2} \right]^2 \frac{1}{\sqrt{4 \pi \omega_l E_r \alpha_2}} e^{-(\omega - E_r)^2/4 \omega_lE_r \alpha_2} \nonumber\\
     &= 2 \pi \int_{\frac{(2 k - \omega)^2}{2 M}}^{\frac{\omega^2}{2 M}} \frac{M\, dE_r}{-k(k - \omega)} \frac{k - \omega}{k} \left( \frac{m}{2 \pi} \right)^2 \left[ \frac{4 \pi Z\alpha}{2 M E_r + \lambda^2} \right]^2 \frac{e^{-(\omega - E_r)^2/4 \omega_DE_r \alpha_2}}{\sqrt{4 \pi \omega_D E_r \alpha_2}}  \nonumber\\
\end{align}
Mathematica cannot handle this integral, so let us first non-dimensionalize and then numerically compute what we need to. We define the dimensionless variables $Q^2, u, \eta, \kappa$ as follows:
\begin{align}
E_r = Q^2 \omega_D \qquad \omega = u\, \omega_D \qquad \lambda^2 = \eta^2 (2 M \omega_D) \qquad k = \kappa\, \omega_D
\end{align}
Note $\kappa \gg 1$ because the incoming particle is energetic, $u\leq \kappa$ because of kinematics, $Q^2 \ll M / \omega_D \sim 10^6$ in order to stay non-relativistic, and $u\gg 1$ in order for the impulse approximation to be valid. In the white dwarf, $\eta^2 \sim 10^{-4}$. Then the cross-section becomes
\begin{align}
  \frac{d \sigma}{d \omega} &= 2 \pi \frac{M}{-k^2} \left( \frac{m}{2 \pi} \right)^2 \left( 4 \pi Z\alpha \right)^2 \int_{\frac{(2 k - \omega)^2}{2 M}}^{\frac{\omega^2}{2 M}} dE_r\, \left[ \frac{1}{2 M E_r + \lambda^2} \right]^2 \frac{e^{-(\omega - E_r)^2/4 \omega_DE_r \alpha_2}}{\sqrt{4 \pi \omega_D E_r \alpha_2}}  \nonumber\\
   &= 2 \pi \frac{M}{-k^2} \left( \frac{m}{2 \pi} \right)^2 \left( 4 \pi Z\alpha \right)^2 \int_{\frac{\omega_D}{2M}(2 \kappa - u)^2}^{\frac{\omega_D}{2M}u^2} d(Q^2)\, \frac{\omega_D}{(2 M\omega_D)^2} \left[ \frac{1}{Q^2 + \eta^2} \right]^2 \frac{e^{-(u - Q^2)^2/4 Q^2 \alpha_2}}{\omega_D \sqrt{4 \pi Q^2 \alpha_2}}  \nonumber\\
   &= \frac{2 \pi m^2 Z^2 \alpha^2}{Mk^2\omega_D^2} \int^{\frac{\omega_D}{2M}(2 \kappa - u)^2}_{\frac{\omega_D}{2M}u^2} d(Q^2)\, \left[ \frac{1}{Q^2 + \eta^2} \right]^2 \frac{e^{-(u - Q^2)^2/4 Q^2 \alpha_2}}{\sqrt{4 \pi Q^2 \alpha_2}}
\end{align}
We could numerically integrate this, but even doing so we find that it is a good approximation to simply treat the Gaussian as a delta function, so we find
\begin{align}
  \frac{d \sigma}{d \omega} &\approx \frac{2 \pi m^2 Z^2 \alpha^2}{Mk^2\omega_D^2} \frac{1}{(u + \eta^2)^2} \approx \frac{2 \pi m^2 Z^2 \alpha^2}{Mk^2\omega^2}
\end{align}
This is off from the result Vijay calculated for the appendix by a factor of $m^2 / k^2$, so it seems we must've made a mistake somewhere. Maybe there was something wrong with the change of variables?

\subsubsection{Free particle}

The previous subsection discovered a wrong result, but I cannot locate the error. Let us try with the structure factor for a free particle to try and narrow it down. The free particle structure factor is given by
\begin{align}
S(q, \omega) &= \delta\left( \omega - E_r \right)
\end{align}
where $E_r = q^2 / 2 M$ is the non-relativistic recoil energy. This is valid for energy transfers $\omega \ll M$. Suppose the incident electron is relativistic both before and after the collision. The momentum transfer is then
\begin{align}
q^2 &= |\textbf{k} - \textbf{k}'|^2 = k^2 + k'^2 - 2 kk'\cos\theta_{k'} \nonumber\\
  &= k^2 + (k - \omega)^2 - 2 k(k - \omega)\cos\theta_{k'} \label{eq:RelativisticMomentumTransfer}
\end{align}
We can then change variables to the recoil energy, and write
\begin{align}
q^2 &= 2 ME_r \\
d(\cos\theta_{k'}) &= \frac{M}{-k(k - \omega)}\,dE_r
\end{align}
where $E_r$ ranges from $(2 k - \omega)^2 / 2 M$ to $\omega^2 / 2 M$ correspondent to $d(\cos\theta_{k'})$ ranging from $-1$ to $1$.
Writing the incident particle mass as $m$, the singly-differential cross-section for an unscreened Coulomb scatter is then given by
\begin{align}
  \frac{d \sigma}{d \omega} &= \int d \Omega\, \frac{k'}{k} \left( \frac{m}{2 \pi} \right)^2 |V_\textbf{q}|^2 S(q, \omega)  \nonumber\\
     &\approx \int d(\cos\theta_{k'})\, d \phi_{k'}\, \frac{k'}{k} \left( \frac{m}{2 \pi} \right)^2 \left[ \frac{4 \pi Z\alpha}{q^2} \right]^2 \delta(\omega - E_r) \nonumber\\
     &= 2 \pi \int_{\frac{(2 k - \omega)^2}{2 M}}^{\frac{\omega^2}{2 M}} \frac{M\, dE_r}{-k(k - \omega)} \frac{k - \omega}{k} \left( \frac{m}{2 \pi} \right)^2 \left[ \frac{4 \pi Z\alpha}{2 M E_r} \right]^2 \delta(\omega - E_r) \nonumber\\
     &= 2 \pi \int^{\frac{(2 k - \omega)^2}{2 M}}_{\frac{\omega^2}{2 M}} \frac{M\, dE_r}{k^2} \left( \frac{m}{2 \pi} \right)^2 \left[ \frac{4 \pi Z\alpha}{2 M E_r} \right]^2 \delta(\omega - E_r) \nonumber\\
     &= 2 \pi  \frac{M}{k^2} \left( \frac{m}{2 \pi} \right)^2 \left[ \frac{4 \pi Z\alpha}{2 M \omega} \right]^2 \nonumber\\
     &= \frac{2 \pi m^2 Z^2 \alpha^2}{k^2 M\omega^2}
\end{align}
Once again this is off from the result Vijay calculated for the appendix by a factor of $m^2 / k^2$, so it seems we must've made a mistake somewhere. It is important we figure this out, because the $1/k^2$ behavior seems to appear generically in how we are evaluating these integrals, including for the harmonic oscillator approximation in the next section.

Could it be that there is a difference between
\begin{align}
\frac{d \sigma}{d \omega} \qquad\text{and}\qquad \frac{d \sigma}{d E'}
\end{align}
where $\omega$ is the energy transfer and $E' = k'$ is the final energy, and there is some density of states or something that transforms between them?


\section{Single harmonic oscillator approximation}

Surjeet is adamant that we ought to be able to simply use the ``$q^2 / m \omega$'' correction to the cross-section and move on. I think we can clarify and formalize this intuition by approximating the phonon spectrum available in the lattice by a single frequency. That is, consider exciting phonons of the lattice as exciting energy states in a single harmonic oscillator, rather than in the collection of harmonic oscillators that is the photon spectrum. As commented after \eqref{eq:LatticeScatteringFunctionExponent}, this is tantamount to approximating the true density of states---e.g., $Z(\omega) = 3 \omega^2 / \omega_D^3$ for the Debye approximation---by a delta function with characteristic frequency $\omega_0$: $Z(\omega) = (\omega / \omega_0) \delta(\omega - \omega_0)$. This allows us to avoid all of the complicated integrals over the phonon spectrum, which simplifies our lives dramatically.

We will begin here by repeating some results derived in \cite{interactions}, Section~12.2. The intermediate scattering function for a single harmonic oscillator is given by
\begin{align}
F(q, t) &= e^{-q^2 \langle X^2 \rangle} e^{q^2 \langle X(0)X(t) \rangle}
\end{align}
were the first term is the Debye-Waller factor, and the second captures the inelastic scattering. Inserting the position operator at finite temperature and manipulating, one finds
\begin{align}
\langle X^2 \rangle &= \frac{1}{2 M \omega_0} \coth(z) \\
\langle X(0)X(t) \rangle &= \frac{1}{2 M \omega_0} \left( n(\omega_0)e^{-i \omega_0 t} + (n(\omega_0) + 1)e^{i \omega_0 t} \right) \nonumber\\
  &= \frac{1}{2 M \omega_0} \left( \frac{e^z e^{i \omega_0 t} + e^{-z} e^{-i \omega_0 t}}{e^z - e^{-z}} \right)
\end{align}
where $z = \omega_0 / 2 T$. For small temperatures, where $z\gg 1$, these reduce to
\begin{align}
\langle X^2 \rangle &= \frac{1}{2 M \omega_0} \\
\langle X(0)X(t) \rangle &= \frac{1}{2 M \omega_0} e^{i \omega_0 t}
\end{align}
Then the dynamic structure factor is
\begin{align}
S(q, \omega) &= \frac{1}{2 \pi} \int_{-\infty}^\infty dt \, e^{-i \omega t} F(q, t) \nonumber\\
   &= \frac{1}{2 \pi} \int_{-\infty}^\infty dt \, e^{-i \omega t} e^{-q^2 / 2 M \omega_0} \left[ 1 + \frac{q^2}{2 M \omega_0}e^{i \omega_0t} + \frac12 \left( \frac{q^2}{2 M \omega_0} \right)^2 e^{i2 \omega_0 t} + \cdots \right] \nonumber\\
   &= e^{-q^2 / 2 M \omega_0} \left[ \delta(\omega) + \frac{q^2}{2 M \omega_0}\delta(\omega - \omega_0) + \frac12 \left( \frac{q^2}{2 M \omega_0} \right)^2 \delta(\omega - 2\omega_0) + \cdots \right] \nonumber\\
\end{align}
For the inelastic part of this, simply neglect the $\delta(\omega)$ term.

It now remains to computes the cross-section and stopping power. The singly-differential cross-section for the $n$th process is given by
\begin{align}
\frac{d \sigma_n}{d \omega} &= \int d \Omega\, \frac{k'}{k} \left( \frac{m}{2 \pi} \right)^2 |V_\textbf{q}|^2 S(q, \omega)  \\
&= \int d(\cos\theta_{k'})\, d \phi_{k'}\, \frac{k'}{k} \left( \frac{m}{2 \pi} \right)^2 \left[ \frac{4 \pi Z\alpha}{q^2 + \lambda^2} \right]^2 e^{-q^2 / 2 M \omega_0} \frac{1}{n!}\left( \frac{q^2}{2 M \omega_0} \right)^n \delta(\omega - n \omega_0) \nonumber
\end{align}
The momentum transfer is related to the solid angle by
\begin{align}
q^2 &= k^2 + (k - \omega)^2 - 2 k(k - \omega)\cos\theta_{k'} \tag{\ref{eq:RelativisticMomentumTransfer}}
\end{align}
Let us define dimensionless variables like we did in the last section:
\begin{align}
Q^2 = \frac{q^2}{2 M \omega_0} \qquad u = \omega / \omega_0 \qquad \kappa = k / \omega_0 \qquad \eta^2 = \frac{\lambda^2}{2 M \omega_0}
\end{align}
Then from the momentum transfer we find
\begin{align}
Q^2 &= \frac{\omega_0}{2 M}\left( \kappa^2 + (\kappa - u)^2 - 2 \kappa(\kappa - u)\cos\theta_{k'} \right) \\
d(Q^2) &= -\frac{\omega_0}{M} \kappa(\kappa - u)\, d(\cos\theta_{k'})
\end{align}
where $Q^2$ ranges from $Q_2^2 \equiv (2 \kappa - u)^2 \omega_0 / 2 M$ to $Q_1^2 \equiv u^2 \omega_0 / 2 M$ correspondent to $d(\cos\theta_{k'})$ ranging from $-1$ to $1$. As in the previous section, note $\kappa \gg 1$ because the incoming particle is energetic, $u\leq \kappa$ because of kinematics, $Q^2 \ll M / \omega_0 \sim 10^6$ in order to stay non-relativistic, and $\eta^2 \sim 10^{-4}$ with typical white dwarf parameters.

The singly-differential cross-section now becomes
\begin{align}
\frac{d \sigma_n}{d \omega} &= 2 \pi\int_{Q_1^2}^{Q_2^2} \frac{M\,d(Q^2)}{\omega_0 \kappa(\kappa - u)}\, \frac{(\kappa - u)}{\kappa} \left( \frac{m}{2 \pi} \right)^2 \left[ \frac{4 \pi Z\alpha}{(2 M \omega_0)(Q^2 + \eta^2)} \right]^2 e^{-Q^2} \frac{Q^{2 n}}{n!} \delta(\omega - n \omega_0) \nonumber\\
   &= \frac{2 \pi Z^2 \alpha^2 m^2}{Mk^2 \omega_0^2} \delta(u - n) \int_{Q_1^2}^{Q_2^2} d(Q^2)\, \frac{e^{-Q^2}}{(Q^2 + \eta^2)^2} \frac{(Q^2)^n}{n!}
\end{align}


\begin{thebibliography}{1}
  \bibitem{quantum} Kittel, \emph{Quantum Theory of Solids}
  \bibitem{dwarf} Baiko, \emph{Ion Structure Factors and Electron Transport in Dense Coulomb Plasmas}
  \bibitem{french} Jancovici, \emph{On the Relativistic Degenerate Electron Gas}
  \bibitem{intro} Kittel, \emph{Introduction to Solid State Physics}
  \bibitem{impulse} Gunn, \emph{The Effect of High Momentum Transfer on Scattering from Oscillators and Crystals}
  \bibitem{interactions} Chen, \emph{Interactions of Photons and Neutrons with Matter}
  \bibitem{multiphonon} Kothari and Sinkwi, \emph{Interaction of Thermal Neutrons with Solids}
  \bibitem{sj} Sj\"{o}lander, A.: Arkiv Fysik \textbf{14}, 315 (1958) (from \cite{impulse} and \cite{multiphonon})
  \bibitem{sch1} P. Schofield and A. Hassitt, \emph{Proc. 2nd Intern. Conf. on Peaceful Uses of Atomic Energy, Geneva} P 18 (1958). (from \cite{multiphonon})
  \bibitem{sch2} Schofield, P., Hassitt, A.: Progress in nuclear energy. Ser. I, Vol. 3, p dots in 194. London: Pergamon Press 1959 (from \cite{impulse})


\end{thebibliography}

\end{document}
