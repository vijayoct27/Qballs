Here we provide a more detailed analysis of the stopping power (energy loss per distance traveled) of high-energy SM particles in a carbon-oxygen WD due to electromagnetic and strong interactions.
We consider incident electrons, photons, pions, and nucleons with kinetic energy greater than an $\MeV$.

\subsection{WD Medium}
For the WD masses that we consider, the stellar medium consists of electrons and fully-ionized carbon nuclei with densities in the range $n_e = Z n_\ion \sim 10^{31} - 10^{33} ~\cm^{-3}$ where $Z=6$.
The internal temperature is $T \sim \keV$~\cite{KippenhahnWeigert}.
The electrons are degenerate and predominantly relativistic free gas, with Fermi energy
\begin{equation}
  E_F \sim (3 \pi^2 n_e)^{1/3} \sim 1 -10 ~\MeV.
\end{equation}
The carbon ions, however, are non-degenerate and do not form a free gas. 
The plasma frequency due to ion-ion Coulomb interactions is given by
\begin{align}
\Omega_p = \l \frac{4 \pi n_\ion Z^2 \alpha}{m_\ion}\r^{1/2} \sim 1 - 10~\keV,
\end{align}
where $m_\ion$ is the ion mass.
Finally, the medium also contains thermal photons, though these are never significant for stopping particles as the photon number density $n_\gamma \sim T^3$ is much smaller than that of electrons or ions.

\subsection{Coulomb Scattering}
\label{sec:coulomb}

\paragraph{Scattering off Carbon Ions.}
\label{sec:coulomb_ion}
Coulomb collisions with ions provide the dominant mechanism by which electrons with energy $1~\MeV \lesssim E \lesssim 10~\MeV$ thermalize ions.
In this scenario we may treat the ions as stationary and ignore their recoil during collisions.
The ionic charge will be screened by the mobile electrons of the medium, so incident particles will scatter via a potential
\begin{align}
  \label{eq:ScreenedPotential}
V(\textbf{r}) = \frac{Z \alpha}{r} e^{-\lambda_\TF r}.
\end{align}
The screening length $\lambda_\TF$ is given in the Thomas-Fermi approximation by \cite{Teukolsky}
\begin{align}
\label{eq:TF}
    \lambda_\TF^{2} = \frac{E_F}{6 \pi \alpha n_e} 
    \sim \frac{1}{\alpha n_e^{2/3}}
\end{align}
where $E_F$ is the electron Fermi energy.
This plasma screening suppresses scatters with momentum transfers below $\sim \lambda_\TF^{-1}$, corresponding to a minimal energy transfer of $\omega_\xmin = \lambda_\TF^{-2} / 2 m_\ion$.
Ions may in principle also cause screening through lattice distortion, however this may be ignored as the sound speed of the lattice $c_s \sim 10^{-2}$ is much smaller than the speed of an incident relativistic electron. 
Using the Born approximation, we have a cross-section for energy transfer $\omega$
\begin{align}
\label{eq:CoulombOffIonsCrossSection}
  \frac{d \sigma}{d \omega} = 
  \frac{2 \pi Z^2 \alpha^2}{m_\ion\beta^2} 
  \frac{1}{(\omega + \omega_\xmin)^2}
\end{align}
and a stopping power 
\begin{align}
  \frac{dE}{d x} &= \int_{0}^{\omega_\xmax} d \omega \, n_\ion 
  \frac{d \sigma}{d \omega} \omega \nonumber\\
  \label{eq:StoppingPowerOffIons}
   &\approx \frac{2 \pi\, n_\ion Z^2 \alpha^2 }{m_\ion\beta^2} 
   \log\left( \frac{\omega_\xmax}{\omega_\xmin} \right)
\end{align}
where the second line is valid if $\omega_\xmax \gg \omega_\xmin$.
$\omega_\xmax$ is the maximum possible energy transfer. 
This may be due to 4-momentum conservation, or in the case of incident electrons, the impossibility of scattering to a final energy less than $E_F$. 
4-momentum conservation sets an upper bound $\omega_\kin$, which for a stationary target is
\begin{align}
  \omega_\kin &= \frac{2 m_\ion p^2}{m_\ion^2 + m^2 + 2E m_\ion}
\end{align}
with $p$, $E$ the incoming momentum and energy. 
The Fermi upper bound is simply $\omega_F = E - E_F$ and for incident electrons $\omega_\xmax = \min\l\omega_\kin, \omega_F\r$.

For scatters that transfer energy less than the plasma frequency $\Omega_p$, one may be concerned about phonon excitations.
We estimate this by treating each ion as an independent oscillator with frequency $\Omega_p$ (an Einstein solid) and compute the stopping power due to scatters which excite a single oscillator quanta. 
There are two key differences between this and the free ion case: incident particles must transfer an energy $\Omega_p$ and the cross-section to transfer momentum $q$ is suppressed by a factor $q^2 / 2 m_\ion\Omega_p = \omega_\x{free}/\Omega_p$. 
$\omega_\x{free}$ is the energy transfer that would accompany a free ion scatter with momentum transfer $q$. 
The resulting stopping power is unchanged from the free case~\eqref{eq:StoppingPowerOffIons}, as the increased energy transfer compensates for the suppressed cross-section:
\begin{align}
  d\sigma \cdot \omega \sim 
  d\sigma_\x{free} \frac{\omega_\x{free}}{\Omega_p} 
  \cdot \Omega_p \sim 
  d\sigma_\x{free} \cdot \omega_\x{free}.
\end{align}

Finally, we note that for more massive or more energetic incident particles the cross-section~\eqref{eq:CoulombOffIonsCrossSection} must be replaced with a more complicated expression to account for the recoil of the ion during collisions. 
In these scenarios, however, this stopping power is far subdominant to hadronic or electromagnetic showers. \\

\paragraph{Scattering off Degenerate Electrons.}
\label{sec:coulomb_elec}
The scattering of incident electrons with the degenerate electron sea determines the termination energy of electromagnetic showers, and so we focus here on that scenario. 
This calculation demands two considerations not present when scattering off ions: the targets are not stationary and they require a threshold energy transfer in order to be scattered out of the Fermi sea.
In the scenario of interest, however, these do not result in a parametrically different stopping power than was found for ions in equation~\eqref{eq:StoppingPowerOffIons}. 

For incident momenta much greater than the Fermi momentum, the relative velocity is of order the incident velocity and the deflection of the incident particle will generally be small. 
It is reasonable then that scattering proceeds, up to $\OO(1)$ factors, as though a heavy incident particle is striking a light, stationary target.  
The cross-section is then given by the usual result, 
\begin{align}
  \frac{d \sigma}{d \omega} \approx
  \frac{2 \pi \alpha^2}{E_F} \frac{1}{\omega^2},
  \label{eq:CoulombRelativisticApprox}
\end{align}
where we have accounted for the target's motion by replacing its mass with the relativistic inertia $\sqrt{m_e^2 + p^2} \approx E_F$.  We have ignored plasma screening, as Pauli-blocking will provide a more stringent cutoff on soft scatters in this case. 
Scatters which transfer an energy $\omega \leq E_F$ will have a suppressed contribution to the stopping power as they can only access a fraction of the Fermi sea. 
For incident energies $E \gg E_F$ it is sufficient to ignore these suppressed scatters, i.e.,
\begin{align}
  \frac{dE}{d x} &= \int_{E_F}^{\omega_\xmax} d \omega \, n_e 
  \frac{d \sigma}{d \omega} \omega \nonumber\\
  \label{eq:StoppingPowerOffElectrons}
   &\approx \frac{2 \pi\, n_e \alpha^2 }{E_F} 
   \log\left( \frac{\omega_\xmax}{E_F} \right)
\end{align}
where, as above, $\omega_\xmax = \min\l\omega_\kin, \omega_F\r$.
This derivation is admittedly quite heuristic, and so it has been checked with a detailed numerical calculation accounting fully for the target's motion and degeneracy.
Equation~\eqref{eq:StoppingPowerOffElectrons} is indeed a good approximation to the stopping power for incient energies larger than the Fermi energy. 

\subsection{Compton Scattering}
\label{sec:compton}
For incident photon energies less than the Fermi energy, the dominant stopping is provided by Compton scatters with degenerate electrons. 
This stopping power differs parametrically from its usual form due to the target electron motion and degeneracy.

The full Fermi sea prevents electrons from losing energy in a Compton scatter, which restricts possible scatters to those in which the photon and electron initially have roughly collinear momentum. 
In this configuration, the energy of the photon in the rest frame of the electron is red-shifted $k_\x{rest} = k/2\gamma$ where $\gamma$ refers to the electron motion in the WD rest frame. 
For electrons near $E_F$ and $k < E_F$, we thus have $k_\x{rest} < m_e$ and the scattering is Thompson-like in the electron rest frame. 
Energy transfer is thus dominated by backward scatters, for which photons have a final energy $k^\prime = k/4\gamma^2 \ll k$ in the WD rest frame, i.e.~an energy transfer $\omega \approx k$. 
The cross-section for these scatters is of order the Thompson cross-section $\sim \alpha^2/m^2$ as well as an `aiming' factor $1/4\pi$ to account for the restriction to initially parallel trajectories.  
Finally, for $k \ll E_F$ only those electrons near the top of the Fermi sea are available to scatter, so the photon interacts with an effective electron density of 
\begin{align}
    n_\x{eff} = n_e \left[ 1 - \left(1 - \frac{k}{E_f} \right)^3\right]
    \approx 3 n_e \frac{k}{E_f}.
\end{align}
This gives the stopping power 
\begin{align}
  \frac{dk}{dx} \approx \frac{\alpha^2 n_e k^2}{m_e^2 E_F} 
\end{align}
and again we replace $m_e \rightarrow E_F$ to account for relativistic electron motion, 
\begin{align}
  \frac{dk}{dx} \approx \frac{2 \alpha^2 n_e k^2}{E_F^3}. 
\end{align}
Note that $E_F \sim n_e^{1/3}$ for a relativistic degenerate gas and thus the low-energy Compton stopping power is surprisingly independent of density. 

\subsection{Bremsstrahlung and Pair Production with LPM Suppression}
\label{sec:emshowers}

Bremsstrahlung and pair production can be a dominant stopping mechanisms for high-energy electrons and photons.
We restrict our attention to radiative processes off target nuclei rather than target electrons as the latter are additionally suppressed by degeneracy, kinematic recoil, and charge factors.
The cross-section for an electron of energy $E$ to radiate a photon of energy $k$ is given by the Bethe-Heitler formula
\begin{equation}
\label{eq:BH}
\frac{d \sigma_\x{BH}}{dk} = \frac{1}{3 k n_\ion X_0} (y^2+2 [1+ (1-y)^2]), ~~~ y = k/E.
\end{equation}
$X_0$ is the radiation length, and is generally of the form
\begin{equation}
\label{eq:radiationlength}
X_0^{-1} = 4 n_\ion Z^2 \frac{\alpha^3}{m_e^2} \log{\Lambda}, ~~~ \log{\Lambda} \sim \int \frac{1}{b}.
\end{equation}
where $\log{\Lambda}$ is a logarithmic form factor containing the maximum and minimum effective impact parameters allowed in the scatter.
Integrating \eqref{eq:BH}, we find the energy loss due to bremsstrahlung is simply
\begin{equation}
\l\frac{dE}{dx}\r \sim \frac{E}{X_0}.
\end{equation}
In \eqref{eq:radiationlength}, the minimum impact parameter is set by a quantum-mechanical bound such that the radiated photon frequency is not larger than the initial electron energy.
For a bare nucleus, this distance is the electron Compton wavelength.
It is important to note that collisions at lesser impact parameters will still radiate but with suppressed intensity.
The maximum impact parameter is set by the distance at which the nuclear target is screened.
For an atomic target this is of order the Bohr radius, and for nuclear targets in the WD this is the Thomas-Fermi screening radius given by \eqref{eq:TF}.
%Evidently, there exists a critical electron number density $n_e \sim 10^{32} ~\cm^{-3}$ for which the logarithmic form factor appears to vanish.
For our purposes, we simply take $\log{\Lambda} \sim \OO(1)$ for all WD densities under consideration and refrain from a full quantum-mechanical calculation at small impact parameters.

However, bremsstrahlung will be suppressed by the ``Landau-Pomeranchuk-Migdal" (LPM) effect - see \cite{Klein:1998du} for an extensive review.
High-energy radiative processes involve very small longitudinal momentum transfers to nuclear targets ($\propto k/E^2$ in the case of bremsstrahlung).
Quantum mechanically, this interaction is delocalized across a formation length over which amplitudes from different scattering centers will interfere.
This interference turns out to be destructive and must be taken into account in the case of high energies or high-density mediums.
Calculations of the LPM effect can be done semi-classically based on average multiple scattering.
It is found that bremsstrahlung is suppressed for $k < E(E-k)/E_\LPM$, where
\begin{equation}
\label{eq:LPM}
E_\LPM = \frac{m_e^2 X_0 \alpha}{4 \pi}.
\end{equation}
For the WD densities in which radiative energy loss is considered, $E_\LPM \sim 1-100 ~\MeV$.
The degree of suppression is found to be
\begin{equation}
\frac{d\sigma_\LPM/dk}{d\sigma_\x{BH}/dk} = \sqrt{\frac{k E_\LPM}{E (E-k)}},
\end{equation}
so that the bremsstrahlung stopping power in the regime of high-suppression is modified
\begin{equation}
\label{eq:bremloss}
\l\frac{dE}{dx}\r_\LPM \sim \l\frac{E_\LPM}{E} \r^{1/2} \frac{E}{X_0}, ~~~ E>E_\LPM.
\end{equation}
We find that the LPM effect diminishes energy loss due to soft radiation so that the radiative stopping power is dominated by single, hard bremsstrahlung.

In addition to the LPM effect, other forms of interaction within a formation length will suppress bremsstrahlung when $k \ll E$.
The emitted photon can coherently scatter off electrons and ions in the media, acquiring an effective mass of order the plasma frequency $\omega_p$.
Semi-classically, this results in a suppression of order $(k/\gamma \omega_p)^2$ when the radiated photon energy $k < \gamma \omega_p$.
This is known as the ``dielectric effect".
For high-energy electrons, this dielectric suppression only introduces a minor correction to \eqref{eq:bremloss}, in which soft radiation is already suppressed by the LPM effect \cite{Klein:1998du}.

We now briefly summarize the stopping of photons via pair production. Similar to \eqref{eq:BH}, the cross-section for a photon of energy $k$ to produce an electron-positron pair with energies $E$ and $k-E$ is
\begin{equation}
\label{eq:PP}
\frac{d \sigma_\x{BH}}{dE} = \frac{1}{3 k n_\ion X_0} (1+ 2[x^2+ (1-x)^2]) ~~~ x = E/k,
\end{equation}
valid beyond the threshold energy $k \gtrsim m_e$.
As a result, the pair production cross-section $\sim 1/(n_\ion X_0)$.
However, the LPM effect suppresses pair production at energies $E(k-E) > k E_\LPM$ so that the cross-section reduces to
\begin{equation}
\sigma_{pp} \sim \l\frac{E_\LPM}{k} \r^{1/2} \frac{1}{n_\ion X_0}, ~~~ E>E_\LPM.
\end{equation}
Note that the LPM effect is less significant for higher-order electromagnetic processes since these generally involve larger momentum transfers for the same final-state kinematics.
Thus, when the suppression factor exceeds $\OO(\alpha)$, these interactions should also be considered.
For instance, the energy loss due to electron direct pair production $eN \to e^+ e^- e N$ has been calculated in \cite{Gerhardt:2010bj} and is found to exceed that of bremsstrahlung at an energy $\sim 10^{8} ~\GeV$.
A similar crossover is to be expected for other higher-order diagrams as well, although such a calculation is beyond the scope of this work.
Rather, at such high energies the stopping power is dominated by photonuclear and electronuclear interactions anyway, and we may simply ignore the contributions from other radiative processes \cite{Kleinconvo}.

\subsection{Nuclear Interactions}
\label{sec:nuclear}

\paragraph{Elastic Scattering.}
Nuclear interactions can be either elastic or inelastic - the nature of the interaction is largely determined by the incident particle energy.
Elastic collisions are most significant for energy loss at scales less than the nuclear binding energy $\sim 10 ~\MeV$.
These are hard scatters, however, as we are primarily concerned with light hadrons incident on heavy nuclei, these scatters will each transfer a small fraction of the incident energy - i.e.,~ping-pong balls bouncing around a sea of bowling balls.
An elastic collision between a incident, non-relativistic hadron of mass $m$, kinetic energy $E$ and a stationary nucleus of mass $M \gg m$ results in an average energy transfer $\omega$
\begin{equation}
\label{eq:elasticratio}
\omega \sim \l \frac{m}{M}\r E
\end{equation}
where we assume the scattering is not dominated by soft, forward scatters.
Above $\sim \MeV$, it is found that electrostatic repulsion is negligible for nuclear interactions of protons and $\pi^+$.
Therefore, the stopping power for any light hadron due to elastic collisions is simply
\begin{equation}
  \frac{dE}{dx} \sim \frac{m}{M} E n_\ion \sigma_\el
\end{equation}
where $\sigma_\el$ is the elastic nuclear scattering cross-section.
Above $10 ~\MeV$ this approaches the geometric cross-section, which for carbon is $\sim 100 ~\mbn$, while at $\MeV$ energies the elastic cross section generally rises to be of order $\sim \bn$~\cite{Tavernier}.
At intermediate energies $1 - 10 ~\MeV$, the interaction is dominated by various nuclear resonances~\cite{Tavernier} which are of no concern here.
We conservatively estimate the elastic cross-section for nucleons and pions to be $\sigma_\el \approx 1 ~\bn$ when $E \lesssim 10 ~\MeV$.
At higher energies, we ignore elastic interactions as inelastic scatters dominate the energy loss.

\paragraph{Inelastic Scattering.}
Now we determine the stopping power due to inelastic nuclear collisions at $E \gtrsim 10 ~\MeV$.
In such a collision, an incoming hadron interacts with one or more nucleons in the nucleus to produce a $\OO(1)$ number of additional hadrons which approximately split the initial energy.
During this process the target nucleus is broken up.
The nuclear fragment is typically left in an unstable state with negligible center-of-mass recoil, and relaxes via slow emission of low-energy $\sim \MeV$ hadrons and photons, which are a negligible fraction of the incident energy.
For energies greater than the nucleon binding energy $\sim \GeV$, the majority of secondary hadrons are pions which carry transverse momentum of order $\sim 100 ~\MeV$ \cite{Tavernier}.
For incident hadrons in the range $10 ~\MeV - \GeV$, it is found that roughly equal fractions of protons, neutrons, and pions are emitted after each collision \cite{Pionnuclear}.
In either case, if secondary hadrons are sufficiently energetic then they will induce further inelastic collisions.
This results in a roughly collinear hadronic shower terminating at an energy $\sim 10~\MeV$, consisting of pions for most of the shower's development and converting to an mix of pions and nucleons in the final decade of energy.
This cascade is described by a radiative stopping power
\begin{equation}
\label{eq:nucshower}
  \frac{dE}{dx} \sim \frac{E}{l_\inel},
\end{equation}
where $l_\inel$ is the inelastic nuclear mean free path characterized by an inelastic cross-section $\sigma_\inel$.
At these energies, $\sigma_\inel \approx 100 ~\mbn$ and is roughly constant in energy~\cite{Tavernier}.
The length of the shower is only logarithmically dependent on the incident energy,
\begin{align}
    X_\x{had} \sim l_\inel \log\l\frac{E}{10~\MeV}\r.
\end{align}

\paragraph{Photonuclear and Electronucelar Interactions.}
Photons of energy $k \gtrsim 10 ~\MeV$ can also strongly interact with nuclei through the production of virtual quark-antiquark pairs.
This destroys the photon and fragments the nucleus, producing outgoing hadrons in a manner similar to the inelastic collisions of hadrons, although the cross-section $\sigma_{\gamma A}$ is roughly a factor $\approx \alpha$ smaller.
Below $\sim \GeV$ the photonuclear cross-section is complicated by nuclear resonances while above $\sim \GeV$, $\sigma_{\gamma A}$ is a slowly increasing function of energy \cite{Tavernier}.
This increase is due to a coherent interaction of the photon over multiple nuclei at higher energies~\cite{Gerhardt:2010bj}, however instead of extrapolating this we conservatively take a constant photonuclear cross-section of order $\sigma_{\gamma A} \approx \mbn$ for energies $k \gtrsim 10 ~\MeV$.

Electrons can similarly lose energy by radiating a virtual photon that interacts hadronically with a nuclei.
The cross-section for this process is roughly given by the photonuclear cross-section and scaled by a factor representing the probability to radiate such a photon.
This is the Weizsacker-Williams approximation, which gives a cross-section for an electron of kinetic energy $E$ to exchange an energy $k$ with a nucleus
\begin{align}
    \frac{d\sigma}{dk} &\approx \frac{dN}{dk} \sigma_{\gamma A}
\end{align}
where $dN/dk$ is the virtual photon flux \cite{Gerhardt:2010bj}
\begin{align}
    \frac{dN}{dk} &\sim \frac{\alpha}{k} \log\l \frac{E}{m_e} \r.
\end{align}
Integrating this give the stopping power,
\begin{align}
    \frac{dE}{dx} &\sim n_\ion \int_{k_\xmin}^E dk \;
    k \cdot \frac{\alpha}{k} \log\l \frac{E}{m_e} \r  \sigma_{\gamma A} \\
    &\approx \alpha \log\l \frac{E}{m_e} \r \sigma_{\gamma A} n_\ion E.
\end{align}
$k_\xmin$ is taken to be the critical energy for photonuclear interactions.
Electronuclear stopping thus proceeds as a radiative process with length scale larger than the photonuclear length by a factor $\sim 10$.
Unlike the photonuclear event, this is a continuous radiative process with equal energy-loss contritions from radiation with all energies up to $E$.