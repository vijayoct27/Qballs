Here we provide a more detailed analysis of the stopping power (energy loss per distance traveled) of high-energy SM particles in a carbon-oxygen WD due to electromagnetic and strong interactions.
We consider incident electrons, photons, pions, and nucleons with kinetic energy greater than an $\MeV$.

%%%%%%%%%%%%%%%%%%%%%%%%%%%%%%%%%%%%%%%%%%%%%%%%%%%%%%%%%%%%%%%%%%%
\subsection{WD Medium}
For the WD masses that we consider, the stellar medium consists of electrons and fully-ionized carbon nuclei with densities in the range $n_e = Z n_\ion \sim 10^{31} - 10^{33} ~\cm^{-3}$ where $Z=6$.
The internal temperature is $T \sim \keV$~\cite{KippenhahnWeigert}.
The electrons are degenerate and predominantly relativistic free gas, with Fermi energy
\begin{equation}
  E_F \sim (3 \pi^2 n_e)^{1/3} \sim 1 -10 ~\MeV.
\end{equation}
The carbon ions, however, are non-degenerate and do not form a free gas. 
The plasma frequency due to ion-ion Coulomb interactions is given by
\begin{align}
\Omega_p = \l \frac{4 \pi n_\ion Z^2 \alpha}{m_\ion}\r^{1/2} \sim 1 - 10~\keV,
\end{align}
where $m_\ion$ is the ion mass.
Finally, the medium also contains thermal photons, though these are never significant for stopping particles as the photon number density $n_\gamma \sim T^3$ is much smaller than that of electrons or ions.

%%%%%%%%%%%%%%%%%%%%%%%%%%%%%%%%%%%%%%%%%%%%%%%%%%%%%%%%%%%%%%%%%%%
\subsection{Nuclear Interactions}
\label{sec:nuclear}

\paragraph{Elastic Scattering.}
Nuclear interactions can be either elastic or inelastic - the nature of the interaction is largely determined by the incident particle energy.
Elastic collisions are most significant for energy loss at scales less than the nuclear binding energy $\sim 10 ~\MeV$.
These are hard scatters, however, as we are primarily concerned with light hadrons incident on heavy nuclei, these scatters will each transfer a small fraction of the incident energy - i.e.,~ping-pong balls bouncing around a sea of bowling balls.
An elastic collision between a incident, non-relativistic hadron of mass $m$, kinetic energy $E$ and a stationary nucleus of mass $M \gg m$ results in an average energy transfer $\omega$
\begin{equation}
\label{eq:elasticratio}
\omega \sim \l \frac{m}{M}\r E
\end{equation}
where we assume the scattering is not dominated by soft, forward scatters.
Above $\sim \MeV$, it is found that electrostatic repulsion is negligible for nuclear interactions of protons and $\pi^+$.
Therefore, the stopping power for any light hadron due to elastic collisions is simply
\begin{equation}
  \frac{dE}{dx} \sim \frac{m}{M} E n_\ion \sigma_\el
\end{equation}
where $\sigma_\el$ is the elastic nuclear scattering cross-section.
Above $10 ~\MeV$ this approaches the geometric cross-section, which for carbon is $\sim 100 ~\mbn$, while at $\MeV$ energies the elastic cross section generally rises to be of order $\sim \bn$~\cite{Tavernier}.
At intermediate energies $1 - 10 ~\MeV$, the interaction is dominated by various nuclear resonances~\cite{Tavernier} which are of no concern here.
We conservatively estimate the elastic cross-section for nucleons and pions to be $\sigma_\el \approx 1 ~\bn$ when $E \lesssim 10 ~\MeV$.
At higher energies, we ignore elastic interactions as inelastic scatters dominate the energy loss.

\paragraph{Inelastic Scattering.}
Now we determine the stopping power due to inelastic nuclear collisions at $E \gtrsim 10 ~\MeV$.
In such a collision, an incoming hadron interacts with one or more nucleons in the nucleus to produce a $\OO(1)$ number of additional hadrons which approximately split the initial energy.
During this process the target nucleus is broken up.
The nuclear fragment is typically left in an unstable state with negligible center-of-mass recoil, and relaxes via slow emission of low-energy $\sim \MeV$ hadrons and photons, which are a negligible fraction of the incident energy.
For energies greater than the nucleon binding energy $\sim \GeV$, the majority of secondary hadrons are pions which carry transverse momentum of order $\sim 100 ~\MeV$ \cite{Tavernier}.
For incident hadrons in the range $10 ~\MeV - \GeV$, it is found that roughly equal fractions of protons, neutrons, and pions are emitted after each collision \cite{Pionnuclear}.
In either case, if secondary hadrons are sufficiently energetic then they will induce further inelastic collisions.
This results in a roughly collinear hadronic shower terminating at an energy $\sim 10~\MeV$, consisting of pions for most of the shower's development and converting to an mix of pions and nucleons in the final decade of energy.
This cascade is described by a radiative stopping power
\begin{equation}
\label{eq:nucshower}
  \frac{dE}{dx} \sim \frac{E}{l_\inel},
\end{equation}
where $l_\inel$ is the inelastic nuclear mean free path characterized by an inelastic cross-section $\sigma_\inel$.
At these energies, $\sigma_\inel \approx 100 ~\mbn$ and is roughly constant in energy~\cite{Tavernier}.
The length of the shower is only logarithmically dependent on the incident energy,
\begin{align}
    X_\x{had} \sim l_\inel \log\l\frac{E}{10~\MeV}\r.
\end{align}

\paragraph{Photonuclear and Electronucelar Interactions.}
Photons of energy $k \gtrsim 10 ~\MeV$ can also strongly interact with nuclei through the production of virtual quark-antiquark pairs.
This destroys the photon and fragments the nucleus, producing outgoing hadrons in a manner similar to the inelastic collisions of hadrons, although the cross-section $\sigma_{\gamma A}$ is roughly a factor $\approx \alpha$ smaller.
Below $\sim \GeV$ the photonuclear cross-section is complicated by nuclear resonances while above $\sim \GeV$, $\sigma_{\gamma A}$ is a slowly increasing function of energy \cite{Tavernier}.
This increase is due to a coherent interaction of the photon over multiple nuclei at higher energies~\cite{Gerhardt:2010bj}, however instead of extrapolating this we conservatively take a constant photonuclear cross-section of order $\sigma_{\gamma A} \approx \mbn$ for energies $k \gtrsim 10 ~\MeV$.

Electrons can similarly lose energy by radiating a virtual photon that interacts hadronically with a nuclei.
The cross-section for this process is roughly given by the photonuclear cross-section and scaled by a factor representing the probability to radiate such a photon.
This is the Weizsacker-Williams approximation, which gives a cross-section for an electron of kinetic energy $E$ to exchange an energy $k$ with a nucleus
\begin{align}
    \frac{d\sigma}{dk} &\approx \frac{dN}{dk} \sigma_{\gamma A}
\end{align}
where $dN/dk$ is the virtual photon flux \cite{Gerhardt:2010bj}
\begin{align}
    \frac{dN}{dk} &\sim \frac{\alpha}{k} \log\l \frac{E}{m_e} \r.
\end{align}
Integrating this give the stopping power,
\begin{align}
    \frac{dE}{dx} &\sim n_\ion \int_{k_\xmin}^E dk \;
    k \cdot \frac{\alpha}{k} \log\l \frac{E}{m_e} \r  \sigma_{\gamma A} \\
    &\approx \alpha \log\l \frac{E}{m_e} \r \sigma_{\gamma A} n_\ion E.
\end{align}
$k_\xmin$ is taken to be the critical energy for photonuclear interactions.
Electronuclear stopping thus proceeds as a radiative process with length scale larger than the photonuclear length by a factor $\sim 10$.
Unlike the photonuclear event, this is a continuous radiative process with equal energy-loss contritions from radiation with all energies up to $E$.

%%%%%%%%%%%%%%%%%%%%%%%%%%%%%%%%%%%%%%%%%%%%%%%%%%%%%%%%%%%%%%%%%%%
\subsection{Radiative Processes}
\label{sec:emshowers}

EM showers due to successive bremsstrahlung and pair production events off carbon ions are the dominant stopping mechanisms for intermediate-energy electrons and photons.
Both of these processes result in radiative stopping powers, expressed semi-classically as~\cite{Klein:1998du} 
\begin{align}
\label{eq:SemiclassicalBrem}
  \frac{dE}{dx} \sim n_\ion \frac{Z^2 \alpha^3}{m_e^2} E \; \log{\Lambda}
\end{align}
where $\log\Lambda$ is a Coulomb form factor given by the range of effective impact parameters $b$:
\begin{align}
  \Lambda = \frac{b_\xmax}{b_\xmin} \sim \lambda_\TF m_e.
\end{align} 
The maximal impact parameter is set by the plasma screening length and the minimum by the electron mass, below which the semi-classical description breaks down. 
For the highest WD densities, we may indeed have that $\Lambda \lesssim 1$, in which case equation~\eqref{eq:SemiclassicalBrem} ought be replaced by a quantum mechanical result, such as the Bethe-Heitler formula~\cite{Bethe1934}.
This still results in a stopping power $\sim n_\ion Z^2 \alpha^3 E/m_e^2$, and so for simplicity we employ equation~\eqref{eq:SemiclassicalBrem} and take $\log{\Lambda} \sim \OO(1)$ for all WD densities.

\paragraph{LPM Suppression}
A radiative event involving momentum transfer $q$ to an ion must, quantum mechanically, occur over a length $\sim q^{-1}$. 
All ions within this region contribute to the scattering of the incident particle, and for sufficiently small $q$ this results in a decoherence that suppresses the formation of photons or electron-position pairs.
This is the ``Landau-Pomeranchuk-Midgal'' (LPM) effect. 
The momentum transfer $q$ in a given event decreases with increasing incident particle energy, and so the LPM effect will suppress radiative processes for energies greater than some scale $E_\LPM$. 
This can be calculated semi-classically~\cite{Klein:1998du}, 
\begin{align}
  E_\LPM = \frac{m_e^4}{4 \pi Z^2 \alpha^2 n_\ion}
  \approx 1~\MeV \l \frac{10^{32} \cm^{-3}}{n_\ion} \r.
\label{eq:LPM}
\end{align}
which is quite small due to the high ion density in the WD. 
The stopping power for $E > E_\LPM$ is
\begin{align}
  \l \frac{dE}{dx} \r \sim 
  n_\ion \frac{Z^2 \alpha^3}{m_e^2} E \cdot
  \l \frac{E_\LPM}{E} \r^{1/2} \label{eq:LPMBrem} 
\end{align}

In addition to the LPM effect, soft bremsstrahlung may be suppressed in a medium as the emitted photon acquires an effective mass of order the plasma frequency $\Omega_p$.
However, for high-energy electrons this dielectric suppression only introduces a minor correction to \eqref{eq:LPMBrem}, in which soft radiation is already suppressed~\cite{Klein:1998du}.

%%%%%%%%%%%%%%%%%%%%%%%%%%%%%%%%%%%%%%%%%%%%%%%%%%%%%%%%%%%%%%%%%%%
\subsection{Coulomb Scattering}
\label{sec:coulomb}

\paragraph{Scattering off Carbon Ions.}
\label{sec:coulomb_ion}
Coulomb collisions with ions provide the dominant mechanism by which electrons with energy $1~\MeV \lesssim E \lesssim 10~\MeV$ thermalize ions.
In this scenario we may treat the ions as stationary and ignore their recoil during collisions.
The ionic charge will be screened by the mobile electrons of the medium, so incident particles will scatter via a potential
\begin{align}
  \label{eq:ScreenedPotential}
V(\textbf{r}) = \frac{Z \alpha}{r} e^{-\lambda_\TF r}.
\end{align}
The screening length $\lambda_\TF$ is given in the Thomas-Fermi approximation by \cite{Teukolsky}
\begin{align}
\label{eq:TF}
    \lambda_\TF^{2} = \frac{E_F}{6 \pi \alpha n_e} 
    \sim \frac{1}{\alpha n_e^{2/3}}
\end{align}
where $E_F$ is the electron Fermi energy.
This plasma screening suppresses scatters with momentum transfers below $\sim \lambda_\TF^{-1}$, corresponding to a minimal energy transfer of $\omega_\xmin = \lambda_\TF^{-2} / 2 m_\ion$.
Ions may in principle also cause screening through lattice distortion, however this may be ignored as the sound speed of the lattice $c_s \sim 10^{-2}$ is much smaller than the speed of an incident relativistic electron. 
Using the Born approximation, we have a cross-section for energy transfer $\omega$
\begin{align}
\label{eq:CoulombOffIonsCrossSection}
  \frac{d \sigma}{d \omega} = 
  \frac{2 \pi Z^2 \alpha^2}{m_\ion\beta^2} 
  \frac{1}{(\omega + \omega_\xmin)^2}
\end{align}
and a stopping power 
\begin{align}
  \frac{dE}{d x} &= \int_{0}^{\omega_\xmax} d \omega \, n_\ion 
  \frac{d \sigma}{d \omega} \omega \nonumber\\
  \label{eq:StoppingPowerOffIons}
   &\approx \frac{2 \pi\, n_\ion Z^2 \alpha^2 }{m_\ion\beta^2} 
   \log\left( \frac{\omega_\xmax}{\omega_\xmin} \right)
\end{align}
where the second line is valid if $\omega_\xmax \gg \omega_\xmin$.
$\omega_\xmax$ is the maximum possible energy transfer. 
This may be due to 4-momentum conservation, or in the case of incident electrons, the impossibility of scattering to a final energy less than $E_F$. 
4-momentum conservation sets an upper bound $\omega_\kin$, which for a stationary target is
\begin{align}
  \omega_\kin &= \frac{2 m_\ion p^2}{m_\ion^2 + m^2 + 2E m_\ion}
\end{align}
with $p$, $E$ the incoming momentum and energy. 
The Fermi upper bound is simply $\omega_F = E - E_F$ and for incident electrons $\omega_\xmax = \min\l\omega_\kin, \omega_F\r$.

For scatters that transfer energy less than the plasma frequency $\Omega_p$, one may be concerned about phonon excitations.
We estimate this by treating each ion as an independent oscillator with frequency $\Omega_p$ (an Einstein solid) and compute the stopping power due to scatters which excite a single oscillator quanta. 
There are two key differences between this and the free ion case: incident particles must transfer an energy $\Omega_p$ and the cross-section to transfer momentum $q$ is suppressed by a factor $q^2 / 2 m_\ion\Omega_p = \omega_\x{free}/\Omega_p$. 
$\omega_\x{free}$ is the energy transfer that would accompany a free ion scatter with momentum transfer $q$. 
The resulting stopping power is unchanged from the free case~\eqref{eq:StoppingPowerOffIons}, as the increased energy transfer compensates for the suppressed cross-section:
\begin{align}
  d\sigma \cdot \omega \sim 
  d\sigma_\x{free} \frac{\omega_\x{free}}{\Omega_p} 
  \cdot \Omega_p \sim 
  d\sigma_\x{free} \cdot \omega_\x{free}.
\end{align}

Finally, we note that for more massive or more energetic incident particles the cross-section~\eqref{eq:CoulombOffIonsCrossSection} must be replaced with a more complicated expression to account for the recoil of the ion during collisions. 
In these scenarios, however, this stopping power is far subdominant to hadronic or electromagnetic showers. \\

\paragraph{Scattering off Degenerate Electrons.}
\label{sec:coulomb_elec}
The scattering of incident electrons with the degenerate electron sea determines the termination energy of electromagnetic showers, and so we focus here on that scenario. 
This calculation demands two considerations not present when scattering off ions: the targets are not stationary and they require a threshold energy transfer in order to be scattered out of the Fermi sea.
In the scenario of interest, however, these do not result in a parametrically different stopping power than was found for ions in equation~\eqref{eq:StoppingPowerOffIons}. 

For incident momenta much greater than the Fermi momentum, the relative velocity is of order the incident velocity and the deflection of the incident particle will generally be small. 
It is reasonable then that scattering proceeds, up to $\OO(1)$ factors, as though a heavy incident particle is striking a light, stationary target.  
The cross-section is then given by the usual result, 
\begin{align}
  \frac{d \sigma}{d \omega} \approx
  \frac{2 \pi \alpha^2}{E_F} \frac{1}{\omega^2},
  \label{eq:CoulombRelativisticApprox}
\end{align}
where we have accounted for the target's motion by replacing its mass with the relativistic inertia $\sqrt{m_e^2 + p^2} \approx E_F$.  We have ignored plasma screening, as Pauli-blocking will provide a more stringent cutoff on soft scatters in this case. 
Scatters which transfer an energy $\omega \leq E_F$ will have a suppressed contribution to the stopping power as they can only access a fraction of the Fermi sea. 
For incident energies $E \gg E_F$ it is sufficient to ignore these suppressed scatters, i.e.,
\begin{align}
  \frac{dE}{d x} &= \int_{E_F}^{\omega_\xmax} d \omega \, n_e 
  \frac{d \sigma}{d \omega} \omega \nonumber\\
  \label{eq:StoppingPowerOffElectrons}
   &\approx \frac{2 \pi\, n_e \alpha^2 }{E_F} 
   \log\left( \frac{\omega_\xmax}{E_F} \right)
\end{align}
where, as above, $\omega_\xmax = \min\l\omega_\kin, \omega_F\r$.
This derivation is admittedly quite heuristic, and so it has been checked with a detailed numerical calculation accounting fully for the target's motion and degeneracy.
Equation~\eqref{eq:StoppingPowerOffElectrons} is indeed a good approximation to the stopping power for incient energies larger than the Fermi energy. 

%%%%%%%%%%%%%%%%%%%%%%%%%%%%%%%%%%%%%%%%%%%%%%%%%%%%%%%%%%%%%%%%%%%
\subsection{Compton Scattering}
\label{sec:compton}
For incident photon energies less than the Fermi energy, the dominant stopping is provided by Compton scatters with degenerate electrons. 
This stopping power differs parametrically from its usual form due to the target electron motion and degeneracy.

Consider an incident photon with energy $k \leq E_F$. 
Compton scatters will only occur off electrons moving roughly collinear with the photon momentum - a head-on collision would result in an energy loss for the electron, which is forbidden by Pauli exclusion. 
In the electron rest frame these collinear scatters are Thompson-like, and so the photon energy loss is dominated by backward scatters. 
For relativistic electrons near the Fermi surface, these scatters transfer an energy
\begin{align}
  \omega \approx k \l 1 - \frac{m_e^2}{4 E_F^2} \r \sim k.
\end{align}  
The cross-section can be taken in the electron rest frame, $\sigma_T \sim \alpha^2/m_e^2$, along with an `aiming' factor $1/4\pi$ to account for the restriction to initially parallel trajectories.  
Finally, for $k \ll E_F$ only those electrons near the top of the Fermi sea are available to scatter, so the photon interacts with an effective electron density of 
\begin{align}
    n_\x{eff} = \int_{E_F - k}^{E_F} g(E) \; dE 
    \approx 3 n_e \frac{k}{E_f}
\end{align}
where $g(E)$ is the Fermi density of states. This gives a stopping power 
\begin{align}
  \frac{dk}{dx} \approx \frac{\alpha^2 n_e k^2}{m_e^2 E_F}. 
\end{align}