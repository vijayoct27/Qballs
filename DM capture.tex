%\documentclass[10pt, twocolumn]{article}
%\documentclass[11pt]{article}
%\documentclass[twocolumn,showpacs,preprintnumbers,amsmath,amssymb,prl, superscriptaddress]{revtex4}
%\documentclass[twocolumn, preprintnumbers,amsmath,amssymb,prd, superscriptaddress]{revtex4}
\documentclass[preprintnumbers,amsmath,amssymb,prd,superscriptaddress]{revtex4}
%\documentclass[10pt, preprint,showpacs,preprintnumbers,amsmath,amssymb, superscriptaddress]{revtex4}
%\documentclass[11pt, prd,preprintnumbers,amsmath,amssymb, superscriptaddress]{revtex4}
%\documentclass[11pt, prd,preprintnumbers, amsmath,amssymb, superscriptaddress, nofootinbib, hyperref]{revtex4}

\usepackage{latexsym}
\usepackage{amssymb}
\usepackage{epsfig,amsmath,graphics}
\usepackage{epstopdf}
\usepackage{verbatim}
\usepackage{wasysym}
\usepackage{hyperref}
\usepackage{feynmp-auto} % feynman diagrams
%\usepackage{subfig}
\usepackage[utf8]{inputenc}
\usepackage{xpatch}
\usepackage{xcolor}
\usepackage{mathtools}
\hypersetup{
    colorlinks,
    linkcolor={red!80!black},
    citecolor={green!60!black},
    urlcolor={blue!60!black}
}
\usepackage{appendix}

\newcommand{\Ez}{\mathcal{E}_0}
\newcommand{\Eboom}{\mathcal{E}_\text{boom}}
\newcommand{\OO}{\mathcal{O}}
\newcommand{\LL}{\mathcal{L}}
\newcommand{\HH}{\mathcal{H}}
\newcommand{\TeV}{\text{TeV}}
\newcommand{\GeV}{\text{GeV}}
\newcommand{\MeV}{\text{MeV}}
\newcommand{\keV}{\text{keV}}
\newcommand{\rad}{\text{rad}}
\newcommand{\cm}{\text{cm}}
\newcommand{\angstrom}{\buildrel _{\circ} \over {\mathrm{A}}}
\newcommand{\pslash}{p\hspace{-0.070in}/\,}
\newcommand{\Mpl}{M_{\text{pl}}}
\newcommand{\ket}[1]{\ensuremath{\left|#1\right>}}
\newcommand{\bra}[1]{\ensuremath{\left<#1\right|}}
\newcommand{\braket}[2]{\ensuremath{\left<#1|#2\right>}}
%Large Parentheses
\def\r{\right)}
\def\l{\left(}

\begin{document}

%\preprint{APS/123-QED}

Thus for a heating event characterized by its $L_0$, $\Ez$, and $T_0 \gtrsim T_f$, there is a \emph{boom condition}:
\begin{align}
    \label{eq:energy_boom_condition}
    \Ez \gtrsim
    \Eboom \cdot \text{max}\left\{1, \frac{L_0}{\lambda_T}\right\}^3.
\end{align}

\paragraph{Boom Condition.}
For a point-like DM-DM collision or DM decay event releasing particles of heating length $L_0$, ignition will occur if the total energy in SM products satisfies condition~\eqref{eq:energy_boom_condition}.
Such an event will likely result in both SM and dark sector products, so we parameterize the resulting energy in SM particles as a fraction $f_\text{SM}$ of the DM mass.
For non-relativistic DM, the DM mass is the dominant source of energy and therefore $f_\text{SM} \lesssim 1$ regardless of the interaction details, although we may well suspect that $f_\text{SM} \ll 1$ for realistic models.
With this parameterization, a single DM-DM collision or DM decay has a boom condition:
\begin{equation}
\label{eq:coldecay}
  m_\text{DM} f_\text{SM}  \gtrsim \Eboom \cdot \text{max} \left \{\frac{L_0}{\lambda_T}, 1 \right \}^3.
\end{equation}
We are thus sensitive to DM masses $m_\text{DM} \gtrsim 10^{16} ~\GeV$.

\paragraph{Event Rate: DM Wind.}
DM with negligible energy loss in the WD medium will traverse the star in $\sim R_\text{WD}/v_\text{esc} \approx 0.1 ~\text{s}$ and have a number density within the WD enhanced relative to the average galactic density by a factor $v_\text{esc}/v \sim \OO(10)$.
In the wind scenario, the DM-DM collision rate inside the WD parameterized by a cross-section $\sigma_\text{DM-DM}$ is:
\begin{align}
  \Gamma_\text{collision}
  %&\sim n_\text{DM} \sigma_\text{DM-DM} v_\text{esc} \cdot n_\text{DM} R_\text{WD}^3 \\
  \sim \l \frac{\rho_\text{DM}}{m_\text{DM}} \r^2 \sigma_\text{DM-DM} \l \frac{v_\text{esc}}{v}\r^2 v_\text{esc} R_\text{WD}^3.
  \label{eq:collisionDM}
\end{align}
Similarly the net DM decay rate inside the WD parameterized by a lifetime $\tau_\text{DM}$ is:
\begin{align}
 \Gamma_\text{decay}
  %&\sim \frac{1}{\tau_\text{DM}} \cdot n_\text{DM} R_\text{WD}^3 \\
  \sim \frac{1}{\tau_\text{DM}} \frac{\rho_{\text{DM}}}{m_\text{DM}} \l \frac{v_\text{esc}}{v}\r R_\text{WD}^3.
  \label{eq:decayDM}
\end{align}


\paragraph{Event Rate}
For the DM to be able to stop in a WD, it must have sufficiently strong interactions with stellar constituents:
\begin{equation}
\left( \frac{d E}{d x} \right)_\text{SP} \gtrsim \frac{m_\text{DM} v^2}{R_\text{WD}}.
\end{equation}
Note that the DM need only lose velocity of order its galactic virial velocity to get captured. 
Implicitly, we are making the assumption that the scatters responsible for slowing the DM are not sufficient to blow up the WD, i.e. \eqref{eq:transitboom} is never satisfied. 
This is reasonable: if $m_\text{DM} \gg m_\text{ion}$, 

First we review the evolution of DM within the star once it has been captured. 
The DM eventually thermalizes at a velocity
\begin{equation}
v_\text{th} \sim \sqrt{\frac{T}{m_\text{DM}}} \approx 10^{-12} \l \frac{10^{16} ~\GeV}{m_\text{DM}}\r^{1/2},
\end{equation}
where $T \sim \keV$ is the WD temperature.
It will then accumulate at the thermal radius set by $v_\text{th}$
\begin{align}
  R_\text{th} &\sim \l \frac{T}{G m_\text{DM} \rho_\text{WD}}\r^{1/2} \\
  &\approx 0.1 ~\text{cm} \l \frac{10^{16} ~\GeV}{m_\text{DM}}\r^{1/2}
  \l \frac{10^{31} ~\cm^{-3}}{n_\text{ion}}\r^{1/2}, \nonumber
\end{align}
where we have assumed a constant WD density $\rho_\text{WD} \sim n_\text{ion} m_\text{ion}$ within $R_\text{th}$.
DM will collect at this radius until its total mass exceeds the WD mass within $R_\text{th}$,
\begin{align}
\label{eq:mcore}
    M_\text{crit} &\sim \rho_\text{WD} R^3_\text{th} \\
    &\approx 10^{29}~\GeV ~\l \frac{10^{16} ~\GeV}{m_\text{DM}}\r^{3/2}
  \l \frac{10^{31} ~\cm^{-3}}{n_\text{ion}}\r^{1/2}. \nonumber
\end{align}
The DM cloud will then begin self-gravitational collapse.
The exact nature of this collapse is model-dependent, eventually being arrested by DM-DM interactions or the formation of a black hole.
For composite DM, it is reasonable to suspect that the collapse stabilizes into a core of radius $R_\text{sta}$ larger than the Schwarzschild radius due to some unknown physics. 

There are several timescales relevant to the DM capture scenario. 
Initially there is a timescale for DM to slow down to the thermal velocity $v_\text{th}$ and ``drift" down to the thermal radius.
This is roughly given by
\begin{align}
\label{eq:tdrift}
  t_\text{drift} \sim \frac{R_\text{WD}}{v_\text{th}}
  \approx 50 ~\text{yr} ~ \l \frac{m_\text{DM}}{10^{16} ~\GeV} \r^{1/2}. 
\end{align}
Evidently for $m_\text{DM} \gtrsim 10^{30}~\GeV$, a DM cloud at $R_\text{th}$ does not even form within the lifetime of the star. 
Once formed, the DM cloud will continue to collect DM until it reaches a critical core mass \eqref{eq:mcore} in a time
\begin{align}
\label{eq:tcol}
  t_\text{collect} &\sim \l \frac{M_\text{crit}}{m_\text{DM}}\r \frac{1}{\Gamma_\text{transit}} \\
  & \approx 10 ~\text{s} \l \frac{10^{16} ~\GeV}{m_\text{DM}} \r^{3/2} \l \frac{0.4 ~\GeV/\text{cm}^3}{\rho_\text{DM}} \r,\nonumber
\end{align}
evaluated at $n_\text{ion} \sim 10^{31} ~\cm^{-3}$.
Note that $t_\text{collect}$ has a non-trivial dependence on WD density: this is manifest in the values for $v_\text{esc}$ and $R_\text{WD}$.
Subsequently, the actual timescale for the collapse is independent of DM mass:
\begin{align}
  t_\text{collapse} \sim \frac{R_\text{th}}{v_\text{th}}
  \approx 3 ~\text{s} ~ \l \frac{10^{31} ~\cm^{-3}}{n_\text{ion}}\r^{1/2}.
\end{align}
This expression is valid for the two cases that the DM core either initially remains thermalized while self-gravitating or collapses at free-fall velocity.  

We now turn towards the rate of explosive collisions and decays for captured DM.
Of course, if the slowing of DM itself is sufficient to ignite a SN, then the boom condition is instead given by the transit condition \eqref{eq:transitexplosion}.
Assuming this is not the case, we calculate the rate of DM events during the various stages of the capture described above.  

For decay heating, capture gives an enhancement due to the increased number of DM particles within the WD.
This can be very large if the DM core admits decays, however it is still significantly enhanced over the wind scenario even for inert cores (as in the case that the DM forms a black hole).
We have an enhancement of the net decay rate \eqref{eq:decayDM} by a factor
\begin{align}
\label{eq:enhancedecay}
  \frac{v_\text{esc}}{v_\text{th}}
  \approx 10^{10} \l \frac{m_\text{DM}}{10^{16}~\GeV} \r^{1/2}
\end{align}
due to the increased time spent by the DM in the WD medium before joining the inert core.

In the case of DM-DM collision heating, it is possible that the collapse of the core will induce an ignition event due to the enhancement of DM number density during the collapse.
This would set the lifetime of WDs to $t_\text{collect}$.
During the collapse, the rate of collisions taking place at a radius $r$ within the enclosed volume is given by
\begin{align}
\Gamma_\text{collision}(r) \sim \l \frac{M_\text{core}}{m_\text{DM}} \r^2 \frac{1}{r^3} \sigma_\text{DM-DM} v(r),
\end{align}
where $v(r)$ is the velocity of DM - this could be either free-fall velocity or $v_\text{th}$ if the DM remains thermalized.
Integrating to the stable radius $R_\text{sta}$, we find the total number of collisions during the collapse is
\begin{align}
  N_\text{col} \sim \l \frac{M_\text{core}}{m_\text{DM}} \r^2 \frac{\sigma_\text{DM-DM}}{R_\text{sta}^2}.
 \end{align}
Assuming the collapse proceeds until the DM core becomes a black hole, the number of collisions is
\begin{align}
  N_\text{col} \sim \frac{\sigma_\text{DM-DM}}{G^2 m_\text{DM}^2}.
 \end{align}
If the collapse itself is not explosive, there is still an enhanced collision rate relative to the wind scenario due to DM colliding while in-falling to the core.
Again we look at the conservative situation of an inert core - the rate is obviously much greater if the core is stabilized in a fluid state which admits DM-DM collisions.
The rate of in-falling collisions is enhanced over the wind collision rate \eqref{eq:collisionDM} by a factor
\begin{align}
\label{eq:enhancecollision}
   \frac{R_\text{WD}}{R_\text{sta}} \times \frac{v_\text{esc}}{v_\text{th}},
\end{align}
which again depends on the physics of $R_\text{sta}$.

\paragraph{Event Rate: DM Capture and Multiple Collisions}

Up until now, we only considered the case that a single DM-DM collision releases sufficient energy \eqref{eq:coldecay} in order to trigger runaway fusion.
However, the possibility of a DM core collapse in the star provides up an interesting alternative:

\end{document}