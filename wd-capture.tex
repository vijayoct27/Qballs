Here we give a more detailed discussion of DM capture. 

\subsection{Capture Rate}
Consider spin-independent DM elastic scattering off ions with cross section $\sigma_{\chi A}$.
This is related to the per-nucleon cross section
\begin{equation}
\sigma_{\chi A} = A^2 \l \frac{\mu_{\chi A}}{\mu_{\chi n}}\r^2 F^2(q) \sigma_{\chi n} = A^4 F^2(q) \sigma_{\chi n},
\end{equation}
where $F^2(q)$ is the Helm form factor~\cite{Helm:1956zz}.
If the DM is at the WD escape velocity, the typical momentum transfer to ions is $q \sim \mu_{\chi A} v_\text{esc} \sim 200 ~\MeV$. 
As this $q$ is less than or of order the inverse nuclear size, DM scattering off nuclei will be coherently enhanced. 
We find $F^2(q) \approx 0.1$ for $q \sim 200 ~\MeV$.

For the DM to ultimately be captured, it must lose energy $\sim m_\chi v^2$, where $v$ is the DM velocity (in the rest frame of the WD) asymptotically far away.
Since typically $v \ll v_\text{esc}$, the DM has velocity $v_\text{esc}$ while in the star and must lose a fraction $(v/v_\text{esc})^2$ of its kinetic energy to become captured.
Properly, the DM velocity is described by a boosted Maxwell distribution peaked at the galactic virial velocity $v_\text{halo} \sim 10^{-3}$.
However, this differs from the ordinary Maxwell distribution by only $\OO(1)$ factors~\cite{Gould:1987ir}, and we can approximate it by (ignoring the exponential Boltzmann tail):
\begin{equation}
\frac{dn_\chi}{dv} \approx
\begin{cases}
  \frac{\rho_\chi}{m_\chi} \l \frac{v^2}{v_\text{halo}^3} \r  & v \leq v_\text{halo} \\
  0 & v > v_\text{halo}
  \end{cases}.
\end{equation}
The DM capture rate is given by an integral of the DM transit rate weighted by a probability for capture $P_\text{cap}$
\begin{equation}
\Gamma_\text{cap} \sim \int dv \frac{d \Gamma_\text{trans}}{dv} P_\text{cap}(v),
\end{equation}
where the (differential) transit rate is
\begin{equation}
\frac{d \Gamma_\text{trans}}{dv} \sim \frac{d n_\chi}{dv} R_\text{WD}^2 \l \frac{v_\text{esc}}{v}\r^2 v.
\end{equation}
$P_\text{cap}$ depends on both the \emph{average} number of scatters in a WD
\begin{equation}
\overbar{N}_\text{scat} \sim n_\text{ion} \sigma_{\chi A} R_\text{WD},
\end{equation}
and the number of scatters \emph{needed} for capture
\begin{equation}
N_\text{cap} \sim \text{max}\left \{1, \frac{m_\chi v^2}{m_\text{ion} v_\text{esc}^2}\right \},
\end{equation}
and is most generally expressed as a Poisson sum
\begin{equation}
P_\text{cap} = 1 - \sum^{N_\text{cap}-1}_{n=0} \exp(-\overbar{N}_\text{scat})\frac{(\overbar{N}_\text{scat})^n}{n!}.
\end{equation}
For our purposes we will approximate the sum as follows:
\begin{equation}
P_\text{cap} \approx
\begin{cases}
 1 & \overbar{N}_\text{scat} > N_\text{cap} \\
 \overbar{N}_\text{scat} & \overbar{N}_\text{scat} < N_\text{cap} ~\text{and}~ N_\text{cap} = 1 \\
 0 & \text{else}
\end{cases}.
\end{equation}
Here we ignore the possibly of capture if $\overbar{N}_\text{scat} < N_\text{cap}$ except in the special case that only one scatter is needed for capture.
If $\overbar{N}_\text{scat} > N_\text{cap}$, we assume all DM is captured.
Most accurately, this capture rate should be computed numerically, e.g. see~\cite{Bramante:2017xlb} for a detailed calculation.
However with the above simplifications we find that the capture rate is of order
\begin{align}
  \Gamma_\text{cap} &\sim \Gamma_\text{trans} \cdot
  \text{min}\left\{1, \overbar{N}_\text{scat} \text{min}\{B,1\}\right\}, \\
  B &\equiv \frac{m_\text{ion} v_\text{esc}^2}{m_\chi v_\text{halo}^2}.
  \nonumber
\end{align}
$B$ here encodes the necessity of multiple scattering for capture.
For ultra-heavy DM $m_\chi > 10^{15} ~\GeV$, $B \ll 1$ and essentially multiple scatters are always needed.

\subsection{Thermalization and Collapse}
For the remainder of this section all numerical quantities are evaluated at a central WD density $n_\text{ion} \sim 10^{31} ~\cm^{-3}$, for which the relevant WD parameters are~\cite{cococubed}:
\begin{align}
M_\text{WD} &\approx 1.25 ~M_{\astrosun} \nonumber \\
R_\text{WD} &\approx 4000 ~\text{km} \nonumber \\
v_\text{esc} &\approx 2 \times 10^{-2}.
\end{align}
Depending on the context, the relevant density may be the average value which is of order $n_\text{ion} \approx 10^{30} ~\cm^{-3}$.
We also assume a typical WD temperature $T \sim \text{keV}$.

Once DM is captured, it thermalizes to an average velocity
\begin{equation}
  v_\text{th} \sim \sqrt{\frac{T}{m_\chi}}
  \approx 10^{-11} \l \frac{m_\chi}{10^{16} ~\GeV}\r^{-1/2},
\end{equation}
and settles to the thermal radius
\begin{align}
  R_\text{th} &\sim \l \frac{T}{G m_\chi \rho_\text{WD}}\r^{1/2} \\
 &\approx 0.1 ~\cm \l \frac{m_\chi}{10^{16} ~\GeV}\r^{-1/2}, \nonumber
\end{align}
where its kinetic energy balances against the gravitational potential energy of the (enclosed) WD mass.
This thermalization time can be explicitly calculated for elastic nuclear scatters~\cite{Kouvaris:2010jy}.
The stopping power due to such scatters is
\begin{align}
    \frac{dE}{dx} \sim \rho_\text{WD} \sigma_{\chi A} \; v \; \text{max}\{v, v_\ion\},
\end{align}
where $v_\ion \sim \sqrt{\frac{T}{m_\text{ion}}}$ is the thermal ion velocity.
The max function indicates the transition between ``viscous" and ``inertial" drag.
DM first passes through the WD many times on a wide orbit until the size of its orbit decays to become contained in the star.
The timescale for this process is
\begin{align}
  t_1 &\sim \l \frac{m_\chi}{m_\text{ion}} \r^{3/2}
  \frac{R_\text{WD}}{v_\text{esc}} \frac{1}{\overbar{N}_\text{scat}}
  \frac{1}{\text{max}\{\overbar{N}_\text{scat}, 1\}^{1/2}} \\
  &\approx 7 \times 10^{16}~\text{s} \l \frac{m_\chi}{10^{16} ~\GeV} \r^{3/2}
  \l \frac{\sigma_{\chi A}}{10^{-35} ~\cm^2} \r^{-3/2}. \nonumber
\end{align}
Subsequently, the DM completes many orbits within the star until dissipation further reduces the orbital size to the thermal radius.
The timescale for this process is
\begin{align}
  t_2  &\sim \l \frac{m_\chi}{m_\text{ion}} \r
  \frac{1}{n_\text{ion} \sigma_{\chi A}} \frac{1}{v_\text{ion}} \\
  &\approx 10^{14}~\text{s}\l \frac{m_\chi}{10^{16} ~\GeV} \r
  \l \frac{\sigma_{\chi A}}{10^{-35} ~\cm^2} \r^{-1}. \nonumber
\end{align}
There is an additional $\OO(10)$ logarithmic enhancement of the timescale once the DM velocity has slowed below $v_\ion$.
Note that time to complete a single orbit is set by the gravitational free-fall timescale:
\begin{equation}
\label{eq:freefalltime}
t_\text{ff} \sim \sqrt{\frac{1}{G \rho_\text{WD}}} \approx 0.5 ~\text{s}.
\end{equation}
Of course, the thermalization is qualitatively different than described above if the DM loses sufficient energy within a single transit.
This occurs for large scattering cross sections:
\begin{equation}
\sigma_{\chi A} > \frac{m_\chi}{\rho_\text{WD} R_\text{WD}}.
\end{equation}

The DM will begin steadily accumulating at $R_\text{th}$ after a time $t_1 + t_2$.
Once the collected mass of DM at the thermal radius exceeds the WD mass within this volume, there is the possibility of self-gravitational collapse.
The time to collect a critical number $N_\text{sg}$ of DM particles is
\begin{align}
\label{eq:Ncore}
    t_\text{sg} &\sim \frac{N_\text{sg}}{\Gamma_\text{cap}}  \sim
    \frac{\rho_\text{WD} R^3_\text{th}}{m_\chi \Gamma_\text{cap}} \\
    &\approx 10^{10} ~\text{s} \l \frac{m_\chi}{10^{16} ~\GeV} \r^{-1/2}
    \l \frac{\sigma_{\chi A}}{10^{-35} ~\cm^2} \r^{-1}, \nonumber
\end{align}
Note that when $m_\chi > 10^{21} ~\GeV$, the number of particles necessary for self-gravitation $N_\text{sg}$ as defined in \eqref{eq:Ncore} is less than 2.
In this case we should properly take $N_\text{sg} = 2$. 
We do not consider such a DM ``core" collapse. 
Typically, the timescale for collapse is then set by the DM sphere's ability to cool and shed gravitational potential energy.
This is initially just $t_2$, while the time to collapse at any given radius $r$ decreases once the DM velocity rises again above $v_\ion$
\begin{align}
  t_\text{cool} &\sim t_2 \text{min}\{v_\text{ion}/v_\chi,1\} \\
  v_\chi &\sim \sqrt{\frac{G N m_\chi}{r}}, \nonumber
\end{align}
where $N$ is the number of collapsing DM particles.

There is a subtlety that arises in this DM core collapse for the large masses $m_\chi$ of interest to us.
The time to collect a self-gravitating number of particles $t_\text{sg}$ decreases for larger DM masses.
However, the dynamics of the collapse is set by the cooling time, which is initially $t_\text{cool} \propto m_\chi$.
For $m_\chi > 10^{15} ~\GeV$, the collection time may be shorter than the cooling time $t_\text{sg} < t_\text{cool}$ (depending on the cross section). 
In fact, the collection time may even be shorter than the dynamical time $t_\text{ff}$. 
If $t_\text{ff} < t_\text{sg} <t_\text{cool}$, the DM core will be driven to shrink because of the gravitational potential of the over-collecting DM.
The timescale for the shrinking is set by the capture rate of DM.
Ultimately, the collapsing DM core will consist of $N_\text{sg}$ enveloped in a ``halo" of $\Gamma_\text{cap} t_\text{cool} \gg N_\text{sg}$ particles, which will also proceed to collapse.
If instead $t_\text{sg} < t_\text{ff} <t_\text{cool}$, the DM core will rapidly accumulate to this large number before dynamically adjusting. 
For the purpose of the collapse constraints on DM annihilation, if $t_\x{sg} < t_\text{cool}$ we will simply assume a number of collapsing particles $N = \Gamma_\text{cap} t_\text{cool}$. 