Here we give a more detailed derivation of the capture rate of DM in a WD. 
For the DM to ultimately be captured, it must lose energy $\sim m_\chi v^2$, where $v$ is the DM velocity (in the rest frame of the WD) asymptotically far away.
Since typically $v \ll v_\text{esc}$, the DM has initial velocity $v_\text{esc}$ in the star and must lose a fraction $(v/v_\text{esc})^2$ of its energy to become captured. 
Properly, this DM velocity $v$ is described by a boosted Maxwell distribution peaked at the galactic virial velocity $v_\text{halo} \sim 10^{-3}$.
However, it can be shown \cite{Gould:1987ir} that such a boost does not affect the velocity distribution by more than $\OO(1)$ factors compared to the ordinary Maxwell distribution.
Thus the distribution can be approximated by (ignoring DM velocities in the exponential Boltzmann tail):
\begin{equation}
\frac{dn_\chi}{dv} \approx
\begin{cases}
  \frac{\rho_\chi}{m_\chi} \l \frac{v^2}{v_\text{halo}^3} \r  & v \leq v_\text{halo} \\
  0 & v > v_\text{halo}
  \end{cases}.
\end{equation} 
The DM capture rate is given by an integral of the DM transit rate weighted by a probability for capture $P_\text{cap}$
\begin{equation}
\Gamma_\text{cap} \sim \int dv \frac{d \Gamma_\text{trans}}{dv} P_\text{cap}(v),
\end{equation}
where the (differential) rate of DM transit through the WD is parametrically
\begin{equation}
\frac{d \Gamma_\text{trans}}{dv} \approx \frac{d n_\chi}{dv} R_\text{WD}^2 \l \frac{v_\text{esc}}{v}\r^2 v.
\end{equation}
$P_\text{cap}$ depends on the \emph{average} number of scatters in a WD
\begin{equation}
\bar{N}_\text{scat} \sim n_\text{ion} \sigma_{\chi A} R_\text{WD},
\end{equation}
and the number of scatters \emph{needed} for capture based on kinematics
\begin{equation}
N_\text{cap} \sim \text{max}\left \{1, \frac{m_\chi v^2}{m_\text{ion} v_\text{esc}^2}\right \},
\end{equation}
and is most generally expressed as a Poisson sum
\begin{equation}
P_\text{cap} = 1 - \sum^{N_\text{cap}-1}_{n=0} \exp(-\bar{N}_\text{scat})\frac{(\bar{N}_\text{scat})^n}{n!}.
\end{equation}
For our purposes we will approximate the sum as follows:
\begin{equation}
P_\text{cap} \approx 
\begin{cases}
 1 & \bar{N}_\text{scat} > N_\text{cap} \\
 \bar{N}_\text{scat} & \bar{N}_\text{scat} < N_\text{cap} ~\text{and}~ N_\text{cap} = 1 \\
 0 & \text{else}
\end{cases}.
\end{equation}
More properly, the DM capture rate should be computed numerically, e.g. see \cite{Bramante:2017xlb} for a more detailed calculation. 
However with the above simplifications, we find that it is parametrically of order
\begin{equation}
\Gamma_\text{cap} \sim \Gamma_\text{trans} \cdot \text{min}\left\{1, \bar{N}_\text{scat} \text{min}\{K,1\}\right\}, ~~~~ K \equiv \frac{m_\text{ion} v_\text{esc}^2}{m_\chi v_\text{halo}^2}. 
\end{equation}
Note that for ultra-heavy DM $m_\chi > 10^{15} ~\GeV$, it is necessarily the case that $K \ll 1$, i.e. multiple scatters are needed to capture most of the DM which transits the WD. 

